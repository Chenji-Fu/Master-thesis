\chapter{Introduction}

Let $F$ be a non-archimedean local field. Let $G$ be a reductive group over $F$. For simplicity, we assume $G$ is split and simply connected (in particular, semisimple). Let $\Lambda=\overline{\mathbb{Z}_{\ell}}$. The categorical local Langlands conjecture predicts that there is a fully faithful embedding
$$\Rep_{\Lambda}(G(F)) \longrightarrow \QCoh(Z^1(W_F, \hat{G})_{\Lambda}/\hat{G})$$
from the category of smooth representations of the $p$-adic group $G(F)$ to the category of quasi-coherent sheaves on the stack of Langlands parameters. In this paper, we compute the two sides explicitly for depth-zero regular supercuspidal part of a split simply connected group $G$ and verify the categorical local Langlands conjecture for depth-zero supercuspidal part of $GL_n$.

Fix an irreducible depth-zero regular supercuspidal representation $\pi \in \Rep_{\overline{\mathbb{F}_{\ell}}}(G(F))$, the (classical) local Langlands conjecture predicts that it should correspond to a tame, regular semisimple, elliptic $L$-parameter (TRSELP for short) $\varphi \in Z^1(W_F, \hat{G}(\overline{\mathbb{F}_{\ell}}))$ (See \cite{debacker2009depth}). As mentioned above, this paper focus on the depth-zero regular supercuspidal part of the categorical local Langlands conjecture, which predicts that there is a fully faithful embedding
$$\Rep_{\Lambda}(G(F))_{[\pi]} \longrightarrow \QCoh([X_{\varphi}/\hat{G}])$$
from the block of $\Rep_{\Lambda}(G(F))$ containing $\pi$ to the category of quasi-coherent sheaves on the connected component $[X_{\varphi}/\hat{G}]$ of the stack of $L$-parameters.

\section{$L$-parameter side}
In this section, we explain Chapter \ref{Chapter MoLP} on how to compute $\QCoh([X_{\varphi}/\hat{G}])$. 

This is done by describing $[X_{\varphi}/\hat{G}]$ explicitly as a quotient stack over $\Lambda=\overline{\mathbb{Z}_{\ell}}$. 

\subsection{Heuristics on the component $[X_{\varphi}/\hat{G}]$}

In this subsection, we describe some heuristics on the component $[X_{\varphi}/\hat{G}]$ which help us to guess what this component should look like.

First, let us recall what is known over $\overline{\mathbb{Q}_{\ell}}$ instead of $\Lambda=\overline{\mathbb{Z}_{\ell}}$. Indeed, assuming $G$ simply connected, the connected component of the stack of $L$-parameters over $\overline{\mathbb{Q}_{\ell}}$ containing an elliptic $L$-parameter $\varphi'$ is known to be one point. More precisely, it is isomorphic to the quotient stack $[*/S_{\varphi'}]$, where $S_{\varphi'}=C_{\hat{G}}(\varphi')$ is the centralizer of $\varphi'$ (See \cite[Section X.2]{fargues2021geometrization}).

Second, let us explain what is the difference between the geometry of the connected components of the stack of $L$-parameters over $\overline{\mathbb{Q}_{\ell}}$ and $\overline{\mathbb{Z}_{\ell}}$. This can be seen from the example $G=GL_1$. Indeed, 
$$Z^1(W_F, \widehat{GL_1}) \cong \mu_{q-1} \times \mathbb{G}_m,$$
both over $\overline{\mathbb{Q}_{\ell}}$ and $\overline{\mathbb{Z}_{\ell}}$ (See Example \ref{Example GL_1}). However, $\mu_{q-1}$ is just $q-1$ discrete points over $\overline{\mathbb{Q}_{\ell}}$, while
the connected components of $\mu_{q-1}$ are isomorphic to $\mu_{\ell^k}$ (over $\overline{\mathbb{F}_{\ell}}$, hence also) over $\overline{\mathbb{Z}_{\ell}}$, where $k$ is maximal such that $\ell^k$ divides $q-1$. So when describing the connected components of the stack of $L$-parameters over $\overline{\mathbb{Z}_{\ell}}$, there will be possibly some non-reduced part $\mu$ appearing.

These two features come together in the description of the connected component $[X_{\varphi}/\hat{G}]$. Under mild assumptions, we prove that
$$[X_{\varphi}/\hat{G}] \cong [*/S_{\varphi}]\times \mu,$$
where $S_{\varphi}=C_{\hat{G}}(\varphi)$ (See Theorem \ref{Thm X/G}).

\subsection{Ingredients of the computation}

The computation follows the theory of moduli space of Langlands parameters developed in \cite[Section 2 and Section 4]{dhkm2020moduli} (See also \cite[Section 3 and Section 4]{dat2022ihes} for a more gentle introduction). It is very helpful to do the example of $GL_2$ first (See Chapter \ref{Chapter GL_n}).

To compute the component $[X_{\varphi}/\hat{G}]$ over $\overline{\mathbb{Z}_{\ell}}$, let us fix a lift $\psi \in Z^1(W_F, \hat{G}(\overline{\mathbb{Z}_{\ell}}))$ of $\varphi \in Z^1(W_F, \hat{G}(\overline{\mathbb{F}_{\ell}}))$.

Recall by \cite[Subsection 4.6]{dat2022ihes},
$$X_{\varphi}=X_{\psi} \simeq \left(\hat{G} \times Z^1(W_F, N_{\hat{G}}(\psi_{\ell}))_{\psi_{\ell}, \overline{\psi}}\right)/C_{\hat{G}}(\psi_{\ell})_{\overline{\psi}},$$
where $Z^1(W_F, N_{\hat{G}}(\psi_{\ell}))_{\psi_{\ell}, \overline{\psi}}$  denotes the space of cocycles whose restriction to $I_F^{\ell}$ equals $\psi_{\ell}$ and whose image in $Z^1(W_F, \pi_0(N_{\hat{G}}(\psi_{\ell})))$ is $\overline{\psi}$. 

Here, $Z^1(W_F, N_{\hat{G}}(\psi_{\ell}))_{\psi_{\ell}, \overline{\psi}}$ is essentially the space of cocycles of the torus $T:=N_{\hat{G}}(\psi_{\ell})^0=C_{\hat{G}}(\psi_{\ell})$ by our TRSELP assumption (See Definition \ref{Def TRSELP}) and that $C_{\hat{G}}(\psi_{\ell})$ is generalized reductive, hence split over $\overline{\mathbb{Z}_{\ell}}$ (See Lemma \ref{Lem gen red}). Since it's not hard to compute the space of tame cocycles of a commutative group scheme using the explicit presentation of the tame Weil group (See \ref{Equation presentation of the tame Weil group} and \ref{Equation space of tame cocycle}), we obtain that 
$$Z^1(W_F, N_{\hat{G}}(\psi_{\ell}))_{\psi_{\ell}, \overline{\psi}} \cong T \times \mu,$$
where $\mu$ is a product of $\mu_{\ell^{k_i}}$'s (See Theorem \ref{Thm X} for details).
And it is not hard to see 
$$C_{\hat{G}}(\psi_{\ell})_{\overline{\psi}}=C_{\hat{G}}(\psi_{\ell})=T.$$

Therefore, we get 
$$X_{\varphi} \cong \left(\hat{G} \times T \times \mu\right)/T.$$
One need to be a bit careful about the $T$ action on $T$, because here a twist by $\psi(\Fr)$ is involved. Then one could compute that 
$$[X_{\varphi}/\hat{G}] \cong [\left(T \times \mu\right)/T] \cong [T/T] \times \mu,$$
where $T$ acts on $T$ via twisted conjugacy. Then we could work in the category of diagonalizable group schemes (whose structure is clear, see \cite[p70, Section 5]{brochard2014autour}) to identify $[T/T]$ with $[*/S_{\varphi}]$ under mild conditions.

\section{Representation side}
In this section, we explain Chapter \ref{Chapter Rep} on how to compute $\Rep_{\Lambda}(G(F))_{[\pi]}$.

\subsection{Equivalence to the block of a finite group of Lie type}

Recall that a depth-zero regular supercuspidal representation is of the form
$$\pi=\cInd_{G_x}^{G(F)}\rho$$
for some representation $\rho$ of the parahoric subgroup $G_x$ corresponding to a vertex $x$ in the Bruhat-Tits building of $G$ over $F$. Moreover, $\rho$ is the inflation of some regular cuspidal representation $\overline{\rho}$ of the finite group of Lie type $\overline{G_x}:=G_x/G_x^+$.

Let $\mathcal{A}_{x,1}$ denote the block $\Rep_{\Lambda}(\overline{G_x})_{[\overline{\rho}]}$ of $\Rep_{\Lambda}(\overline{G_x})$ containing $\overline{\rho}$. Similarly, let 
$$\mathcal{B}_{x,1}:=\Rep_{\Lambda}(G_x)_{[\rho]},\qquad \mathcal{C}_{x,1}:=\Rep_{\Lambda}(G(F))_{[\pi]}.$$

$\mathcal{A}_{x,1}$ admits an explicit description as being equivalent to a block of a finite torus via Broué's equivalence \ref{Thm Broué}. And it is not hard to show that the inflation induces an equivalence of categories $\mathcal{A}_{x,1} \cong \mathcal{B}_{x,1}$.

The main theorem we proved for the representation side is Theorem \ref{Thm Main}: the compact induction induces an equivalence of categories
$$\cInd_{G_x}^{G(F)}: \mathcal{B}_{x,1} \to \mathcal{C}_{x,1}.$$
Once this is proven, then $\mathcal{C}_{x,1} \cong \mathcal{A}_{x,1}$, hence admits an explicit description. The proof of the Theorem \ref{Thm Main} occupies the most of Chapter \ref{Chapter Rep}. 

\subsection{Proof of the main theorem for the representation side}

In the rest of the section, let us briefly explain the idea of the proof of Theorem \ref{Thm Main}.

The fully faithfulness of 
$$\cInd_{G_x}^{G(F)}: \mathcal{B}_{x,1} \to \mathcal{C}_{x,1}$$
is a usual computation by Frobenius reciprocity and Mackey's formula. Since a similar computation will be used later, we record it as a Theorem \ref{Thm Hom}. The key point is that 
$$\Hom_G(\cInd_{G_x}^{G(F)}\rho_1, \cInd_{G_y}^{G(F)}\rho_2)$$
could be computed explicitly assuming one of $\rho_1, \rho_2$ has cuspidal reduction. 

There is a little subtlety that we want not only $\rho$ to have cuspidal reduction, but also any representation $\rho' \in \mathcal{B}_{x,1}$ to have cuspidal reduction. This is the content of Theorem \ref{Thm Cusp Red}. And this is why we need the \textbf{regular} supercuspidal assumption. To prove Theorem \ref{Thm Cusp Red}, we make a digression on regular cuspidal blocks of finite groups of Lie type in Section \ref{Sec Reg Cusp}. As the \textbf{regular} assumption is impulsed here on purpose to make the result to hold, the readers are welcome to skip Section \ref{Sec Reg Cusp} and accept Theorem \ref{Thm Cusp Red} in a first reading.

The difficulty lies in proving
$$\cInd_{G_x}^{G(F)}: \mathcal{B}_{x,1} \to \mathcal{C}_{x,1}$$
is essentially surjective. For this, we prove that an explicit compact induction $$\Pi_{x,1}:=\cInd_{G_x}^{G(F)}\sigma_{x,1}$$ 
is a projective generator of $\mathcal{C}_{x,1}$. 

The first key is that $\Pi_{x,1}$ is a summand of a projective generator of a larger category. Indeed, $\Pi_{x,1}$ is a summand of
$$\Pi:=\cInd_{G_x^+}^{G(F)}\Lambda,$$
which is known to be a projective generator of the category $\Rep_{\Lambda}(G(F))_0$ of depth-zero representations, i.e., 
$$\Pi=\Pi_{x,1} \oplus \Pi^{x,1}.$$

The second key is that the complement $\Pi^{x,1}$ doesn't interfere with $\Pi_{x,1}$. More precisely, we could compute using Theorem \ref{Thm Hom} that 
$$\Hom_{G}(\Pi^{x,1}, \Pi_{x,1})=\Hom_{G}(\Pi_{x,1}, \Pi^{x,1})=0.$$

The above two keys allow us to conclude that $\Pi_{x,1}$ is a projective generator of $\mathcal{C}_{x,1}$. 




\section{The example of $GL_n$}

To illustrate the theory, we do the example of $GL_n$ in Chapter \ref{Chapter GL_n}. It is quite concise once we have the theories developed in Chapter \ref{Chapter MoLP} and \ref{Chapter Rep}, so let us don't say anything more here. However, the readers are encouraged to do the example of $GL_2$ throughout the paper, which will definitely help to understand the theory. Actually, the author always do the example of $GL_2$ first, and then figure out the theory for general group $G$.

\section{The categorical local Langlands conjecture for $GL_n$}

As an application, we could deduce the categorical local Langlands conjecture for depth-zero supercuspidal blocks of $GL_n$ (notice supercuspidal implies regular supercuspidal automatically in the $GL_n$ case) in Chapter \ref{Chapter CLLC}.

\section{Acknowledgements}

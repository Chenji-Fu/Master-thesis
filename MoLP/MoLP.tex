\documentclass{article}

\special{dvipdfmx:config z 0}

\usepackage{amsmath,amssymb,amsfonts,amsthm,extarrows}
\usepackage{mathtools}
\usepackage{enumitem}
\usepackage{stmaryrd}
\usepackage{tikz-cd} 
\usepackage{bbm}

\usepackage{color}
\newcommand{\red}[1]{\textcolor{red}{#1}}
\newcommand{\blue}[1]{\textcolor{blue}{#1}}

\usepackage{nameref}

\usepackage{graphicx}
\graphicspath{ {./images/} }

\usepackage{soul}

%%%% todo notes %%%%
\usepackage[colorinlistoftodos,textsize=footnotesize]{todonotes}
\setlength{\marginparwidth}{2.5cm}
\newcommand{\leftnote}[1]{\reversemarginpar\marginnote{\footnotesize #1}}
\newcommand{\rightnote}[1]{\normalmarginpar\marginnote{\footnotesize #1}\reversemarginpar}


\usepackage[colorlinks]{hyperref}

\newtheorem*{remark}{Remark}
\newtheorem{theorem}{Theorem}
\newtheorem{lemma}{Lemma}
\newtheorem{question}{Question}
\newtheorem{answer}{Answer}
\newtheorem{proposition}{Proposition}
\newtheorem{definition}{Definition}
\newtheorem{exer}{Exercise}
\newtheorem{corollary}{Corollary}
\newtheorem{example}{Example}
\newtheorem{warning}{Warning}

\DeclareMathOperator{\cInd}{\operatorname{c-Ind}}
\DeclareMathOperator{\Ind}{\operatorname{Ind}}
\newcommand{\Res}{\operatorname{Res}}
\newcommand{\Hom}{\operatorname{Hom}}
\newcommand{\Rep}{\operatorname{Rep}}
\newcommand{\End}{\operatorname{End}}
\newcommand{\GL}{\operatorname{GL}}
\newcommand{\diag}{\operatorname{diag}}
\newcommand{\Mod}{\operatorname{Mod}}
\newcommand{\Irr}{\operatorname{Irr}}
\newcommand{\Modr}{\operatorname{Mod-}}
\newcommand{\Modl}{\operatorname{-Mod}}
\newcommand{\Perf}{\operatorname{Perf}}
\newcommand{\Spec}{\operatorname{Spec}}
\newcommand{\Ob}{\operatorname{Ob}}



\begin{document}
	
	\section{Description of the connected component $X_{\varphi}$ containing a TRSELP $\varphi$}
	
	\subsection{Recollections on the moduli space of Langlands parameters}
	
	Let $p \neq 2$ be a fixed prime number and $\ell \neq 2$ be a prime number different from $p$. Let $F$ ba a non-archimedean local field with residue characteristic $q=p^r$ for some $r \in \mathbb{Z}_{\geq 1}$. Let $W_F$ be the Weil group of $F$, $I_F \subset W_F$ be the inertia subgroup, $P_F$ be the wild inertia subgroup. Fix $Fr \in W_F$ any lift of the arithmetic Frobenius. Let $W_t:=W_F/P_F$ be the tame Weil group. Let $I_t:=I_F/P_F$ be the tame inertia subgroup. I will abuse the notation and denote $Fr$ the image of $Fr$ in $W_t$. Then $W_t \simeq (I_t \rtimes <Fr>)$, where $<Fr> \simeq \mathbb{Z}$ is the subgroup generated by $Fr$. Here $I_t$ is non-canonically isomorphic to $\prod_{p'\neq p}\mathbb{Z}_{p'}$, which is procyclic. We fix such an isomorphism. And we fix a topological generator $s_0$ of $I_t$. For example, we can choose $s_0$ which corresponds to $(1, 1, ...)$ under the chosen isomorphism $I_t \simeq \prod_{p'\neq p}\mathbb{Z}_{p'}$. Let us recall the following important relation in $I_F/P_F$:
	$$Fr.s_0.Fr^{-1}=s_0^q.$$
	In fact, this is true for any $s \in I_t$ instead of $s_0$.
	
	Let $W_t^0:=<s_0, Fr>=\mathbb{Z}[1/p]^{s_0} \rtimes \mathbb{Z}^{Fr}$ be the subgroup of $W_t$ generated by $s_0$ and $Fr$. Denote $W_F^0 \subset W_F$ the preimage of $W_t^0$ under $W_F \to W_t$. This is known as the discretization of the Weil group. To summarize, $W_t^0$ is generated by two elements $Fr$ and $s_0$ with a single relation $Fr.s_0.Fr^{-1}=s_0^q$. 
	
	Let $G$ be a connected split reductive group over $F$. Let $\hat{G}$ be its dual group over $\mathbb{Z}$. Then the space of cocycles from the discretization
	\begin{equation}\label{explicit}
		Z^1(W_t^0, \hat{G})=\underline{Hom}(W_t^0, \hat{G})=\{(x, y) \in \hat{G} \times \hat{G} | yxy^{-1}=x^q\}
	\end{equation}
	is an explicit closed subscheme of $\hat{G} \times \hat{G}$ (See \cite[Section 3]{dat2022ihes}). An important fact (See \cite[Proposition 3.9]{dat2022ihes}) is that over a $\mathbb{Z}_{\ell}$-algebra $R$ (the cases $R=\overline{\mathbb{F}_{\ell}}, \overline{\mathbb{Z}_{\ell}}, \overline{\mathbb{Q}_{\ell}}$ are most relevant for us), the restriction from $W_t$ to $W_t^0$ induces an isomorphism
	$$Z^1(W_t, \hat{G}) \simeq Z^1(W_t^0, \hat{G}).$$ 
	Therefore, we can compute $Z^1(W_t, \hat{G})$ using the explicit formula \ref{explicit} above. This is fundamental for the analysis of the moduli space of Langlands parameters $Z^1(W_t, \hat{G})$. I refer the readers to \cite[Section 3 and Section 4]{dat2022ihes} for the precise definition and properties of $Z^1(W_t, \hat{G})$. 
	
	(\red{maybe add an example here})
	
	Let $I_F^{\ell}$ be the prime-to-$\ell$ inertia subgroup of $W_F$, i.e., $I_F^{\ell}:=ker(t_{\ell})$, where 
	$$t_\ell: I_F \to I_F/P_F \simeq \prod_{p' \neq p}\mathbb{Z}_{p'} \to \mathbb{Z}_\ell$$
	is the composition. In other words, it is the maximal subgroup of $I_F$ with pro-order prime to $\ell$. This property makes $I_F^{\ell}$ important when determining the connected components of $Z^1(W_F, \hat{G})$ over $\overline{\mathbb{Z}_{\ell}}$ (See \cite[Theorem 4.2 and Subsection 4.6]{dat2022ihes}). I assume the readers to be familiar with the moduli space of Langlands parameters, see for example \cite[Section 3 and Section 4]{dat2022ihes}, or \cite[Section 2 and Section 4]{dhkm2020moduli}. (\red{I could also recollect the theory in the appendix.})
	
	\subsection{Tame regular semisimple elliptic $L$-parameters}
	
	I want to define a class of $L$-parameters, called TRSELP, which roughly corresponds to depth-zero regular supercuspidal representations. Before that, let me define the concept of schematic centralizer, which will be used throughout the article.
	
	\begin{definition}[schematic centralizer]
	Let $H$ be an affine algebraic group over a ring $R$, $\Gamma$ be a finite group. Let $u \in Z^1(\Gamma, H(R'))$ be a $1$-cocycle for some $R$-algebra $R'$. Let $\alpha_u: H_{R'} \to Z^1(\Gamma, H)_{R'}, h \mapsto hu(-)h^{-1}$ be the orbit morphism. Then the schematic centralizer $C_H(u)$ is defined to be the fiber of $\alpha_u$ at $u$.
		
	\begin{tikzcd}
		{C_H(u)} \arrow[r, ""] \arrow[d, ""] & {H_{R'}} \arrow[d, "{\alpha_u}"] \\
		{R'} \arrow[r, "u"]                & {Z^1(\Gamma, H)_{R'}}               
	\end{tikzcd}
		
	\end{definition}
	
	One can show that $C_H(u)(R'')=C_{H(R'')}(u)$ is the set-theoretic centralizer for all $R'$-algebra $R''$, see for example \cite[Appendix]{dhkm2020moduli}.
	
	\begin{remark}
			Note this is enough for our applications where $\Gamma$ is more generally taken as a profinite group, because $u: \Gamma \to H$ usually factors through a finite quotient $\Gamma'$ of $\Gamma$.
	\end{remark}
	
	Let me now define a tame, regular semisimple, elliptic Langlands parameter (TRSELP for short) over $\overline{\mathbb{F}_{\ell}}$, roughly in the sense of \cite[Section 3.4 and Section 4.1]{debacker2009depth} in the case $G$ is $F$-split, but with $\overline{\mathbb{F}_{\ell}}$-coefficients instead of $\mathbb{C}$-coefficients.
	
	\begin{definition}
		A \textbf{tame regular semisimple elliptic $L$-parameter (TRSELP) over $\overline{\mathbb{F}_{\ell}}$} is a homomorphism $\varphi: W_F \to \hat{G}(\overline{\mathbb{F}_{\ell}})$ such that:
		\begin{enumerate}
			\item (smooth) $\varphi(I_F)$ is a finite subgroup of $\hat{G}(\overline{\mathbb{F}_{\ell}})$.
			\item (Frobenius semisimple) $\varphi(Fr)$ is a semisimple element of $\hat{G}(\overline{\mathbb{F}_{\ell}})$.
			\item (tame) The restriction of $\varphi$ to $P_F$ is trivial.
			\item \label{elliptic} (elliptic) The identity component of the centralizer $C_{\hat{G}}(\varphi)^0$ is equal to the identity component of the center $Z(\hat{G})^0$.
			\item \label{regular semisimple}(regular semisimple) The centralizer of the inertia $C_{\hat{G}}(\varphi|_{I_F})$ is a torus (in particular, connected).
		\end{enumerate}
	\end{definition}

    Concretely, a TRSELP consists of the following data:
    
    \begin{enumerate}
    	\item The restriction to the inertia $\varphi|_{I_F}$, which is a direct sum of characters of a finite abelian group since $I_F/P_F \simeq \varprojlim\mathbb{F}_{q^n}^*$. In particular, it factors through (the $\overline{\mathbb{F}_{\ell}}$-points of) some maximal torus, say $S$. Then regular semisimple means that $C_{\hat{G}(\overline{\mathbb{F}_{\ell}})}(\varphi(I_F))=S$.
    	\item The image of Frobenius $\varphi(Fr)$, which turns out to be an element of the normalizer $N_{\hat{G}(\overline{\mathbb{F}_{\ell}})}(S)$ (Since $Fr.s.Fr^{-1}=s^q \in I_t$ for any $s \in I_t$ implies that $\varphi(Fr)$ normalizes $C_{\hat{G}(\overline{\mathbb{F}_{\ell}})}(\varphi(I_F))=S$.). 
        And "elliptic" means that the center $Z(\hat{G})$ has finite index in the centralizer $C_{\hat{G}}(\varphi)$. As we will see later, ellipticity implies that $\hat{G}(\overline{\mathbb{F}_{\ell}})$ acts transitively on the connected component $X_{\varphi}(\overline{\mathbb{F}_{\ell}})$ of the moduli space of $L$-parameters containing $\varphi$, which is essential for the description
    	$$[X_{\varphi}/\hat{G}] \simeq [*/\underline{S_{\varphi}}]$$
    	where $S_\varphi=C_{\hat{G}(\overline{\mathbb{F}_{\ell}})}(\varphi(W_F))$ is the centralizer of the whole $L$-parameter $\varphi$.
    \end{enumerate}

    (\red{Maybe add an example here})

  \begin{remark}
  	\begin{enumerate}
  		\item Let $\overline{\Lambda} \in \{\overline{\mathbb{Z}_{\ell}}, \overline{\mathbb{Q}_{\ell}}, \overline{\mathbb{F}_{\ell}}\}$. It is important for my purpose to distinguish between the set-theoretic centralizer (for example, $C_{\hat{G}(\overline{\Lambda})}(\varphi(I_F))$) and the schematic centralizer (for example, $C_{\hat{G}}(\varphi)$). However, I might still use $\hat{G}$ to mean $\hat{G}(\overline{\Lambda})$ sometimes by abuse of notation, for which I hope the readers could recognize. One reason for that is that $\hat{G}$ is split over $\overline{\Lambda}$, hence $\hat{G}$ is completely determined by its $\overline{\Lambda}$-points. And many statements can either be phrased in terms of the $\overline{\Lambda}$-scheme or its $\overline{\Lambda}$-points (for example, \ref{elliptic} and \ref{regular semisimple}).
  		\item As we will see later, $S=C_{\hat{G}(\overline{\mathbb{F}_{\ell}})}(\varphi(I_F))$ turns out to be the $\overline{\mathbb{F}_{\ell}}$-points of the split torus $T=C_{\hat{G}}(\psi|_{I_F^{\ell}})$ for some lift $\psi$ of $\varphi$ over $\overline{\mathbb{Z}_{\ell}}$.
  	\end{enumerate}
  \end{remark}

\subsection{Description of the component}

Now given a TRSELP $\varphi \in Z^1(W_F, \hat{G}(\overline{\mathbb{F}_{\ell}}))$. Pick any lift $\psi \in Z^1(W_F, \hat{G}(\overline{\mathbb{Z}_{\ell}}))$ of $\varphi$, whose existence is ensured by the flatness of $Z^1(W_F, \hat{G})_{\overline{\mathbb{Z}_{\ell}}}$ (See Lemma \ref{Lem generalizing}). Let $\psi_{\ell}:=\psi|_{I_F^{\ell}}$ denotes the restriction of $\psi$ to the prime-to-$\ell$ inertia. Note that $\psi \in Z^1(W_F, \hat{G})$ factors through $N_{\hat{G}}(\psi_{\ell})$ (Since $I_F^{\ell}$ is normal in $W_F$). Let $\overline{\psi}$ denotes the image of $\psi$ in $Z^1(W_F, \pi_0(N_{\hat{G}}(\psi_{\ell})))$. Let $X_{\varphi}$ be the connected component of $Z^1(W_F, \hat{G})_{\overline{\mathbb{Z}_{\ell}}}$ containing $\varphi$. Note $X_{\varphi}$ also contains $\psi$ since $\psi$ specializes to $\varphi$. So we sometimes also denote $X_{\varphi}$ as $X_{\psi}$.

\begin{theorem}
	Let $\varphi \in Z^1(W_F, \hat{G}(\overline{\mathbb{F}_{\ell}}))$ be a TRSELP over $\overline{\mathbb{F}_{\ell}}$. Let $\psi \in Z^1(W_F, \hat{G}(\overline{\mathbb{Z}_{\ell}}))$ be any lifting of $\varphi$. Then at least when the center $Z(\hat{G})$ is smooth over $\overline{\mathbb{Z}_{\ell}}$, the connected component $X_{\varphi}=X_{\psi}$ of $Z^1(W_F, \hat{G})_{\overline{\mathbb{Z}_{\ell}}}$ containing $\varphi$ is isomorphic to 
	$$\left(\hat{G} \times C_{\hat{G}}(\psi_{\ell})^0 \times \mu\right)/C_{\hat{G}}(\psi_{\ell})_{\overline{\psi}},$$
	where
	\begin{enumerate}
		\item $C_{\hat{G}}(\psi_{\ell})^0$ is the identity component of the schematic centralizer $C_{\hat{G}}(\psi_{\ell})$, which turns out to be a split torus $T$ over $\overline{\mathbb{Z}_{\ell}}$ with $\overline{\mathbb{F}_{\ell}}$-points $S=C_{\hat{G}(\overline{\mathbb{F}_{\ell}})}(\varphi(I_F))$.
		\item $\mu$ is the connected component of $T^{Fr=(-)^q}$ (the subscheme of $T$ on which $Fr$ acts by raising to $q$-th power) containing $1$ (See \cite[Example 3.14]{dat2022ihes}), which is a product of some $\mu_{\ell^{k_i}}$ (the group scheme of $\ell^{k_i}$-th roots of unity over $\overline{\mathbb{Z}_{\ell}}$), $k_i \in \mathbb{Z}_{\geq 0}$. Note that $\mu$ could be trivial, depending on $\hat{G}$ and some congruence relations between $q, \ell$.
		\item $C_{\hat{G}}(\psi_{\ell})_{\overline{\psi}}$ is the (schematic) stabilizer (\red{definition see Appendix}) of $\overline{\psi}$ in $C_{\hat{G}}(\psi_{\ell})$.
	\end{enumerate}
    In other words, we have the following isomorphism of schemes over $\overline{\mathbb{Z}_{\ell}}$:
    $$X_{\varphi} \simeq \left(\hat{G} \times T \times \mu\right)/T.$$
    And we will specify in the next subsection what the $T$-action on $\left(\hat{G} \times T \times \mu\right)$ is.
    
    \begin{proof}
    	First, recall by \cite[Subsection 4.6]{dat2022ihes},
    	$$X_{\psi} \simeq \left(\hat{G} \times Z^1(W_F, N_{\hat{G}}(\psi_{\ell}))_{\psi_{\ell}, \overline{\psi}}\right)/C_{\hat{G}}(\psi_{\ell})_{\overline{\psi}},$$
    	where $Z^1(W_F, N_{\hat{G}}(\psi_{\ell}))_{\psi_{\ell}, \overline{\psi}}$  denotes the space of cocycles whose restriction to $I_F^{\ell}$ equals $\psi_{\ell}$ and whose image in $Z^1(W_F, \pi_0(N_{\hat{G}}(\psi_{\ell})))$ is $\overline{\psi}$. 
    	\red{Explanation:} Recall (See \cite[Subsection 4.6]{dat2022ihes}) first that the component $X_{\varphi}=X_{\psi}$ morally consists of the $L$-parameters whose restriction to $I_F^{\ell}$ is $\hat{G}$-conjugate to $\psi_{\ell}$ and whose image in $Z^1(W_F, \pi_0(N_{\hat{G}}(\psi_{\ell})))$ is $\hat{G}$-conjugate to $\overline{\psi}$. Hence $X_{\varphi}$ is isomorphic to 
    	$$(\hat{G} \times Z^1(W_F, N_{\hat{G}}(\psi_{\ell}))_{\psi_{\ell}, \overline{\psi}})/C_{\hat{G}}(\psi_{\ell})_{\overline{\psi}}$$
    	via $g\eta(-)g^{-1} \mapsfrom (g, \eta)$, with $C_{\hat{G}}(\psi_{\ell})_{\overline{\psi}}$ acting on $(\hat{G} \times Z^1(W_F, N_{\hat{G}}(\psi_{\ell}))_{\psi_{\ell}, \overline{\psi}})$ by 
    	$$(t, (g, \psi')) \mapsto (gt^{-1}, t\psi'(-)t^{-1}),$$
    	where $t \in C_{\hat{G}}(\psi_{\ell})_{\overline{\psi}}$ and $(g, \psi') \in (\hat{G} \times Z^1(W_F, N_{\hat{G}}(\psi_{\ell}))_{\psi_{\ell}, \overline{\psi}})$.
    	
    	Second, $\eta.\psi \mapsfrom \eta$ defines an isomorphism
    	$$Z^1(W_F, N_{\hat{G}}(\psi_{\ell}))_{\psi_{\ell}, \overline{\psi}} \simeq Z^1_{Ad(\psi)}(W_F, N_{\hat{G}}(\psi_{\ell})^0)_{1_{I_F^{\ell}}}=:Z^1_{Ad(\psi)}(W_F, N_{\hat{G}}(\psi_{\ell})^0)_1$$
    	where $Z^1_{Ad(\psi)}(W_F, N_{\hat{G}}(\psi_{\ell}))$ means the space of cocycles with $W_F$ acting on $N_{\hat{G}}(\psi_{\ell})$ via conjugacy action through $\psi$, and the subscript $1_{I_F^{\ell}}$ or $1$ means the cocycles whose restriction to $I_F^{\ell}$ is trivial. 
    	\red{Explanation:} This is clear by unraveling the definitions: two cocycles whose restriction to $I_F^\ell$ are both $\psi_{\ell}$ differ by something whose restriction to $I_F^{\ell}$ is trivial; two cocycles whose pushforward to $Z^1(W_F, \pi_0(N_{\hat{G}}(\psi_{\ell})))$ are both $\overline{\psi}$ differ by something whose pushforward to $Z^1(W_F, \pi_0(N_{\hat{G}}(\psi_{\ell})))$ is trivial, i.e., which factors through the identity component $N_{\hat{G}}(\psi_{\ell})^0$.
    	
    	Next, I show that $C_{\hat{G}}(\psi_{\ell})$ is a split torus over $\overline{\mathbb{Z}_{\ell}}$. By \cite[Subsection 3.1]{dat2022ihes}, the centralizer $C_{\hat{G}}(\psi_{\ell})$ is generalized reductive (See Lemma \ref{Lem gen red}), hence split over $\overline{\mathbb{Z}_{\ell}}$, and $N_{\hat{G}}(\psi_{\ell})^0=C_{\hat{G}}(\psi_{\ell})^0$. So we can determine $C_{\hat{G}}(\psi_{\ell})$ by computing its $\overline{\mathbb{F}_{\ell}}$-points. Indeed,
    	$$C_{\hat{G}}(\psi_{\ell})(\overline{\mathbb{F}_{\ell}})=C_{\hat{G}(\overline{\mathbb{F}_{\ell}})}(\varphi(I_F^\ell))=C_{\hat{G}(\overline{\mathbb{F}_{\ell}})}(\varphi(I_F)),$$
    	where the last equality follows since $I_F/I_F^{\ell}$ doesn't contribute to the image of $\varphi$ (See Lemma \ref{Lem I_F^ell}). Therefore, $C_{\hat{G}}(\psi_{\ell})$ is a split torus over $\overline{\mathbb{Z}_{\ell}}$ with $\overline{\mathbb{F}_{\ell}}$-points $S=C_{\hat{G}(\overline{\mathbb{F}_{\ell}})}(\varphi(I_F))$. Denote $T=C_{\hat{G}}(\psi_{\ell})$. In particular, $C_{\hat{G}}(\psi_{\ell})$  is connected, hence 
    	$$N_{\hat{G}}(\psi_{\ell})^0=C_{\hat{G}}(\psi_{\ell})^0=C_{\hat{G}}(\psi_{\ell})=T.$$
    	
    	Now we could compute
    	$$Z^1_{Ad(\psi)}(W_F, N_{\hat{G}}(\psi_{\ell})^0)=Z^1_{Ad(\psi)}(W_F, T) \simeq T \times T^{Fr=(-)^q},$$
    	where the last isomorphism is given by $\eta \mapsto (\eta(Fr), \eta(s_0))$, where $s_0 \in W_t^0$ is the topological generator of $I_t$ fixed before (See \cite[Example 3.14]{dat2022ihes}).
    	
    	Then we show that the identity component of $T^{Fr=(-)^q}$ gives $\mu$ in the statement of the theorem. Note $T^{Fr=(-)^q}$ is a diagonalizable group scheme over $\overline{\mathbb{Z}_{\ell}}$ of dimension zero (This can be seen either by $\dim Z^1(W_F/P_F, T)=\dim T$, or by noticing that $\eta(s_0) \in T^{Fr=(-)^q}$ is semisimple with finitely many possible eigenvalues), hence of the form $\prod_i\mu_{n_i}$ for some $n_i \in \mathbb{Z}_{\geq 0}$. Hence its connected component $(T^{Fr=(-)^q})^0$ over $\overline{\mathbb{Z}_{\ell}}$ is of the form $\prod_i\mu_{\ell^{k_i}},$ with $k_i$ maximal such that $\ell^{k_i}$ divides $n_i$. Therefore, 
    	$$Z^1_{Ad(\psi)}(W_F, N_{\hat{G}}(\psi_{\ell})^0)_1 \simeq (T \times T^{Fr=(-)^q})^0 \simeq T \times (T^{Fr=(-)^q})^0 \simeq T \times \mu,$$
    	(\red{See Lemma ? for the first isomorphism}) where $\mu$ is of the form $\prod_i\mu_{\ell^{k_i}}$.
    	
    	Finally, we show that $C_{\hat{G}}(\psi_{\ell})_{\overline{\psi}}=C_{\hat{G}}(\psi_{\ell})$. Recall $C_{\hat{G}}(\psi_{\ell})$ acts on $Z^1(W_F, N_{\hat{G}}(\psi_{\ell}))$ by conjugation, inducing an action of $C_{\hat{G}}(\psi_{\ell})$ on $Z^1(W_F, \pi_0(N_{\hat{G}}(\psi_{\ell}))).$ And $C_{\hat{G}}(\psi_{\ell})_{\overline{\psi}}$ is by definition the stabilizer of $\overline{\psi} \in Z^1(W_F, \pi_0(N_{\hat{G}}(\psi_{\ell})))$ in $C_{\hat{G}}(\psi_{\ell})$. Now $C_{\hat{G}}(\psi_{\ell})=T$ is connected, hence acts trivially on the component group $\pi_0(N_{\hat{G}}(\psi_{\ell}))$ (\red{See Lemma ?}), hence also acts trivially on $Z^1(W_F, \pi_0(N_{\hat{G}}(\psi_{\ell})))$. Therefore, the stabilizer $C_{\hat{G}}(\psi_{\ell})_{\overline{\psi}}=C_{\hat{G}}(\psi_{\ell})$.
    	
    	Above all, we have 
    	$$X_{\varphi} \simeq (\hat{G} \times Z^1_{Ad(\psi)}(W_F, N_{\hat{G}}(\psi_{\ell})^0)_1)/C_{\hat{G}}(\psi_{\ell})_{\overline{\psi}} \simeq (\hat{G} \times T \times \mu)/T.$$
    \end{proof}
\end{theorem}

\subsection{The $T$-action on $(\hat{G} \times T \times \mu)$}

For later use, let me make it explicit the $T$-action on $(\hat{G} \times T \times \mu)$.

Recall (See \cite[Subsection 4.6]{dat2022ihes}) first that the component $X_{\varphi}=X_{\psi}$ morally consists of the $L$-parameters whose restriction to $I_F^{\ell}$ is $\hat{G}$-conjugate to $\psi_{\ell}$ and whose image in $Z^1(W_F, \pi_0(N_{\hat{G}}(\psi_{\ell})))$ is $\hat{G}$-conjugate to $\overline{\psi}$. Hence $X_{\varphi}$ is isomorphic to 
$$(\hat{G} \times Z^1(W_F, N_{\hat{G}}(\psi_{\ell}))_{\psi_{\ell}, \overline{\psi}})/C_{\hat{G}}(\psi_{\ell})_{\overline{\psi}}$$
via $g\eta(-)g^{-1} \mapsfrom (g, \eta)$, with $C_{\hat{G}}(\psi_{\ell})_{\overline{\psi}}$ acting on $(\hat{G} \times Z^1(W_F, N_{\hat{G}}(\psi_{\ell}))_{\psi_{\ell}, \overline{\psi}})$ by 
$$(t, (g, \psi')) \mapsto (gt^{-1}, t\psi'(-)t^{-1}),$$
where $t \in C_{\hat{G}}(\psi_{\ell})_{\overline{\psi}} \simeq T$ and $(g, \psi') \in (\hat{G} \times Z^1(W_F, N_{\hat{G}}(\psi_{\ell}))_{\psi_{\ell}, \overline{\psi}})$.

Next, recall that $\eta.\psi \mapsfrom \eta \mapsto (\eta(Fr), \eta(s_0))$ defines isomorphisms
$$Z^1(W_F, N_{\hat{G}}(\psi_{\ell}))_{\psi_{\ell}, \overline{\psi}} \simeq Z^1_{Ad\psi}(W_F, N_{\hat{G}}(\psi_{\ell})^0)_1 \simeq T \times \mu.$$
Let's focus on the isomorphism $\eta.\psi \mapsfrom \eta$:
$$Z^1(W_F, N_{\hat{G}}(\psi_{\ell}))_{\psi_{\ell}, \overline{\psi}} \simeq Z^1_{Ad\psi}(W_F, N_{\hat{G}}(\psi_{\ell})^0)_1.$$
Recall that $T \subset \hat{G}$ acts on $Z^1(W_F, N_{\hat{G}}(\psi_{\ell}))_{\psi_{\ell}, \overline{\psi}}$ via conjugation. Hence the above isomorphism induces an $T$-action on $Z^1_{Ad\psi}(W_F, N_{\hat{G}}(\psi_{\ell})^0)_1$, by
$$(t, \eta) \mapsto (t(\eta\psi(-)) t^{-1})\psi^{-1}.$$

Hence in $(\hat{G} \times T \times \mu)/T$, we compute by tracking the above isomorphisms that 
\begin{enumerate}
	\item $T$ acts on $\hat{G}$ via $(t, g) \mapsto gt^{-1}$.
	\item $T=C_{\hat{G}}(\psi_{\ell})_{\overline{\psi}}$ acts on $T \subset (T \times \mu)$ (corresponds to $\eta(Fr)$) by twisted conjugacy (due to the isomorphisms $\eta.\psi \mapsfrom \eta \mapsto (\eta(Fr), \eta(s_0))$), i.e., 
	$$(t, t') \mapsto \left(t(t'n)t^{-1}\right)n^{-1}=tt'(nt^{-1}n^{-1})=t(nt^{-1}n^{-1})t'=(tnt^{-1}n^{-1})t',$$
	where $n=\psi(Fr)$; Note here $n$, a prior lies in $\hat{G}$, actually lies in $N_{\hat{G}}(T)$ (Since $Fr.s.Fr^{-1}=s^q$ implies that $\psi(Fr)$ normalizes $C_{\hat{G}}(\psi|_{I_F})=C_{\hat{G}}(\psi|_{I_F^{\ell}})=T$, \red{See Lemma ?}). To summarize, $t \in T$ acts on $T$ via multiplication by $tnt^{-1}n^{-1}$.
	\item $T$ acts trivially on $\mu$. This is because $\eta(s_0) \in T$ and $\psi(s_0) \in T$. (\red{See Lemma ?})
\end{enumerate}

On the other hand, recall we have the natural $\hat{G}$-action on $Z^1(W_F, \hat{G})$ by conjugation, hence the $\hat{G}$-action on this component $X_{\varphi}$. Under the isomorphism $X_{\varphi} \simeq (\hat{G} \times T \times \mu)/T$, the $\hat{G}$-action becomes
$$(g', (g, t, m)) \mapsto  (g'g, t, m), \text{ for any } g' \in \hat{G} \text{ and } (g, t, m) \in (\hat{G} \times T \times \mu)/T.$$

Note that the $T$-action and the $\hat{G}$-action on $(\hat{G} \times T \times \mu)$ commute with each other, we thus have the following:

\begin{proposition}\label{T times mu/T}
	$$[X_{\varphi}/\hat{G}]=\left[\left((\hat{G} \times T \times \mu)/T\right)/\hat{G}\right] \simeq \left[\left((\hat{G} \times T \times \mu)/\hat{G}\right)/T\right] \simeq [(T \times \mu)/T],$$ 
	with $t \in T$ acting on $T$ via multiplication by $tnt^{-1}n^{-1}$, and $t \in T$ acting trivially on $\mu$. 
\end{proposition}

\subsection{Some lemmas}

\begin{lemma}\label{Lem generalizing}
	Let $\varphi' \in Z^1(W_t, \hat{G}(\overline{\mathbb{F}_{\ell}}))$. Then there exists $\psi' \in Z^1(W_t, \hat{G}(\overline{\mathbb{Z}_{\ell}}))$ such that $\psi'$ is a lift of $\varphi'$.
\end{lemma}

\begin{proof}
	In the statement, $Z^1(W_t, \hat{G})$ is the abbreviation for $Z^1(W_t, \hat{G})_{\overline{\mathbb{Z}_{\ell}}}$. Recall that $Z^1(W_t, \hat{G}) \to \overline{\mathbb{Z}_{\ell}}$ is flat (See \cite[Proposition 3.3]{dat2022ihes}), hence generalizing (\red{See Stack Project, 01U2}). Therefore, given $\varphi' \in Z^1(W_t, \hat{G}(\overline{\mathbb{F}_{\ell}}))$, there exists $\xi \in Z^1(W_t, \hat{G}(\overline{\mathbb{Q}_{\ell}}))$ such that $\xi$ specializes to $\varphi'$. In other words, $ker(\xi) \subset ker(\varphi')$. I'm going to show that $\xi: W_t \to \hat{G}(\overline{\mathbb{Q}_{\ell}})$ factors through  $\hat{G}(\overline{\mathbb{Z}_{\ell}})$.
	
	This is true by the following more general statement: Let $Y=\Spec(R)$ be an affine scheme over $\overline{\mathbb{Z}_{\ell}}$, let $y_{\eta} \in Y(\overline{\mathbb{Q}_{\ell}})$ specializing to $y_s \in Y(\overline{\mathbb{F}_{\ell}})$.  Then $y_{\eta} \in Y(\overline{\mathbb{Q}_{\ell}})=\Hom(R, \overline{\mathbb{Q}_{\ell}})$ factors through $\overline{\mathbb{Z}_{\ell}}$.
	
    To prove the above statement, let $\mathfrak{p}:=\ker(y_\eta)$ and $\mathfrak{q}:=\ker(y_s)$ be the corresponding prime ideas. Then "$y_{\eta}$ specializes to $y_s$" translates to "$\mathfrak{p} \subset \mathfrak{q}$". Recall that we are going to show that $y_{\eta}: R \to \overline{\mathbb{Q}_{\ell}}$ factors through $\overline{\mathbb{Z}_{\ell}}$. We argue by contradiction. Otherwise there is some element $f \in R$ mapping to $\ell^{-m}u$ for some $m \in \mathbb{Z}_{\geq 1}$ and $u \in \overline{\mathbb{Z}_{\ell}}^*$. Hence 
    \begin{equation}\label{eq ell}
    	\ell^mu^{-1}f-1 \in ker(y_{\eta}) \subset ker(y_s).
    \end{equation}
    However, $\ell \in ker(y_s)$ since $y_s$ lives on the special fiber. This together with equation \ref{eq ell} implies that $1 \in ker(y_s)$. Contradiction!
\end{proof}

\begin{lemma}\label{Lem gen red}
	The schematic centralizer $C_{\hat{G}}(\psi_{\ell})$ is a generalized reductive group scheme over $\overline{\mathbb{Z}_{\ell}}$.
\end{lemma}

\begin{proof}
	To invoke \cite[Lemma 3.2]{dat2022ihes}, I first show that $$C_{\hat{G}}(\psi_{\ell})=C_{\hat{G}}(\psi(I_F^{\ell})),$$
	where $C_{\hat{G}}(\psi(I_F^{\ell}))$ is the schematic centralizer of the subgroup $\psi(I_F^{\ell}) \subset \hat{G}(\overline{\mathbb{Z}_{\ell}})$ in $\hat{G}$. This can be checked by Yoneda Lemma on $R$-valued points for any $\overline{\mathbb{Z}_{\ell}}$-algebra $R$.
	
	Then we could conclude by \cite[Lemma 3.2]{dat2022ihes}. Indeed, $\psi_{\ell}$ factors through some finite quotient $Q$ of $I_F^{\ell}$, which has order invertible in the base $\overline{\mathbb{Z}_{\ell}}$. So the conditions of \cite[Lemma 3.2]{dat2022ihes} are satisfied. 
	
	\red{Some explanations to use \cite[Lemma 3.2]{dat2022ihes}: } While \cite[Lemma 3.2]{dat2022ihes} is phrased in the setting that $R$ is a normal subring of a number field, it still works for $\overline{\mathbb{Z}_{\ell}} \subset \overline{\mathbb{Q}_{\ell}}$ instead of $\mathbb{Z} \subset \mathbb{Q}$ (\red{Why?}). There is also a small issue that $\overline{\mathbb{Z}_{\ell}}$ is not finite over $\mathbb{Z}_{\ell}$, but this can be resolved since everything is already defined over some sufficiently large finite extension $\mathcal{O}$ of $\mathbb{Z}_{\ell}$.
\end{proof}

\begin{lemma}\label{Lem I_F^ell}
	$$C_{\hat{G}}(\psi_{\ell})(\overline{\mathbb{F}_{\ell}})=C_{\hat{G}(\overline{\mathbb{F}_{\ell}})}(\varphi(I_F^\ell))=C_{\hat{G}(\overline{\mathbb{F}_{\ell}})}(\varphi(I_F)).$$
\end{lemma}

\begin{proof}
	The first equation is by definition (and that $C_{\hat{G}}(\psi_{\ell})$ represents the set-theoretic centralizer).
	
	For the second equation, note that $\varphi|_{I_t}=\gamma_1 + ...+ \gamma_d$ is a direct sum of characters (Since $I_t \simeq \prod_{p'\neq p}\mathbb{Z}_{p'}$), so it suffices to show that each $\gamma_i$ is trivial on the summand $\mathbb{Z}_{\ell}$ of $I_t\simeq \prod_{p'\neq p}\mathbb{Z}_{p'}$.
	Indeed,
	$$\Hom_{Cont}(\mathbb{Z}_{\ell}, \overline{\mathbb{F}_{\ell}}^*)=\Hom_{Cont}(\varprojlim\mathbb{Z}/\ell^n\mathbb{Z}, \overline{\mathbb{F}_{\ell}}^*)=\varinjlim\Hom(\mathbb{Z}/\ell^n\mathbb{Z}, \overline{\mathbb{F}_{\ell}}^*)=\{1\}.$$
\end{proof}

\section{Main Theorem: description of $[X_{\varphi}/\hat{G}]$}

Let $F$ be a non-archimedean local field, $G$ be a connected split reductive group over $F$. Let $\varphi \in Z^1(W_F, \hat{G}(\overline{\mathbb{F}_{\ell}}))$ be a tame, regular semisimple, elliptic $L$-parameter (TRSELP for short). Recall that this means that the centralizer 
$$C_{\hat{G}(\overline{\mathbb{F}_{\ell}})}(\varphi(I_F)) =: S \subset \hat{G}(\overline{\mathbb{F}_{\ell}})$$ 
is a maximal torus, and $\varphi(Fr) \in N_{\hat{G}}(S)$ gives rise to an element $w=\overline{\varphi(Fr)} \in N_{\hat{G}}(S)/S$ in the Weyl group (and that $\varphi$ is tame and elliptic). (\red{Maybe add a remark on the relation between $\varphi(Fr), n=\psi(Fr),$ and $w$}) 

Assume that 
\begin{enumerate}
	\item[(assumption 1)] \label{assumption 1} The center $Z(\hat{G})$ is smooth over $\overline{\mathbb{Z}_{\ell}}$.
	\item[(assumption 2)] \label{assumption 2} $Z(\hat{G})$ is finite.
%	\item \label{assumption 3} $\ell$ doesn't divide the order of $w=\overline{\varphi(Fr)}$ in the Weyl group $N_{\hat{G}}(S)/S$.
\end{enumerate}

Let $\psi \in Z^1(W_F, \hat{G}(\overline{\mathbb{Z}_{\ell}}))$ be any lifting of $\varphi$. Let $\psi_{\ell}$ denotes the restriction $\psi|_{I_F^{\ell}}$, and $\overline{\psi}$ denotes the image of $\psi$ in $Z^1(W_F, \pi_0(N_{\hat{G}}(\psi_{\ell})))$. Recall that the schematic centralizer $C_{\hat{G}}(\psi_{\ell})=T$ is a split torus over $\overline{\mathbb{Z}_{\ell}}$ with $\overline{\mathbb{F}_{\ell}}$-points $C_{\hat{G}(\overline{\mathbb{F}_{\ell}})}(\varphi(I_F)) = S$. 

For later use, I record the following lemma -- $w$ can also be defined in terms of $\psi$ instead of $\varphi$. This is helpful because we will reduce to a computation on the special fiber later. First, notice that since $T$ is a split torus over $\overline{\mathbb{Z}_{\ell}}$ with $\ell \neq 2$, we can identify
$$\left(N_{\hat{G}}(T)/T\right)(\overline{\mathbb{Z}_{\ell}}) \simeq \left(N_{\hat{G}}(T)/T\right)(\overline{\mathbb{F}_{\ell}}),$$
and denote it by $\Omega$. (\red{See Lemma \ref{Lem Wely} below}) 

\begin{remark}
	Lemma \ref{Lem Wely} below shows that $N_{\hat{G}}(T)/T$ is representable by a group scheme which is split over $\overline{\mathbb{Z}_{\ell}}$. Therefore, we will slightly abuse notations and use $\Omega, N_{\hat{G}}(T)/T, N_{\hat{G}}(S)/S$ interchangeably.
\end{remark}

\begin{lemma}
	The image of $\varphi(Fr)$ and $\psi(Fr)$ in the Weyl group $\Omega$ agree, hence giving a well defined element $w$ in the Weyl group $\Omega$. (\red{Check carefully!})
\end{lemma}

\begin{proof}
	Let 
	$$\Omega=\left(N_{\hat{G}}(T)/T\right)(\overline{\mathbb{Z}_{\ell}}) = \left(N_{\hat{G}}(T)/T\right)(\overline{\mathbb{F}_{\ell}})$$ 
	as above and $\underline{\Omega}$ be the associated constant group scheme. Since $\psi$ is a lift of $\varphi$, $\psi(Fr)$ specializes to $\varphi(Fr)$ in $N_{\hat{G}}(T)$. Then the lemma follows since 
	$$N_{\hat{G}}(T) \to N_{\hat{G}}(T)/T=\underline{\Omega}$$
	is a morphism of schemes, hence the following diagram commutes:
	
	\begin{tikzcd}
		{N_{\hat{G}}(T)(\overline{\mathbb{Z}_\ell})} && {N_{\hat{G}}(T)(\overline{\mathbb{F}_\ell})} \\
		\\
		{\underline{\Omega}(\overline{\mathbb{Z}_\ell})=\Omega} && {\underline{\Omega}(\overline{\mathbb{F}_\ell})=\Omega}
		\arrow[from=1-1, to=1-3]
		\arrow[from=1-1, to=3-1]
		\arrow[from=3-1, to=3-3]
		\arrow[from=1-3, to=3-3]
	\end{tikzcd}

\end{proof}

Our main theorem is the following.

\begin{theorem}
	Let $X_{\varphi}$ ($= X_{\psi}$) be the connected component of $Z^1(W_F, \hat{G})_{\overline{\mathbb{Z}_{\ell}}}$ containing $\varphi$ (hence also containing $\psi$). The we have isomorphisms of quotient stacks
	$$[X_{\varphi}/\hat{G}] \simeq [(T \times \mu)/T] \simeq [*/{C_T(n)}] \times \mu,$$
	where $C_T(n)$ is the schematic centralizer of $n=\psi$ in $T=C_{\hat{G}}(\psi|_{I_F^{\ell}})$, and $\mu=\prod_{i=1}^m\mu_{\ell^{k_i}}$ for some $k_i \in \mathbb{Z}_{\geq 1}$, $m \in \mathbb{Z}_{\geq 0}$ is a product of group schemes of roots of unity. 
	
	If moreover assume
	\item[(assumption 3)] \label{assumption 3} $\ell$ doesn't divide the order of $w=\overline{\varphi(Fr)}$ in the Weyl group $N_{\hat{G}}(S)/S$;\\
	then 
	$$[X_{\varphi}/\hat{G}] \simeq [(T \times \mu)/T] \simeq [*/\underline{S_{\varphi}}] \times \mu,$$
	where $S_{\varphi}=C_{\hat{G}}(\overline{\mathbb{F}_{\ell}})(\varphi(W_F))$, and $\underline{S_{\varphi}}$ is the corresponding constant group scheme.
\end{theorem}

\begin{proof}
	Recall that $X_{\varphi}$ is isomorphic to the contracted product 
	$$(\hat{G} \times Z^1(W_F, N_{\hat{G}}(\psi_{\ell}))_{\psi_{\ell}, \overline{\psi}})/C_{\hat{G}}(\psi_{\ell})_{\overline{\psi}},$$ 
	and that $\eta.\psi \mapsfrom \eta \mapsto (\eta(Fr), \eta(s_0))$ induces isomorphisms
	$$Z^1(W_F, N_{\hat{G}}(\psi_{\ell}))_{\psi_{\ell}, \overline{\psi}} \simeq Z^1_{Ad\psi}(W_F, N_{\hat{G}}(\psi_{\ell})^0)_1 \simeq T \times \mu.$$
	
	This implies that $[X/\hat{G}] \simeq [(T \times \mu)/T]$ with $T$ acting on $T$ by twisted conjugacy:
	$$(t, t') \mapsto \left(t(t'n)t^{-1}\right)n^{-1}=tt'(nt^{-1}n^{-1})=t(nt^{-1}n^{-1})t'=(tnt^{-1}n^{-1})t',$$
	where $n=\psi(Fr)$. In other words, $T$ acts on $T$ via multiplication by $tnt^{-1}n^{-1}$. And $T$ acts trivially on $\mu$ (See Proposition \ref{T times mu/T}).
	
	So we are reduced to compute $[T/T]$ with respect to a nice action of the split torus $T$ on $T$, which should be and turns out to be very explicit.
	
	For clarification, let me denote the source torus $T$ by $T^{(1)}$ and the target torus $T$ by $T^{(2)}$. Consider the morphism
	$$f: T^{(1)} = T \to T = T^{(2)}, s \mapsto sns^{-1}n^{-1}.$$
	This is surjective on $\overline{\mathbb{F}_{\ell}}$-points by our assumption \ref{assumption 2} that $Z(\hat{G})$ is finite and $\varphi$ is elliptic (\red{See Lemma below}). Hence $f$ is an epimorphism in the category of diagonalizable $\overline{\mathbb{Z}_{\ell}}$-group schemes (\red{See the same Lemma below})(\red{maybe add an appendix on diagonalizable group schemes?}). Therefore, $f$ induces an isomorphism 
	$$T^{(1)}/ker(f) \simeq T^{(2)},$$
	as diagonalizable $\overline{\mathbb{Z}_{\ell}}$-group schemes. Moreover, if we let $t \in T$ act on $T^{(1)}$ by left multiplication by $t$, and on $T^{(2)}$ via multiplication by $(tnt^{-1}n^{-1})$, this isomorphism induced by $f$ is $T$-equivariant.
	
	Note $T^{(1)}=T$ is commutative, so the $T$-action (via multiplication by $tnt^{-1}n^{-1}$) and the $ker(f)$-action (via left multiplication) on $T$ commutes with each other. Hence by the $T$-equivariant isomorphism $T^{(1)}/ker(f) \simeq T^{(2)}$ above, we have
	$$[T/T] = [T^{(2)}/T] \simeq \left[\left(T^{(1)}/ker(f)\right)/T\right] \simeq \left[\left(T^{(1)}/T\right)/ker(f)\right] \simeq [*/ker(f)] = [*/C_T(n)].$$ 
	
%	Finally, notice that 
%	$$ker(f)=C_T(n)=C_{\hat{G}}(\psi),$$
%	since $T=C_{\hat{G}}(\psi|_{I_F^{\ell}})$ (\red{is this same as $C_{\hat{G}}(\psi|_{I_F})$ ?}) and $n=\psi(Fr)$ (\red{See Lemma ? below}).
	
%	Now let's proof the last assertion: To show $ker(f)=\underline{S_{\varphi}}$ under assumption \ref{assumption 3} (Need adjust) -- $\ell$ does not divide the order of $w$ in the Weyl group $N_{\hat{G}}(T)/T$ (\red{Use T or S?}). Then $ker(f) \simeq \underline{S_{\varphi}}$ is the constant group scheme of the finite abelian group $S_{\varphi}=C_{\hat{G}(\overline{\mathbb{F}_{\ell}})}(\varphi(W_F))$, \red{See Lemma below}. We win!

    For the last assertion, see Lemma \ref{Lem ker(f)} below.
	
\end{proof}

\red{Does $C_T(n) \simeq C_{\hat{G}}(\psi)$ holds?}

\subsubsection{Some lemmas}

\begin{lemma}\label{Lem epic}
	The morphism 
	$$f: T^{(1)} = T \to T = T^{(2)}, s \mapsto sns^{-1}n^{-1}$$
	is epimorphic in the category of diagonalizable $\overline{\mathbb{Z}_{\ell}}$-group schemes. And it induces and isomorphism $T^{(1)}/ker(f) \simeq T^{(2)}$ as diagonalizable $\overline{\mathbb{Z}_{\ell}}$-group schemes.
\end{lemma}

\begin{proof}
	Recall that $T$ is a split torus over $\overline{\mathbb{Z}_{\ell}}$, hence a diagonalizable $\overline{\mathbb{Z}_{\ell}}$-group scheme. Notice that $f$ is a morphism of $\overline{\mathbb{Z}_{\ell}}$-group schemes, hence a morphism of diagonalizable $\overline{\mathbb{Z}_{\ell}}$-group schemes. Recall that the category of diagonalizable $\overline{\mathbb{Z}_{\ell}}$-group schemes is equivalent to the category of abelian groups (See \cite[p70, Section 5]{brochard2014autour} or \cite{conrad2014reductive}) via
	$$D \mapsto \Hom_{\overline{\mathbb{Z}_{\ell}}-GrpSch}(D, \mathbb{G}_m),$$
	and the inverse is given by 
	$$\overline{\mathbb{Z}_{\ell}}[M] \mapsfrom M,$$
	where $\overline{\mathbb{Z}_{\ell}}[M]$ is the group algebra of $M$ with $\overline{\mathbb{Z}_{\ell}}$-coefficients.
	
	Therefore, we could argue in the category of abelian groups via the above category equivalence: $f$ is epimorphic if and only if the map $f^*$ in the category of abelian groups is monomorphic. Note ellipticity and $Z(\hat{G})$ finite imply that $S_{\varphi}$ is finite, hence 
	$$ker(f)(\overline{\mathbb{F}_{\ell}})=C_T(n)=S_{\varphi}$$
	is finite (\red{See Lemma ? for the last equality} \blue{Should be true over $\overline{\mathbb{F}_{\ell}}$. But maybe not true over $\overline{\mathbb{Z}_{\ell}}$?}), hence $coker(f^*)$ is finite. Therefore, 
	$$f^*:\Hom(T^{(2)}, \mathbb{G}_m) \to \Hom(T^{(1)}, \mathbb{G}_m)$$
	is injective (i.e., monomorphism). Indeed, otherwise $ker(f^*)$ would be a nonzero sub-$\mathbb{Z}$-module of the finite free $\mathbb{Z}$-module $\Hom(T^{(2)}, \mathbb{G}_m)$, hence a free $\mathbb{Z}$-module of positive rank, which contradicts with $coker(f^*)$ being finite.
	
	The statement on the quotient follows from the corresponding result in the category of abelian groups: $f^*$ induces an isomorphism
	$$\Hom(T^{(1)}, \mathbb{G}_m)/\Hom(T^{(2)}, \mathbb{G}_m) \simeq coker(f^*)$$
	(See \cite[p71, Subsection 5.3]{brochard2014autour}.)
\end{proof}

\begin{lemma}\label{Lem ker(f)}
	(Assume \ref{assumption 3} (\red{Need adjust}): $\ell$ doesn't divide the order of $w$.) $ker(f) \simeq \underline{S_{\varphi}}$ is the constant group scheme of the finite abelian group $S_{\varphi}=C_{\hat{G}(\overline{\mathbb{F}_{\ell}})}(\varphi(W_F))$.
\end{lemma}

\begin{proof}
	We recall the following fact: Let $H$ be a smooth affine group scheme over some ring $R$, let $\Gamma$ be a finite group whose order is invertible in $R$. Then the fixed point functor $H^{\Gamma}$ is representable and is smooth over $R$.
	
	For a proof of the above fact, see \cite[Proposition 3.4]{edixhoven1992neron} or \cite[Lemma A.1, A.13]{dhkm2020moduli}.
	
	In our case, let $H=T$, $\Gamma=<w>$ the subgroup of the Weyl group $W_{\hat{G}}(T)$ generated by $w$. Hence $$ker(f)=C_T(n)=H^{\Gamma}$$
	(\red{See Lemma ? for the last equality}) is smooth over $\overline{\mathbb{Z}_{\ell}}$. Therefore, $ker(f)$ is finite etale over $\overline{\mathbb{Z}_{\ell}}$ (\red{See Lemma ?}). Hence $ker(f)$ is a constant group scheme over $\overline{\mathbb{Z}_{\ell}}$, since $\overline{\mathbb{Z}_{\ell}}$ has no non-trivial finite etale cover.
	
	Since $ker(f)$ is constant, we can determine it by computing its $\overline{\mathbb{F}_{\ell}}$-points:
	$$ker(f)(\overline{\mathbb{F}_{\ell}})=C_{T(\overline{\mathbb{F}_{\ell}})}(n)=C_{\hat{G}(\overline{\mathbb{F}_{\ell}})}(\varphi(W_F)),$$
	where the middle equality follows by noticing $T(\overline{\mathbb{F}_{\ell}})=C_{\hat{G}(\overline{\mathbb{F}_{\ell}})}(\varphi(I_F))$ and $n=\varphi(Fr)$.
	
	Finally, note by our TRSELP assumption, $C_{\hat{G}(\overline{\mathbb{F}_{\ell}})}(\varphi(I_F))$ is (the $\overline{\mathbb{F}_{\ell}}$-points of) a torus. Hence $S_{\varphi}=C_{\hat{G}(\overline{\mathbb{F}_{\ell}})}(\varphi(W_F))$ is abelian, hence finite abelian as we've noticed in the proof of the previous lemma that $S_\varphi$ is finite (by ellipticity and $Z(\hat{G})$ finite).
\end{proof}

\begin{lemma}\label{Lem Wely}
	Let $\hat{G}$ be a connected reductive group over $\overline{\mathbb{Z}_{\ell}}$, and $T$ a maximal torus of $\hat{G}$. Then the Weyl group $N_{\hat{G}}(T)/T$ is split over $\overline{\mathbb{Z}_{\ell}}$.
\end{lemma}

\begin{proof}
	By \cite[Proposition 3.2.8]{conrad2014reductive}, the Weyl group $N_{\hat{G}}(T)/C_{\hat{G}}(T)$ is finite etale over $\overline{\mathbb{Z}_{\ell}}$ and hence split over $\overline{\mathbb{Z}_{\ell}}$. In our case, $C_{\hat{G}}(T)=T$ since $\hat{G}$ is connected (For example, use the proof of \cite[Theorem 3.1.12]{conrad2014reductive}).
\end{proof}
	
\bibliographystyle{plain}
\bibliography{reference}
\end{document}
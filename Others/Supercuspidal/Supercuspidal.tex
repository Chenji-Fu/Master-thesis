\documentclass[reqno,11pt]{book}

%%%%%%%%%%%%%%%Setting for fast compiling
\special{dvipdfmx:config z 0}
%\includeonly{chapters/contents}
%\includeonly{chapters/introduction, chapters/chapter1}
%\includeonly{chapters/affinepaving}
%\includeonly{chapters/chapter6}

%for beamers, you have to change a lot. Especially, remove the package enumitem!!!



%%%%%%%%%%%%%%%%%%%% setting for fast compiling

%\special{dvipdfmx:config z 0}		% no compression

%\includeonly{chapters/chapter9}		% In practice, use an empty document called "chapter9"	% usually for printing books






%%%%%%%%%%%%%%%%%%%% here we include packages

%%%basic packages for math articles
\usepackage{amssymb}
\usepackage{amsthm}
\usepackage{amsmath}
\usepackage{amsfonts}
\usepackage{mathtools}
\usepackage[shortlabels]{enumitem}	% It supersedes both enumerate and mdwlist. The package option shortlabels is included to configure the labels like in enumerate.

%%%packages for special symbols
\usepackage{pifont}					% Access to PostScript standard Symbol and Dingbats fonts
\usepackage{wasysym}				% additional characters
\usepackage{bm}						% bold fonts: \bm{...}
\usepackage{extarrows}				% may be replaced by tikz-cd
%\usepackage{unicode-math}			% unicode maths for math fonts, now I don't know how to include it
%\usepackage{ctex}					% Chinese characters, huge difference.


%%%basic packages for fancy electronic documents
\usepackage[colorlinks]{hyperref}
\usepackage[table,hyperref]{xcolor} 			% before tikz-cd. 
%\usepackage[table,hyperref,monochrome]{xcolor}	% disable colored output (black and white)

%%%packages for figures and tables (general setting)
\usepackage{float}				%Improved interface for floating objects
\usepackage{caption,subcaption}
\usepackage{adjustbox}			% for me it is usually used in tables 
\usepackage{stackengine}		%baseline changes

%%%packages for commutative diagrams
\usepackage{tikz-cd}
%\usepackage{quiver}			% see https://q.uiver.app/

%%%packages for pictures

\usepackage{graphicx}			% Enhanced support for graphics
\usepackage{stmaryrd}

%%%packages for tables and general settings
\usepackage{array}
\usepackage{makecell}
\usepackage{multicol}
\usepackage{multirow}
\usepackage{diagbox}
\usepackage{longtable}

%%%packages for ToC, LoF and LoT







%https://tex.stackexchange.com/questions/58852/possible-incompatibility-with-enumitem










%%%%%%%%%%%%%%%%%%%% here we include theoremstyles

\numberwithin{equation}{section}

\theoremstyle{plain}
\newtheorem{theorem}{Theorem}[section]

\newtheorem{setting}[theorem]{Setting}
\newtheorem{definition}[theorem]{Definition}
\newtheorem{lemma}[theorem]{Lemma}
\newtheorem{proposition}[theorem]{Proposition}
\newtheorem{corollary}[theorem]{Corollary}
\newtheorem{conjecture}[theorem]{Conjecture}

\newtheorem{claim}[theorem]{Claim}
\newtheorem{eg}[theorem]{Example}
\newtheorem{ex}[theorem]{Exercise}
\newtheorem{fact}[theorem]{Fact}
\newtheorem{ques}[theorem]{Question}
\newtheorem{warning}[theorem]{Warning}



\newtheorem*{bbox}{Black box}
\newtheorem*{notation}{Conventions and Notations}


\numberwithin{equation}{section}


\theoremstyle{remark}

\newtheorem{remark}[theorem]{Remark}
\newtheorem*{remarks}{Remarks}

%%% for important theorems
%\newtheoremstyle{theoremletter}{4mm}{1mm}{\itshape}{ }{\bfseries}{}{ }{}
%\theoremstyle{theoremletter}
%\newtheorem{theoremA}{Theorem}
%\renewcommand{\thetheoremA}{A}
%\newtheorem{theoremB}{Theorem}
%\renewcommand{\thetheoremB}{B}







%%%%%%%%%%%%%%%%%%%% here we declare some symbols

%%%%%%%DeclareMathOperator
%see here for why newcommand is better for DeclareMathOperator: https://tex.stackexchange.com/questions/67506/newcommand-vs-declaremathoperator

%%%%%basic symbols. Keep them!

%%%symbols for sets and maps
\DeclareMathOperator{\pt}{\operatorname{pt}}	%points. Other possibilities are \{pt\}, \{*\}, pt, * ...
\DeclareMathOperator{\Id}{\operatorname{Id}}	%identity in groups.
\DeclareMathOperator{\Img}{\operatorname{Im}}

\DeclareMathOperator{\Ob}{\operatorname{Ob}}
\DeclareMathOperator{\Mor}{\operatorname{Mor}}	%difference of Mor and Hom: Hom is usually for abelian categories
\DeclareMathOperator{\Hom}{\operatorname{Hom}}	\DeclareMathOperator{\End}{\operatorname{End}}
\DeclareMathOperator{\Aut}{\operatorname{Aut}}

%%%symbols for linear algebras and 
%%linear algebras
\DeclareMathOperator{\tr}{\operatorname{tr}}
\DeclareMathOperator{\diag}{\operatorname{diag}}	%for diagonal matrices

%%abstract algebras
\DeclareMathOperator{\ord}{\operatorname{ord}}
\DeclareMathOperator{\gr}{\operatorname{gr}}
\DeclareMathOperator{\Frac}{\operatorname{Frac}}

%%%symbols for basic geometries
\DeclareMathOperator{\vol}{\operatorname{vol}}	%volume
\DeclareMathOperator{\dist}{\operatorname{dist}}
\DeclareMathOperator{\supp}{\operatorname{supp}}

%%%symbols for category
%%names of categories
\DeclareMathOperator{\Mod}{\operatorname{Mod}}
\DeclareMathOperator{\Vect}{\operatorname{Vect}}
\DeclareMathOperator{\rep}{\operatorname{rep}} %usually rep means the category of finite dimensional representations, while Rep means the category of representations.
\DeclareMathOperator{\Rep}{\operatorname{Rep}}
\newcommand{\coker}{\operatorname{coker}}


%%%symbols for homological algebras
\DeclareMathOperator{\Tor}{\operatorname{Tor}}
\DeclareMathOperator{\Ext}{\operatorname{Ext}}
\DeclareMathOperator{\gldim}{\operatorname{gl.dim}}
\DeclareMathOperator{\projdim}{\operatorname{proj.dim}}
\DeclareMathOperator{\injdim}{\operatorname{inj.dim}}
\DeclareMathOperator{\rad}{\operatorname{rad}}


%%%symbols for algebraic groups
\DeclareMathOperator{\GL}{\operatorname{GL}}
\DeclareMathOperator{\SL}{\operatorname{SL}}

%%%symbols for typical varieties
\DeclareMathOperator{\Gr}{\operatorname{Gr}}
\DeclareMathOperator{\Flag}{\operatorname{Flag}}

%%%symbols for basic algebraic geometry
\DeclareMathOperator{\Spec}{\operatorname{Spec}}
\DeclareMathOperator{\Coh}{\operatorname{Coh}}
\newcommand{\Dcoh}{\mathcal{D}_{\operatorname{Coh}}}%%%This one shows the difference between \DeclareMathOperator and \newcommand
\DeclareMathOperator{\Pic}{\operatorname{Pic}}
\DeclareMathOperator{\Jac}{\operatorname{Jac}}


%%%%%advanced symbols. Choose the part you need!

%%%symbols for algebraic representation theory
\DeclareMathOperator{\Irr}{\operatorname{Irr}}
\DeclareMathOperator{\ind}{\operatorname{ind}}	%\ind(Q) means the set of  equivalence classes of finite dimensional indecomposable representations
\DeclareMathOperator{\Res}{\operatorname{Res}}
\DeclareMathOperator{\Ind}{\operatorname{Ind}}
\DeclareMathOperator{\cInd}{\operatorname{c-Ind}}


%%%symbols for algebraic topology
\DeclareMathOperator{\EGG}{\operatorname{E}\!}
\DeclareMathOperator{\BGG}{\operatorname{B}\!}

\DeclareMathOperator{\chern}{\operatorname{ch}^{*}}
\DeclareMathOperator{\Td}{\operatorname{Td}}
\DeclareMathOperator{\AS}{\operatorname{AS}}	%Atiyah--Segal completion theorem 

%%%symbols for Auslander--Reiten theory 
\DeclareMathOperator{\Modup}{\overline{\operatorname{mod}}}
\DeclareMathOperator{\Moddown}{\underline{\operatorname{mod}}}
\DeclareMathOperator{\Homup}{\overline{\operatorname{Hom}}}
\DeclareMathOperator{\Homdown}{\underline{\operatorname{Hom}}}


%%%symbols for operad
\DeclareMathOperator{\Com}{\operatorname{\mathcal{C}om}}
\DeclareMathOperator{\Ass}{\operatorname{\mathcal{A}ss}}
\DeclareMathOperator{\Lie}{\operatorname{\mathcal{L}ie}}
\DeclareMathOperator{\calEnd}{\operatorname{\mathcal{E}nd}} %cal=\mathcal


%%%%%personal symbols. Use at your own risk!

%%%symbols only for master thesis
\DeclareMathOperator{\ptt}{\operatorname{par}}	%the partition map
\DeclareMathOperator{\str}{\operatorname{str}}	%strict case
\DeclareMathOperator{\RRep}{\widetilde{\operatorname{Rep}}}
\DeclareMathOperator{\Rpt}{\operatorname{R}}
\DeclareMathOperator{\Rptc}{\operatorname{\mathcal{R}}}
\DeclareMathOperator{\Spt}{\operatorname{S}}
\DeclareMathOperator{\Sptc}{\operatorname{\mathcal{S}}}
\DeclareMathOperator{\Kcurl}{\operatorname{\mathcal{K}}}
\DeclareMathOperator{\Hcurl}{\operatorname{\mathcal{H}}}
\DeclareMathOperator{\eu}{\operatorname{eu}}
\DeclareMathOperator{\Eu}{\operatorname{Eu}}
\DeclareMathOperator{\dimv}{\operatorname{\underline{\mathbf{dim}}}}
\DeclareMathOperator{\St}{\mathcal{Z}}
\DeclareMathOperator{\Perf}{\operatorname{Perf}}
\DeclareMathOperator{\QCoh}{\operatorname{QCoh}}

%%%%%symbols which haven't been classified. Add your own math operators here!


\DeclareMathOperator{\Modr}{\operatorname{-Mod}}





%%%%%%%newcommand

%%%basic symbols
\newcommand{\norm}[1]{\Vert{#1}\Vert}

%%%symbols only for master thesis
\newcommand{\dimvec}[1]{\mathbf{#1}}
\newcommand{\abdimvec}[1]{|\dimvec{#1}|}
\newcommand{\ftdimvec}[1]{\underline{\dimvec{#1}}}

\newcommand{\absgp}[1]{\mathbb{#1}}
\newcommand{\WWd}{\absgp{W}_{\abdimvec{d}}}
\newcommand{\Wd}{W_{\dimvec{d}}}
\newcommand{\MinWd}{\operatorname{Min}(\absgp{W}_{\abdimvec{d}},W_{\dimvec{d}})}
\newcommand{\Compd}{\operatorname{Comp}_{\dimvec{d}}}
\newcommand{\Shuffled}{\operatorname{Shuffle}_{\dimvec{d}}}

\newcommand{\Omcell}{\Omega}
\newcommand{\OOmcell}{\boldsymbol{\Omega}}
\newcommand{\Vcell}{\mathcal{V}}
\newcommand{\VVcell}{\boldsymbol{\mathcal{V}}}
\newcommand{\Ocell}{\mathcal{O}}
\newcommand{\OOcell}{\boldsymbol{\mathcal{O}}}
\newcommand{\preimage}[1]{\widetilde{#1}}
\newcommand{\orde}{\operatorname{ord}_e}
\newcommand{\fakestar}{*}
\newcommand{\Fr}{\operatorname{Fr}}
\newcommand{\Cont}{\operatorname{Cont}}
\renewcommand{\Modr}{\operatorname{Mod-}}
\newcommand{\Modl}{\operatorname{-Mod}}

%as the subscription of Hom
\newcommand{\Alggp}{\text{-Alg gp}}







%%%%%%%%%%%%%%%%%%%% here we make some blocks for special features. 

%%%% todo notes %%%%
\usepackage[colorinlistoftodos,textsize=footnotesize]{todonotes}
\setlength{\marginparwidth}{2.5cm}
\newcommand{\leftnote}[1]{\reversemarginpar\marginnote{\footnotesize #1}}
\newcommand{\rightnote}[1]{\normalmarginpar\marginnote{\footnotesize #1}\reversemarginpar}









%%%%%%%%%%%%%%%%%%%% here we make some global settings. Understand everything here before you make a document!

\usepackage[a4paper,left=3cm,right=3cm,bottom=4cm]{geometry}
\usepackage{indentfirst}	% Indent first paragraph after section header

\setcounter{tocdepth}{2}


%https://latexref.xyz/_005cparindent-_0026-_005cparskip.html
\setlength{\parindent}{15pt}	
\setlength{\parskip}{0pt plus1pt}

%\setlength\intextsep{0cm}
%\setlength\textfloatsep{0cm}
\def\arraystretch{1}
%\setcounter{secnumdepth}{3}

\allowdisplaybreaks


\begin{document}
	
	
	
	%Name of the author of the thesis 
\authornew{Chenji Fu}
%Date of birth of the Author
\geburtsdatum{15th July 1998}
%Place of Birth
\geburtsort{Jiaxing, China}
%Date of submission of the thesis
\date{\textcolor{red}{to enter}}

%Name of the Advisor
% z.B.: Prof. Dr. Peter Koepke
\betreuer{Advisor: Prof. Dr. Peter Scholze}
%name of the second advisor of the thesis
\zweitgutachter{\textcolor{red}{to delete}}

%Name of the Insitute of the advisor
%z.B.: Mathematisches Institut
\institut{Mathematical Institute}
%\institut{Institut f\"ur Angewandte Mathematik}
%\institut{Institut f\"ur Numerische Simulation}
%\institut{Forschungsinstitut f\"ur Diskrete Mathematik}
%Title of the thesis 
\title{\Large On the categorical local Langlands conjectures for depth-zero regular supercuspidal representations}
%Do not change!
\ausarbeitungstyp{Master's Thesis  Mathematics}


\maketitle
\tableofcontents

%%%%%%%%%%%%%%%%%%%%%%%%%%%%%%%%%%%%%%%%%%%%%%%%%%%%%%%%%%%%%%%%%%%%%%%%%%%%%%%%%%%%%%%%%%%%%

	\chapter{Introduction}

Let $F$ be a non-archimedean local field. Assume that the residue field of $F$ is $\mathbb{F}_q$, with characteristic $p$. Let $G$ be a connected reductive group over $F$. For simplicity, we assume that $G$ is split, semisimple, and simply connected. Let $\Lambda=\overline{\mathbb{Z}}_{\ell}$, the integral closure of $\mathbb{Z}_{\ell}$ in $\overline{\mathbb{Q}}_{\ell}$, with $\ell \neq p$. Let $W_F$ be the Weil group of $F$ and $\hat{G}$ the Langlands dual group of $G$. The categorical local Langlands conjecture predicts that there is a fully faithful embedding
$$\Rep_{\Lambda}(G(F)) \longrightarrow \QCoh(Z^1(W_F, \hat{G})_{\Lambda}/\hat{G})$$
from the category of smooth representations of the $p$-adic group $G(F)$ to the category of quasi-coherent sheaves on the stack of Langlands parameters. In this paper, we compute the two sides explicitly for depth-zero regular supercuspidal part of the group $G$ and verify the categorical local Langlands conjecture for depth-zero supercuspidal part of $GL_n$.\footnote{We will see that although $GL_n$ is not simply connected, the theory still works without much change. Also, we do not need to assume $G$ is simply connected for the results on the $L$-parameter side.}

Fixing an irreducible depth-zero regular supercuspidal representation $\pi \in \Rep_{\overline{\mathbb{F}}_{\ell}}(G(F))$,\footnote{Note that we really want to start with a representation with $\overline{\mathbb{F}}_{\ell}$-coefficients instead of $\overline{\mathbb{Q}}_{\ell}$-coefficients, because we are interested in describing the ($\overline{\mathbb{Z}}_{\ell}$-)block of 
$\Rep_{\Lambda}(G(F))$.} the (classical) local Langlands conjecture predicts that it should correspond to a tame, regular semisimple, elliptic $L$-parameter (TRSELP for short) $\varphi \in Z^1(W_F, \hat{G}(\overline{\mathbb{F}}_{\ell}))$ (see \cite{debacker2009depth}). As mentioned above, this paper focuses on the depth-zero regular supercuspidal part of the categorical local Langlands conjecture, which predicts a fully faithful embedding
$$\Rep_{\Lambda}(G(F))_{[\pi]} \longrightarrow \QCoh([X_{\varphi}/\hat{G}])$$
from the block of $\Rep_{\Lambda}(G(F))$ containing $\pi$ to the category of quasi-coherent sheaves on the connected component $[X_{\varphi}/\hat{G}]$ of the stack of $L$-parameters $Z^1(W_F, \hat{G})_{\Lambda}/\hat{G}$.

\section{$L$-parameter side}
Let $G$ be a connected split reductive group over $F$. Let $\varphi \in Z^1(W_F, \hat{G}(\overline{\mathbb{F}}_{\ell}))$ be a TRSELP (see Definition \ref{Def TRSELP}). In this section, we explain Chapter \ref{Chapter MoLP} on how to compute $\QCoh([X_{\varphi}/\hat{G}])$. 

This is done by describing $[X_{\varphi}/\hat{G}]$ explicitly as a quotient stack over $\Lambda=\overline{\mathbb{Z}}_{\ell}$. 

\subsection{Heuristics on the component $[X_{\varphi}/\hat{G}]$}

In this subsection, we describe some heuristics on the component $[X_{\varphi}/\hat{G}]$ which help us to guess what this component should look like.

First, let us recall what is known over $\overline{\mathbb{Q}}_{\ell}$ instead of $\Lambda=\overline{\mathbb{Z}}_{\ell}$. Indeed, assuming that the center $Z(\hat{G})$ of $\hat{G}$ is finite, the connected component of the stack of $L$-parameters $Z^1(W_F, \hat{G})_{\overline{\mathbb{Q}}_{\ell}}/\hat{G}$ over $\overline{\mathbb{Q}}_{\ell}$ containing an elliptic $L$-parameter $\varphi'$ is known to be one point. More precisely, it is isomorphic to the quotient stack $[*/S_{\varphi'}]$, where $S_{\varphi'}=C_{\hat{G}}(\varphi')$ is the centralizer of $\varphi'$ (see \cite[Section X.2]{fargues2021geometrization}).

Second, let us explain the difference between the geometry of the connected components of the stack of $L$-parameters over $\overline{\mathbb{Q}}_{\ell}$ and $\overline{\mathbb{Z}}_{\ell}$. This can be seen from the example $G=GL_1$. Indeed, 
$$Z^1(W_F, \widehat{GL_1}) \cong \mu_{q-1} \times \mathbb{G}_m,$$
both over $\overline{\mathbb{Q}}_{\ell}$ and $\overline{\mathbb{Z}}_{\ell}$ (see Example \ref{Example GL_1}). However, $\mu_{q-1}$ is just $q-1$ discrete points over $\overline{\mathbb{Q}}_{\ell}$, while
the connected components of $\mu_{q-1}$ are isomorphic to $\mu_{\ell^k}$ (over $\overline{\mathbb{F}}_{\ell}$, hence also) over $\overline{\mathbb{Z}}_{\ell}$, where $k$ is the maximal integer such that $\ell^k$ divides $q-1$. So when describing the connected components of the stack of $L$-parameters over $\overline{\mathbb{Z}}_{\ell}$, there will be possibly some non-reduced part $\mu$ appearing.

These two features come together in the description of $[X_{\varphi}/\hat{G}]$, the connected component of $Z^1(W_F, \hat{G})/\hat{G}$ containing $\varphi$. Under mild assumptions, we prove that
$$[X_{\varphi}/\hat{G}] \cong [*/S_{\varphi}]\times \mu,$$
where $S_{\varphi}=C_{\hat{G}}(\varphi)$ and $\mu$ is some product of $\mu_{\ell^{k_i}}$'s (see Theorem \ref{Thm X/G}).

\subsection{Ingredients of the computation}

The computation follows the theory of moduli space of Langlands parameters developed in \cite[Section 2, 4]{dhkm2020moduli} (see also \cite[Section 3, 4]{dat2022ihes} for a more gentle introduction). It is very helpful to do the example of $GL_2$ first (see Chapter \ref{Chapter GL_n}).

To compute the component $[X_{\varphi}/\hat{G}]$ over $\overline{\mathbb{Z}}_{\ell}$, let us fix a lift $\psi \in Z^1(W_F, \hat{G}(\overline{\mathbb{Z}}_{\ell}))$ of $\varphi \in Z^1(W_F, \hat{G}(\overline{\mathbb{F}}_{\ell}))$.

Recall by \cite[Subsection 4.6]{dat2022ihes},
$$X_{\varphi}=X_{\psi} \cong \left(\hat{G} \times Z^1(W_F, N_{\hat{G}}(\psi_{\ell}))_{\psi_{\ell}, \overline{\psi}}\right)/C_{\hat{G}}(\psi_{\ell})_{\overline{\psi}},$$
where $Z^1(W_F, N_{\hat{G}}(\psi_{\ell}))_{\psi_{\ell}, \overline{\psi}}$  denotes the space of cocycles whose restriction to $I_F^{\ell}$ equals $\psi_{\ell}$ and whose image in $Z^1(W_F, \pi_0(N_{\hat{G}}(\psi_{\ell})))$ is $\overline{\psi}$. 

Here, $Z^1(W_F, N_{\hat{G}}(\psi_{\ell}))_{\psi_{\ell}, \overline{\psi}}$ is essentially the space of cocycles of the torus $$T:=N_{\hat{G}}(\psi_{\ell})^0=C_{\hat{G}}(\psi_{\ell})$$
by our TRSELP assumption (see Definition \ref{Def TRSELP}) and that $C_{\hat{G}}(\psi_{\ell})$ is generalized reductive, hence split over $\overline{\mathbb{Z}}_{\ell}$ (see Lemma \ref{Lem gen red}). Since it is not hard to compute the space of tame cocycles of a commutative group scheme using the explicit presentation of the tame Weil group (see \eqref{Equation presentation of the tame Weil group} and \eqref{Equation space of tame cocycle}), we obtain that 
$$Z^1(W_F, N_{\hat{G}}(\psi_{\ell}))_{\psi_{\ell}, \overline{\psi}} \cong T \times \mu,$$
where $\mu$ is a product of $\mu_{\ell^{k_i}}$'s (see Theorem \ref{Thm X} for details).
And it is not hard to see that
$$C_{\hat{G}}(\psi_{\ell})_{\overline{\psi}}=C_{\hat{G}}(\psi_{\ell})=T.$$

Therefore, we get 
$$X_{\varphi} \cong \left(\hat{G} \times T \times \mu\right)/T.$$
One needs to be a bit careful about the $T$ action on $T$, because here a twist by $\psi(\Fr)$ is involved. One can compute that 
$$[X_{\varphi}/\hat{G}] \cong [\left(T \times \mu\right)/T] \cong [T/T] \times \mu,$$
where $T$ acts on $T$ via twisted conjugacy.\footnote{Note that so far, we do not assume $Z(\hat{G})$ to be finite, hence the result also applies for $GL_n$.} After that, assuming that $Z(\hat{G})$ is finite, we can work in the category of diagonalizable group schemes (whose structure is clear, see \cite[p70, Section 5]{brochard2014autour}) to identify $[T/T]$ with $[*/S_{\varphi}]$ under mild conditions.

\section{Representation side}
Let $G$ be a connected reductive group over $F$. We assume that $G$ is split, semisimple, and simply connected. Let $\pi \in \Rep_{\overline{\mathbb{F}}_{\ell}}(G(F))$ be an irreducible depth-zero regular supercuspidal representation. In this section, we explain Chapter \ref{Chapter Rep} on how to compute the block $\Rep_{\Lambda}(G(F))_{[\pi]}$ of $\Rep_{\Lambda}(G(F))$ containing $\pi$.

\subsection{Equivalence to the block of a finite group of Lie type}

Recall that a depth-zero regular supercuspidal representation of $G(F)$ is of the form
$$\pi=\cInd_{G_x}^{G(F)}\rho$$
for some representation $\rho$ of the parahoric subgroup $G_x$ corresponding to a vertex $x$ in the Bruhat-Tits building of $G$ over $F$. Moreover, $\rho$ is the inflation of some regular supercuspidal representation $\overline{\rho}$ of the finite group of Lie type $\overline{G_x}:=G_x/G_x^+$.

Let $\mathcal{A}_{x,1}$ denote the block $\Rep_{\Lambda}(\overline{G_x})_{[\overline{\rho}]}$ of $\Rep_{\Lambda}(\overline{G_x})$ containing $\overline{\rho}$. Similarly, let 
$$\mathcal{B}_{x,1}:=\Rep_{\Lambda}(G_x)_{[\rho]},\qquad \mathcal{C}_{x,1}:=\Rep_{\Lambda}(G(F))_{[\pi]}.$$

Assume that the residue field of $F$ is $\mathbb{F}_q$. For simplicity, we assume that $q$ is greater than the Coxeter number of $\overline{G_x}$ (see Theorem \ref{Thm Broué} for reason). Then $\mathcal{A}_{x,1}$ is equivalent to a block of a finite torus via Broué's equivalence \ref{Thm Broué}. And it is not hard to show that the inflation induces an equivalence of categories $\mathcal{A}_{x,1} \cong \mathcal{B}_{x,1}$.

The main theorem we will prove for the representation side is Theorem \ref{Thm Main}: Assume that $q$ is greater than the Coxeter number of $\overline{G_x}$. Then the compact induction induces an equivalence of categories
$$\cInd_{G_x}^{G(F)}: \mathcal{B}_{x,1} \to \mathcal{C}_{x,1}.$$
Once this is proven, $\mathcal{C}_{x,1}$ is equivalent to $\mathcal{A}_{x,1}$, hence admits an explicit description. The proof of the Theorem \ref{Thm Main} occupies the most of Chapter \ref{Chapter Rep}. 

\subsection{Proof of the main theorem for the representation side}

In the rest of the section, let us briefly explain the idea of the proof of Theorem \ref{Thm Main}.

The fully faithfulness of 
$$\cInd_{G_x}^{G(F)}: \mathcal{B}_{x,1} \to \mathcal{C}_{x,1}$$
is a usual computation by Frobenius reciprocity and Mackey's formula. Since a similar computation will be used later, we record it in Theorem \ref{Thm Hom}. The key point is that 
$$\Hom_G\left(\cInd_{G_x}^{G(F)}\rho_1, \cInd_{G_y}^{G(F)}\rho_2\right)$$
can be computed explicitly assuming that one of $\rho_1, \rho_2$ has supercuspidal reduction (i.e. $\overline{\rho_1}$ or $\overline{\rho_2}$ is supercuspidal).\footnote{There is a little subtlety that we want not only $\rho$ to have supercuspidal reduction but also any representation $\rho' \in \mathcal{B}_{x,1}$ to have supercuspidal reduction. This subtlety is dealt with in Theorem \ref{Thm SC Red}. And this is why we need the \textbf{regular} supercuspidal assumption. }

%\begin{remark}
%	Although supercuspidal representations of $G(F)$ come from \textbf{supercuspidal} representation $\overline{\rho}$ of the finite group of Lie type $\overline{G_x}$, only the \textbf{cuspidality} of $\overline{\rho}$ is relevant in our proof of Theorem \ref{Thm Hom}. That's why we analyze cuspidality instead of supercuspidality in detail for finite group of Lie types in Chapter \ref{Chapter Rep}.
%\end{remark}

The difficulty lies in proving that
$$\cInd_{G_x}^{G(F)}: \mathcal{B}_{x,1} \to \mathcal{C}_{x,1}$$
is essentially surjective. For this, we prove that the compact induction $$\Pi_{x,1}:=\cInd_{G_x}^{G(F)}\sigma_{x,1}$$ 
is a projective generator of $\mathcal{C}_{x,1}$. 

The first key is that $\Pi_{x,1}$ is a summand of a projective generator of a larger category. Indeed, $\Pi_{x,1}$ is a summand of
$$\Pi:=\cInd_{G_x^+}^{G(F)}\Lambda,$$
where $G_x^+$ is the pro-unipotent radical of the parahoric subgroup $G_x$; and $\Pi$ is known to be a projective generator of the category $\Rep_{\Lambda}(G(F))_0$ of depth-zero representations, i.e., 
$$\Pi=\Pi_{x,1} \oplus \Pi^{x,1}.$$

The second key is that the complement $\Pi^{x,1}$ does not interfere with $\Pi_{x,1}$. More precisely, we can compute using Theorem \ref{Thm Hom} that 
$$\Hom_{G}(\Pi^{x,1}, \Pi_{x,1})=\Hom_{G}(\Pi_{x,1}, \Pi^{x,1})=0.$$

The above two keys allow us to conclude that $\Pi_{x,1}$ is a projective generator of $\mathcal{C}_{x,1}$. 




\section{The example of $GL_n$}

To illustrate the theory, we do the example of $GL_n$ in Chapter \ref{Chapter GL_n}.\footnote{Although $GL_n$ is not simply connected, the theory still works without much change} It is quite concise once we have the theories developed in Chapter \ref{Chapter MoLP} and \ref{Chapter Rep}, so let us do not say anything more here. However, the readers are encouraged to do the example of $GL_2$ throughout the paper, which will help to understand the theories in Chapter \ref{Chapter MoLP} and \ref{Chapter Rep}. 

\section{The categorical local Langlands conjecture for $GL_n$}

As an application, we will deduce the categorical local Langlands conjecture in Fargues-Scholze's form (see \cite[Conjecture X.3.5]{fargues2021geometrization}) for depth-zero supercuspidal blocks of $GL_n$ in Chapter \ref{Chapter CLLC}.\footnote{Notice that supercuspidal implies regular supercuspidal automatically in the $GL_n$ case.}

The idea is that we can unravel both sides of the categorical conjecture explicitly using our computation in Chapter \ref{Chapter GL_n}. They both turn out to be
$$\bigoplus_{\mathbb{Z}}\Perf(\mathbb{G}_m \times \mu).$$ 
We want to show that the spectral action gives an equivalence. This reduces to the degree-zero part (of the $\mathbb{Z}$-grading) by compatibility of the spectral action with the $\mathbb{Z}$-grading (see Proposition \ref{Prop Spectral action}). And the degree-zero part reduces to the theory of local Langlands in families (see \cite{helm2018converse}). Fortunately, several technical results we need to do the reductions are already available by \cite{zou2022categorical}.

\section{Preliminaries and notations}
Although the main chapters \ref{Chapter MoLP}, \ref{Chapter Rep}, \ref{Chapter CLLC} are almost logically independent of each other, let us introduce some notations that are used throughout the thesis.
\begin{enumerate}
	\item Let $G$ be a connected split reductive group.\footnote{The assumption ``split" is intended for simplicity. It is expected that the results generalize to more general reductive groups.} In Chapter \ref{Chapter Rep}, we moreover assume that $G$ is simply connected.\footnote{Again, the assumption ``simply connected" is intended for simplicity. So that we do not need to consider the extension from $G_x$ to its normalizer when doing compact induction.}
	\item Let $\overline{\mathbb{Z}}_{\ell}$ be the integral closure of $\mathbb{Z}_{\ell}$ in $\overline{\mathbb{Q}}_{\ell}$. $\overline{\mathbb{Z}}_{\ell}$ is also the valuation ring of  $\overline{\mathbb{Q}}_{\ell}$, i.e., it consists of the elements with valuations $\geq 0$. $\overline{\mathbb{Z}}_{\ell}$ is a local ring with the unique maximal ideal consisting of elements with valuations $> 0$. $\overline{\mathbb{Z}}_{\ell}$ is strictly henselian, hence all finite étale covers of $\overline{\mathbb{Z}}_{\ell}$ split. An important fact is that any reductive group scheme over $\overline{\mathbb{Z}}_{\ell}$ is split.\footnote{One drawback of $\overline{\mathbb{Z}}_{\ell}$ is that it is not Neotherian. However, in practice, things defined over $\overline{\mathbb{Z}}_{\ell}$ are already defined over some finite extension $\mathcal{O}$ of $\mathbb{Z}_{\ell}$. So we can first argue for $\mathcal{O}$ and then base change to $\overline{\mathbb{Z}}_{\ell}$. As an aside, the author believes that it does not make a difference if we replace $\overline{\mathbb{Z}}_{\ell}$ by $W(\overline{\mathbb{F}}_{\ell})$, the ring of Witt vectors of $\overline{\mathbb{F}}_{\ell}$, which is the unique complete discrete valuation ring with residue field $\overline{\mathbb{F}}_{\ell}$.}
\end{enumerate}


\section{Logical dependence and suggestions for reading}
Chapter \ref{Chapter MoLP} and \ref{Chapter Rep} are logically independent of each other. Chapter \ref{Chapter CLLC} needs the description of $[X_{\varphi}/\hat{G}]$ for $G=GL_n$ obtained in Chapter \ref{Chapter GL_n}, by applying the theory in Chapter \ref{Chapter MoLP} to the case $G=GL_n$. Also, it might be easier to assume $G=GL_2$ or $SL_2$ throughout the paper, and one would not miss many essential points in doing so.

\section{Acknowledgements}

It is a pleasure to thank my advisor Peter Scholze for giving me this topic to study and for sharing ideas related to it. Next, I would like to thank my families and friends for their love and support. I thank Johannes Anschütz, Jean-François Dat, Olivier Dudas, Jessica Fintzen, Linus Hamann, Eugen Hellmann, David Helm, Alexander Ivanov, Tasho Kaletha, David Schwein, Vincent Sécherre, Maarten Solleveld, and Marie-France Vignéras for some nice conversations related to this work. I thank my bachelor advisors Kei Yuen Chan and Haining Wang for some helpful suggestions. I thank Anne-Marie Aubert for some helpful discussions and for pointing out some errors. Special thanks go to Jiaxi Mo, Mingjia Zhang, Pengcheng Zhang, Xiaoxiang Zhou, and Konrad Zou for their consistent interest in this work as well as emotional support.

	


\chapter{TRSELP components of the stack of $L$-parameters} \label{Chapter MoLP}

    Let $\varphi \in Z^1(W_F, \hat{G}(\overline{\mathbb{F}}_{\ell}))$ be a tame regular semisimple elliptic $L$-parameter. In this chapter, we compute the connected component $[X_{\varphi}/\hat{G}]$ of the stack of $L$-parameters $Z^1(W_F, \hat{G})_{\overline{\mathbb{Z}}_{\ell}}/\hat{G}$ containing $\varphi$. In Section \ref{Section X_phi}, following the theory developed in \cite[Section 3, 4]{dat2022ihes}, we first compute the connected component $X_{\varphi}$ of the space of $1$-cocycles $Z^1(W_F, \hat{G})_{\overline{\mathbb{Z}}_{\ell}}$ (without modulo out the $\hat{G}$-action).
    The result turns out to be very explicit:
    \begin{equation}\label{Equation X}
    	X_{\varphi} \cong (\hat{G} \times T \times \mu)/T
    \end{equation}
    (see Theorem \ref{Thm X} for details). In Section \ref{Section X/hatG}, we use \eqref{Equation X} to obtain a particular simple description of $[X_{\varphi}/\hat{G}]$ under mild conditions (see Theorem \ref{Thm X/G}):
    \begin{equation}
    	[X_{\varphi}/\hat{G}] \cong [*/S_{\varphi}] \times \mu.
    \end{equation}
    
	\section{The connected component $X_{\varphi}$ containing a TRSELP $\varphi$}\label{Section X_phi}
	
	The goal of this section is to compute the connected component $X_{\varphi}$ containing a TRSELP $\varphi$. In \ref{Subsection MoLP}, we recall the theory of moduli space of Langlands parameters. In \ref{Subsection TRSELP}, we define the class of $L$-parameters that we are interested in -- tame regular semisimple elliptic $L$-parameters (TRSELP for short). In \ref{Subsection the component}, we compute $X_{\varphi}$ explicitly as $(\hat{G}\times T\times \mu)/T$ using the theory of moduli space of Langlands parameters. In \ref{Subsection T-action}, we spell out the $T$-action on $(\hat{G}\times T\times \mu)$ to prepare for the next section.
	
	\subsection{Recollections on the moduli space of Langlands parameters}\label{Subsection MoLP}
	
	Since our computation heavily uses the theory of moduli space of Langlands parameters, we recollect some basic facts here. For more sophisticated knowledge that will be used, we refer to \cite[Section 3, 4]{dat2022ihes}, or \cite[Section 2, 4]{dhkm2020moduli}. 
%	we assume the readers to be familiar with the theory of the moduli space of Langlands parameters, see for example \cite[Section 3 and Section 4]{dat2022ihes}, or \cite[Section 2 and Section 4]{dhkm2020moduli}. 
%	(\textcolor{red}{we can also recollect the theory in the appendix.})
	
	Let us first fix some notations.
	\begin{itemize}
		\item Let $p \neq 2$ be a fixed prime number and $\ell \neq 2$ be a prime number different from $p$. 
		\item Let $F$ be a non-archimedean local field with residue field $\mathbb{F}_q$, where $q=p^r$ for some $r \in \mathbb{Z}_{\geq 1}$.
		\item Let $W_F$ be the Weil group of $F$, $I_F \subseteq W_F$ be the inertia subgroup, and $P_F \subseteq W_F$ be the wild inertia subgroup.
		\item Let $W_t:=W_F/P_F$ be the tame Weil group, $I_t:=I_F/P_F$ be the tame inertia subgroup in $W_t$.
		\item Let $G$ be a connected split reductive group over $F$.
	\end{itemize}
	     Fix $\Fr \in W_F$ any lift of the arithmetic Frobenius element. We will abuse the notation and denote by $\Fr$ the image of $\Fr$ in $W_t$. We have $W_t \cong I_t \rtimes \left<\Fr\right>$. Here, $I_t$ is non-canonically isomorphic to $\prod_{p'\neq p}\mathbb{Z}_{p'}$, which is procyclic. We fix such an isomorphism
	     \begin{equation}\label{Eq I_t}
	     	I_t \cong \prod_{p'\neq p}\mathbb{Z}_{p'}.
	     \end{equation}
    This gives rise to a topological generator $s_0$ of $I_t$, which corresponds to $(1, 1, ...)$ under the isomorphism \eqref{Eq I_t}. Let us recall the following relation in $I_F/P_F$:
	\begin{equation}\label{Eq Fr s_0}
		\Fr s_0 \Fr^{-1}=s_0^q.
	\end{equation}
	In fact, this is true for any $s \in I_t$ instead of $s_0$.
	
	Let 
	$$W_t^0:=\left<s_0, \Fr\right>=\mathbb{Z}[1/p]^{s_0} \rtimes \mathbb{Z}^{\Fr}$$ 
	be the subgroup of $W_t$ generated by $s_0$ and $\Fr$. Let $W_F^0$ denote the preimage of $W_t^0$ under the natural projection $W_F \to W_t$. $W_F^0$ is referred to as the discretization of the Weil group. To summarize, $W_t^0$ is generated by two elements $\Fr$ and $s_0$ with a single relation, i.e., 
	\begin{equation}\label{Equation presentation of the tame Weil group}
		W_t^0=\left<\Fr, s_0\;|\;\Fr s_0 \Fr^{-1}=s_0^q\right>.
	\end{equation} 
	
	Let $G$ be a connected split reductive group over $F$. Let $\hat{G}$ be its dual group over $\mathbb{Z}$. Then the space of cocycles from the discretization
	\begin{equation}\label{Equation space of tame cocycle}
		Z^1(W_t^0, \hat{G})=\underline{\Hom}(W_t^0, \hat{G})=\{(x, y) \in \hat{G} \times \hat{G}\;|\;yxy^{-1}=x^q\}
	\end{equation}
	is an explicit closed subscheme of $\hat{G} \times \hat{G}$ (see \cite[Section 3]{dat2022ihes}). An important fact (see \cite[Proposition 3.9]{dat2022ihes}) is that over a $\mathbb{Z}_{\ell}$-algebra $R$ (the cases $R=\overline{\mathbb{F}}_{\ell}, \overline{\mathbb{Z}}_{\ell}, \overline{\mathbb{Q}}_{\ell}$ are most relevant for us), the restriction from $W_t$ to $W_t^0$ induces an isomorphism
	$$Z^1(W_t, \hat{G}) \cong Z^1(W_t^0, \hat{G}).$$ 
	Therefore, we can compute $Z^1(W_t, \hat{G})$ using the explicit formula \eqref{Equation space of tame cocycle} above. This is fundamental for the study of the moduli space of Langlands parameters $Z^1(W_t, \hat{G})$. We refer the readers to \cite[Section 3, 4]{dat2022ihes} for the precise definition and properties of $Z^1(W_t, \hat{G})$.\footnote{Although we start with a split reductive group, the space of cocycles of certain non-split reductive group would occur when describing the TRSELP component $X_{\varphi}$ of $Z^1(W_F, \hat{G})$ (for example, the space $Z^1_{Ad(\psi)}(W_F, N_{\hat{G}}(\psi_{\ell})^0)$ occurring in the proof of Theorem \ref{Thm X}). We refer the reader to \cite{dat2022ihes} and \cite{dhkm2020moduli} for the definition of $Z^1(W_F, H)$ for general group $H$.} 
	
	\begin{eg}\label{Example GL_1}
		For $G=GL_1$,
	  \begin{equation}
	  \begin{aligned}
		&Z^1(W_t, \hat{G}) \cong Z^1(W_t^0, \hat{G})\\
		=\;&\{(x, y) \in GL_1 \times GL_1\;|\;yxy^{-1}=x^q\}\\
		=\;&\{(x, y) \in GL_1 \times GL_1\;|\;x=x^q\} \cong \mu_{q-1} \times \mathbb{G}_m.
	  \end{aligned}
      \end{equation}
      
      More generally, let $\hat{T}$ be a (possibly non-split) torus equipped with a $W_F$-action. We can compute similarly by tracing the image of $s_0$ and $\Fr$ that
      \begin{equation}\label{Equation: Z^1(W, T)}
      Z^1(W_t, \hat{T}) \cong \hat{T} \times \hat{T}^{\Fr=(-)^q},
      \end{equation} 
      where $\hat{T}^{\Fr=(-)^q}$ is the subscheme of $\hat{T}$ on which $\Fr$ acts by raising to $q$-th power.\footnote{See \cite[Example 3.14]{dat2022ihes} for details. See also the proof of Theorem \ref{Thm X} for an example -- $Z^1_{Ad(\psi)}(W_F, N_{\hat{G}}(\psi_{\ell})^0)$.}
	\end{eg}
	
	Let $I_F^{\ell}$ be the prime-to-$\ell$ inertia subgroup of $W_F$, i.e., $I_F^{\ell}:=\ker(t_{\ell})$, where 
	$$t_\ell: I_F \to I_F/P_F \cong \prod_{p' \neq p}\mathbb{Z}_{p'} \to \mathbb{Z}_\ell$$
	is the composition. In other words, it is the maximal subgroup of $I_F$ with pro-order prime to $\ell$. This property makes $I_F^{\ell}$ important when determining the connected components of $Z^1(W_F, \hat{G})$ over $\overline{\mathbb{Z}}_{\ell}$ (see \cite[Theorem 4.2 and Subsection 4.6]{dat2022ihes}). 
	
	\subsection{Tame regular semisimple elliptic $L$-parameters}\label{Subsection TRSELP}
	
	We want to define a class of $L$-parameters, called TRSELP, which roughly corresponds to depth-zero regular supercuspidal representations. Before that, let us define the concept of schematic centralizer, which will be used throughout the chapter.
	
	\begin{definition}[Schematic centralizer]\label{Definition: Schematic centralizer}
	Let $H$ be an affine algebraic group over a ring $R$, and let $\Gamma$ be a finite group. Let $u \in Z^1(\Gamma, H(R'))$ be a $1$-cocycle for some $R$-algebra $R'$. Let 
	$$\alpha_u: H_{R'} \longrightarrow Z^1(\Gamma, H)_{R'}\qquad h \longmapsto hu(-)h^{-1}$$
	 be the orbit morphism. Then the schematic centralizer $C_H(u)$ is defined as the fiber of $\alpha_u$ at $u$.
	$$	
	\begin{tikzcd}
		{C_H(u)} \arrow[r, ""] \arrow[d, ""] & {H_{R'}} \arrow[d, "{\alpha_u}"] \\
		{\Spec(R')} \arrow[r, "u"]                & {Z^1(\Gamma, H)_{R'}}               
	\end{tikzcd}
	$$	
	\end{definition}
	
	One can show that its $R''$-valued points $C_H(u)(R'')=C_{H(R'')}(u)$ is the set-theoretic centralizer for all $R'$-algebra $R''$, see for example \cite[Appendix A]{dhkm2020moduli}.
	
	\begin{remark}
			Note that this is enough for our applications where $\Gamma$ is more generally taken as a profinite group, because $u: \Gamma \to H$ will factor through a finite quotient $\Gamma'$ of $\Gamma$ in practice.
	\end{remark}
	
	Let us now define a tame, regular semisimple, elliptic Langlands parameter (TRSELP) over $\overline{\mathbb{F}}_{\ell}$, roughly in the sense of \cite[Section 3.4, 4.1]{debacker2009depth}, but with $\overline{\mathbb{F}}_{\ell}$-coefficients instead of $\mathbb{C}$-coefficients.
	
	\begin{definition}\label{Def TRSELP}
		A \textbf{tame regular semisimple elliptic $L$-parameter (TRSELP) over $\overline{\mathbb{F}}_{\ell}$} is a homomorphism $\varphi: W_F \to \hat{G}(\overline{\mathbb{F}}_{\ell})$ such that:
		\begin{enumerate}
			\item (smooth) $\varphi(I_F)$ is a finite subgroup of $\hat{G}(\overline{\mathbb{F}}_{\ell})$.
			\item (Frobenius semisimple) $\varphi(\Fr)$ is a semisimple element of $\hat{G}(\overline{\mathbb{F}}_{\ell})$.
			\item (tame) The restriction of $\varphi$ to $P_F$ is trivial.
			\item \label{regular semisimple}(regular semisimple) The centralizer of the inertia $C_{\hat{G}}(\varphi|_{I_F})$ is a torus (in particular, connected).
			\item \label{elliptic} (elliptic) The identity component $C_{\hat{G}}(\varphi)^0$ of the centralizer $C_{\hat{G}}(\varphi)$ is equal to the identity  component $Z(\hat{G})^0$ of the center $Z(\hat{G})$.
			\end{enumerate}
	\end{definition}

    Concretely, a TRSELP consists of the following data:
    
    \begin{enumerate}
    	\item The restriction to the inertia $\varphi|_{I_F}$, which is essentially a direct sum of characters of some $\mathbb{F}_{q^n}^*$ (think about the example of $GL_n$). Indeed, $I_F/P_F \cong \varprojlim\mathbb{F}_{q^n}^*$ and that
    	$$\Hom_{\Cont}(I_F/P_F, \overline{\mathbb{F}}_{\ell}^*) \cong \Hom_{\Cont}(\varprojlim\mathbb{F}_{q^n}^*, \overline{\mathbb{F}}_{\ell}^*) \cong \varinjlim\Hom_{\Cont}(\mathbb{F}_{q^n}^*, \overline{\mathbb{F}}_{\ell}^*).$$
    	In particular, it factors through (the $\overline{\mathbb{F}}_{\ell}$-points of) some maximal torus, say $S$. Then $\varphi$ being regular semisimple means that $C_{\hat{G}(\overline{\mathbb{F}}_{\ell})}(\varphi(I_F))=S$.
    	\item The image of the Frobenius $\varphi(\Fr)$, which turns out to be an element of the normalizer $N_{\hat{G}(\overline{\mathbb{F}}_{\ell})}(S)$ (Since $\Fr.s.\Fr^{-1}=s^q \in I_t$ for any $s \in I_t$ implies that $\varphi(\Fr)$ normalizes $C_{\hat{G}(\overline{\mathbb{F}}_{\ell})}(\varphi(I_F))=S$.). 
        And ``elliptic" means that the center $Z(\hat{G})$ has finite index in the centralizer $C_{\hat{G}}(\varphi)$. As we will see later, ellipticity implies that $\hat{G}(\overline{\mathbb{F}}_{\ell})$ acts transitively on the connected component $X_{\varphi}(\overline{\mathbb{F}}_{\ell})$ of the moduli space of $L$-parameters containing $\varphi$ (see the proof of Lemma \ref{Lem epic}), which is essential for the description (roughly, see Theorem \ref{Thm X/G} for the precise statement)
    	$$[X_{\varphi}/\hat{G}] \cong [*/\underline{S_{\varphi}}],$$
    	where $S_\varphi=C_{\hat{G}(\overline{\mathbb{F}}_{\ell})}(\varphi(W_F))$ is the centralizer of the whole $L$-parameter $\varphi$.
    \end{enumerate}

    \begin{eg}
    	For $G=GL_n$, a TRSELP is the same as an irreducible tame $L$-parameter. See Section \ref{Example Lparam} for the irreducible tame $L$-parameters of $GL_n$ expressed in explicit matrices.
    \end{eg}

  \begin{remark}\
  	\begin{enumerate}
  		\item Let $A \in \{\overline{\mathbb{Z}}_{\ell}, \overline{\mathbb{Q}}_{\ell}, \overline{\mathbb{F}}_{\ell}\}$. It is important for our purpose to distinguish between the set-theoretic centralizer (for example, $C_{\hat{G}(A)}(\varphi(W_F))$) and the schematic centralizer (for example, $C_{\hat{G}}(\varphi)$). However, we might still use $\hat{G}$ to mean $\hat{G}(A)$ sometimes by abuse of notation, which we hope the reader can recognize. One reason for doing so is that $\hat{G}$ is split over $A$; hence, $\hat{G}$ is completely determined by its $A$-points. Many statements can either be phrased in terms of the $A$-scheme or its $A$-points (for example, \ref{regular semisimple} and \ref{elliptic} in Definition \ref{Def TRSELP}).
  		\item As we will see later in Theorem \ref{Thm X}, $S=C_{\hat{G}(\overline{\mathbb{F}}_{\ell})}(\varphi(I_F))$ turns out to be the $\overline{\mathbb{F}}_{\ell}$-points of the split torus $T=C_{\hat{G}}(\psi|_{I_F^{\ell}})$ for some lift $\psi$ of $\varphi$ over $\overline{\mathbb{Z}}_{\ell}$.
  	\end{enumerate}
  \end{remark}

\subsection{Description of the component}\label{Subsection the component}

Now let us fix a TRSELP $\varphi \in Z^1(W_F, \hat{G}(\overline{\mathbb{F}}_{\ell}))$. Pick any lift $\psi \in Z^1(W_F, \hat{G}(\overline{\mathbb{Z}}_{\ell}))$ of $\varphi$, whose existence is ensured by the flatness of $Z^1(W_F, \hat{G})_{\overline{\mathbb{Z}}_{\ell}}$ (see Lemma \ref{Lem generalizing}). Let $\psi_{\ell}:=\psi|_{I_F^{\ell}}$ denote the restriction of $\psi$ to the prime-to-$\ell$ inertia $I_F^{\ell}$. Note that $\psi \in Z^1(W_F, \hat{G})$ factors through $N_{\hat{G}}(\psi_{\ell})$ (since $I_F^{\ell}$ is normal in $W_F$). Let $\overline{\psi}$ denote the image of $\psi$ in $Z^1(W_F, \pi_0(N_{\hat{G}}(\psi_{\ell})))$. Let $X_{\varphi}$ be the connected component of $Z^1(W_F, \hat{G})_{\overline{\mathbb{Z}}_{\ell}}$ containing $\varphi$. Note that $X_{\varphi}$ also contains $\psi$ since $\psi$ specializes to $\varphi$. Therefore,  we sometimes also denote $X_{\varphi}$ as $X_{\psi}$. Such a component is referred to as a TRSELP component.

We shall compute $X_{\psi}$ directly using the theory developed in \cite[Section 4]{dat2022ihes}. One might want to compare with the example of $GL_2$ (see Example \ref{Example: GL_2}) to understand what is happening below.

It turns out that the component $X_{\varphi}=X_{\psi}$ of $Z^1(W_F, \hat{G})_{\overline{\mathbb{Z}}_{\ell}}$ consists of the $L$-parameters whose restriction to $I_F^{\ell}$ and whose image in $Z^1(W_F, \pi_0(N_{\hat{G}}(\psi_{\ell})))$ is $\hat{G}$-conjugate to $(\psi_{\ell}, \overline{\psi})$. This is the content of the next lemma.

\begin{lemma}
%	Assume the center $Z(\hat{G})$ is smooth over $\overline{\mathbb{Z}}_{\ell}$.\footnote{This is to make sure that the results in \cite[Section 5.4, 5.5]{dat2022ihes} apply.} 
	We have an isomorphism of schemes
	\begin{equation}\label{Equation: X_psi_dat_4.6}
		X_{\psi} \cong \left(\hat{G} \times Z^1(W_F, N_{\hat{G}}(\psi_{\ell}))_{\psi_{\ell}, \overline{\psi}}\right)/C_{\hat{G}}(\psi_{\ell})_{\overline{\psi}} \qquad g\eta(-)g^{-1} \mapsfrom (g, \eta),
	\end{equation}
	where $Z^1(W_F, N_{\hat{G}}(\psi_{\ell}))_{\psi_{\ell}, \overline{\psi}}$  denotes the space of cocycles whose restriction to $I_F^{\ell}$ equals $\psi_{\ell}$ and whose image in $Z^1(W_F, \pi_0(N_{\hat{G}}(\psi_{\ell})))$ is $\overline{\psi}$; where $C_{\hat{G}}(\psi_{\ell})_{\overline{\psi}}$ is the (schematic) stabilizer of $\overline{\psi}$ in $C_{\hat{G}}(\psi_{\ell})$; and where $C_{\hat{G}}(\psi_{\ell})_{\overline{\psi}}$ acts on $(\hat{G} \times Z^1(W_F, N_{\hat{G}}(\psi_{\ell}))_{\psi_{\ell}, \overline{\psi}})$ by 
	$$(t, (g, \psi')) \mapsto (gt^{-1}, t\psi'(-)t^{-1}),$$
	where $t \in C_{\hat{G}}(\psi_{\ell})_{\overline{\psi}}$ and $(g, \psi') \in (\hat{G} \times Z^1(W_F, N_{\hat{G}}(\psi_{\ell}))_{\psi_{\ell}, \overline{\psi}})$.
\end{lemma}

\begin{proof}
	This is proven in \cite[Subsection 4.6]{dat2022ihes}.\footnote{To apply \cite[Subsection 4.6]{dat2022ihes}, we need to ensure that the center of $H^0$ in Dat's notation is smooth over $\overline{\mathbb{Z}}_{\ell}$ so that we can find a lifting $\psi$ such that $\psi$ fixes a Borel pair of $H^0$. However, this is automatic in our case because here $H^0=N_{\hat{G}}(\psi_{\ell})^0$ will turn out to be a torus (see the proof of Theorem \ref{Thm X}).} As a rough outline, we first notice that $Z^1(W_F, \hat{G})_{[\psi_{\ell}]}$, the space of cocycles whose restriction to $I_F^{\ell}$ is $\hat{G}$-conjugate to $\psi_{\ell}$, is open and closed in $Z^1(W_F, \hat{G})$\footnote{This is done by considering the restriction map $Z^1(W_F, \hat{G}) \to Z^1(I_F^{\ell}, \hat{G})$, since we know the connected components of $Z^1(I_F^{\ell}, \hat{G})$ quite well, thanks to \cite[Theorem 4.2]{dat2022ihes}. Note that $I_F^{\ell}$ has pro-order prime to $\ell$, so that we can apply \cite[Theorem 4.2]{dat2022ihes}. This is the reason that we consider $I_F^{\ell}$, the maximal subgroup of $W_F$ with pro-order prime to $\ell$.}. Next, we notice that
	$g\eta(-)g^{-1} \mapsfrom (g, \eta)$
    defines an isomorphism
    $$Z^1(W_F, \hat{G})_{[\psi_{\ell}]} \cong \left(\hat{G} \times Z^1(W_F, \hat{G})_{\psi_{\ell}}\right)/C_{\hat{G}}(\psi_{\ell}),$$
    where $Z^1(W_F, \hat{G})_{\psi_{\ell}}$ is the space of cocycles whose restriction to $I_F^{\ell}$ is $\psi_{\ell}$. Thus, $\psi$ is contained in the open and closed subscheme $Z^1(W_F, \hat{G})_{[\psi_{\ell}]}$ of $Z^1(W_F, \hat{G})$. However, $Z^1(W_F, \hat{G})_{[\psi_{\ell}]}$ is usually not connected, since it maps to the discrete space 
    $$Z^1(W_F/P_F, \pi_0(N_{\hat{G}}(\psi_{\ell}))).$$
    Nevertheless, it turns out that this is the only obstruction for being connected, i.e., if we moreover consider the subspace of $Z^1(W_F, \hat{G})_{[\psi_{\ell}]}$ that consists of $L$-parameters whose image in $Z^1(W_F/P_F, \pi_0(N_{\hat{G}}(\psi_{\ell})))$ is $\overline{\psi}$, it becomes connected. Therefore, we obtain the desired formular \eqref{Equation: X_psi_dat_4.6}.
\end{proof}

\begin{theorem}\label{Thm X}
	Let $\varphi \in Z^1(W_F, \hat{G}(\overline{\mathbb{F}}_{\ell}))$ be a TRSELP over $\overline{\mathbb{F}}_{\ell}$. Let $\psi \in Z^1(W_F, \hat{G}(\overline{\mathbb{Z}}_{\ell}))$ be any lifting of $\varphi$. 
%	Assume that the center $Z(\hat{G})$ is smooth over $\overline{\mathbb{Z}}_{\ell}$. 
	Then the connected component $X_{\varphi}=X_{\psi}$ of $Z^1(W_F, \hat{G})_{\overline{\mathbb{Z}}_{\ell}}$ containing $\varphi$ is isomorphic to 
	$$\left(\hat{G} \times C_{\hat{G}}(\psi_{\ell})^0 \times \mu\right)/\;C_{\hat{G}}(\psi_{\ell})_{\overline{\psi}},$$
	where
	\begin{enumerate}
		\item $C_{\hat{G}}(\psi_{\ell})^0$ is the identity component of the schematic centralizer $C_{\hat{G}}(\psi_{\ell})$. In addition, $C_{\hat{G}}(\psi_{\ell})=C_{\hat{G}}(\psi_{\ell})^0$ is a split torus $T$ over $\overline{\mathbb{Z}}_{\ell}$ with $\overline{\mathbb{F}}_{\ell}$-points $S=C_{\hat{G}(\overline{\mathbb{F}}_{\ell})}(\varphi(I_F))$.
		\item $\mu:=\left(T^{\Fr=(-)^q}\right)^0$ is the identity component of $T^{\Fr=(-)^q}$ \footnote{This is the subscheme of $T$ on which $\Fr$ acts by raising to $q$-th power, see Equation \eqref{Equation: Z^1(W, T)}. See also \cite[Example 3.14]{dat2022ihes}.} containing $1$, which is a product of some $\mu_{\ell^{k_i}}$ (the group scheme of $\ell^{k_i}$-th roots of unity over $\overline{\mathbb{Z}}_{\ell}$), $k_i \in \mathbb{Z}_{\geq 0}$. \footnote{Note that $\mu$ can be trivial, depending on $\hat{G}$ and some congruence relations between $q, \ell$.}
%		\item $C_{\hat{G}}(\psi_{\ell})_{\overline{\psi}}$ is the (schematic) stabilizer of $\overline{\psi}$ in $C_{\hat{G}}(\psi_{\ell})$.
	\end{enumerate}
    In other words, we have the following isomorphism of schemes over $\overline{\mathbb{Z}}_{\ell}$:
    $$X_{\varphi} \cong \left(\hat{G} \times T \times \mu\right)/T.$$
    We will specify in the next subsection what the $T$-action on $\left(\hat{G} \times T \times \mu\right)$ is.
    
    \begin{proof}
    	We begin with an outline of the proof. The idea is to use the formula \eqref{Equation: X_psi_dat_4.6}. We shall express the terms $Z^1(W_F, N_{\hat{G}}(\psi_{\ell}))_{\psi_{\ell}, \overline{\psi}}$ and $C_{\hat{G}}(\psi_{\ell})_{\overline{\psi}}$ on the right hand side of \eqref{Equation: X_psi_dat_4.6} explicitly. 
    	\begin{enumerate}
    		\item We show that $Z^1(W_F, N_{\hat{G}}(\psi_{\ell}))_{\psi_{\ell}, \overline{\psi}}$ is isomorphic to $Z^1_{Ad(\psi)}(W_F, N_{\hat{G}}(\psi_{\ell})^0)_{1_{I_F^{\ell}}}$, a certain subspace of the space of cocycles of the identity component $N_{\hat{G}}(\psi_{\ell})^0$ of $N_{\hat{G}}(\psi_{\ell})$.
    		\item We show that $C_{\hat{G}}(\psi_{\ell})$ is a split torus over $\overline{\mathbb{Z}}_{\ell}$ and that
            $N_{\hat{G}}(\psi_{\ell})^0=C_{\hat{G}}(\psi_{\ell})^0=C_{\hat{G}}(\psi_{\ell})$.
            \item Now we know that $N_{\hat{G}}(\psi_{\ell})^0$ is a torus, and we can compute $Z^1_{Ad(\psi)}(W_F, N_{\hat{G}}(\psi_{\ell})^0)$, as in Example \ref{Example GL_1}.
            \item We compute $Z^1_{Ad(\psi)}(W_F, N_{\hat{G}}(\psi_{\ell})^0)_{1_{I_F^{\ell}}}$ as the identity component of $Z^1_{Ad(\psi)}(W_F, N_{\hat{G}}(\psi_{\ell})^0)$.
            \item We show that $C_{\hat{G}}(\psi_{\ell})_{\overline{\psi}}=C_{\hat{G}}(\psi_{\ell})$.
    	\end{enumerate}  
    	
   
    	
    	(\textbf{Step 1}) We show that $\eta.\psi \mapsfrom \eta$ defines an isomorphism
    	$$Z^1(W_F, N_{\hat{G}}(\psi_{\ell}))_{\psi_{\ell}, \overline{\psi}} \cong Z^1_{Ad(\psi)}(W_F, N_{\hat{G}}(\psi_{\ell})^0)_{1_{I_F^{\ell}}}=:Z^1_{Ad(\psi)}(W_F, N_{\hat{G}}(\psi_{\ell})^0)_1$$
    	where $Z^1_{Ad(\psi)}(W_F, N_{\hat{G}}(\psi_{\ell}))$ means the space of cocycles with $W_F$ acting on $N_{\hat{G}}(\psi_{\ell})$ via conjugacy action through $\psi$, and the subscript $1_{I_F^{\ell}}$ or $1$ means the cocycles whose restriction to $I_F^{\ell}$ is trivial. 
    	Indeed, this is clear by unraveling the definitions: two cocycles whose restriction to $I_F^\ell$ are both $\psi_{\ell}$ differ by something whose restriction to $I_F^{\ell}$ is trivial; two cocycles whose pushforward to $Z^1(W_F, \pi_0(N_{\hat{G}}(\psi_{\ell})))$ are both $\overline{\psi}$ differ by something whose pushforward to $Z^1(W_F, \pi_0(N_{\hat{G}}(\psi_{\ell})))$ is trivial, i.e., which factors through the identity component $N_{\hat{G}}(\psi_{\ell})^0$.
    	
    	(\textbf{Step 2}) We show that $C_{\hat{G}}(\psi_{\ell})$ is a split torus over $\overline{\mathbb{Z}}_{\ell}$ and that $N_{\hat{G}}(\psi_{\ell})^0=C_{\hat{G}}(\psi_{\ell})^0=C_{\hat{G}}(\psi_{\ell})$. By \cite[Subsection 3.1]{dat2022ihes}, the centralizer $C_{\hat{G}}(\psi_{\ell})$ is generalized reductive (see Lemma \ref{Lem gen red}), hence split over $\overline{\mathbb{Z}}_{\ell}$, and $N_{\hat{G}}(\psi_{\ell})^0=C_{\hat{G}}(\psi_{\ell})^0$. Therefore, we can determine $C_{\hat{G}}(\psi_{\ell})$ by computing its $\overline{\mathbb{F}}_{\ell}$-points. Indeed,
    	$$C_{\hat{G}}(\psi_{\ell})(\overline{\mathbb{F}}_{\ell})=C_{\hat{G}(\overline{\mathbb{F}}_{\ell})}(\varphi(I_F^\ell))=C_{\hat{G}(\overline{\mathbb{F}}_{\ell})}(\varphi(I_F)),$$
    	where the last equality follows since $I_F/I_F^{\ell}$ does not contribute to the image of $\varphi$ (see Lemma \ref{Lem I_F^ell}). Therefore, $C_{\hat{G}}(\psi_{\ell})$ is a split torus over $\overline{\mathbb{Z}}_{\ell}$ with $\overline{\mathbb{F}}_{\ell}$-points $S=C_{\hat{G}(\overline{\mathbb{F}}_{\ell})}(\varphi(I_F))$. Denote $T=C_{\hat{G}}(\psi_{\ell})$. In particular, $C_{\hat{G}}(\psi_{\ell})$  is connected; hence, 
    	\begin{equation}\label{Equation: T}
    	N_{\hat{G}}(\psi_{\ell})^0=C_{\hat{G}}(\psi_{\ell})^0=C_{\hat{G}}(\psi_{\ell})=T.
    	\end{equation}
    	
    	
    	(\textbf{Step 3}) We compute that
    	$$Z^1_{Ad(\psi)}(W_F, N_{\hat{G}}(\psi_{\ell})^0)=Z^1_{Ad(\psi)}(W_F, T) \cong T \times T^{\Fr=(-)^q},$$
    	where the last isomorphism is given by $\eta \mapsto (\eta(\Fr), \eta(s_0))$, where $s_0 \in W_t^0$ is the topological generator of $I_t$ fixed before (see \cite[Example 3.14]{dat2022ihes}).
    	
    	(\textbf{Step 4}) We show that the identity component of $T^{\Fr=(-)^q}$ gives $\mu$ in the statement of the theorem. Note that $T^{\Fr=(-)^q}$ is a diagonalizable group scheme over $\overline{\mathbb{Z}}_{\ell}$ of dimension zero (this can be seen either by $\dim Z^1(W_F/P_F, T)=\dim T$, or by noticing that $\eta(s_0) \in T^{\Fr=(-)^q}$ is semisimple with finitely many possible eigenvalues), hence of the form $\prod_i\mu_{n_i}$ for some $n_i \in \mathbb{Z}_{\geq 0}$. Hence its identity component $(T^{\Fr=(-)^q})^0$ over $\overline{\mathbb{Z}}_{\ell}$ is of the form $\prod_i\mu_{\ell^{k_i}},$ with $k_i$ maximal such that $\ell^{k_i}$ divides $n_i$. Therefore, 
    	$$Z^1_{Ad(\psi)}(W_F, N_{\hat{G}}(\psi_{\ell})^0)_1 \cong (T \times T^{\Fr=(-)^q})^0 \cong T \times (T^{\Fr=(-)^q})^0 \cong T \times \mu,$$
    	(see Lemma \ref{Lem_Z^1()_1} for the first isomorphism) where $\mu$ is of the form $\prod_i\mu_{\ell^{k_i}}$.
    	
    	(\textbf{Step 5}) We show that $C_{\hat{G}}(\psi_{\ell})_{\overline{\psi}}=C_{\hat{G}}(\psi_{\ell})$. Recall that $C_{\hat{G}}(\psi_{\ell})$ acts on $Z^1(W_F, N_{\hat{G}}(\psi_{\ell}))$ by conjugation, inducing an action of $C_{\hat{G}}(\psi_{\ell})$ on $Z^1(W_F, \pi_0(N_{\hat{G}}(\psi_{\ell}))).$ In addition, $C_{\hat{G}}(\psi_{\ell})_{\overline{\psi}}$ is by definition the stabilizer of $\overline{\psi} \in Z^1(W_F, \pi_0(N_{\hat{G}}(\psi_{\ell})))$ in $C_{\hat{G}}(\psi_{\ell})$. Now $C_{\hat{G}}(\psi_{\ell})=T$ is connected, hence acting trivially on the component group $\pi_0(N_{\hat{G}}(\psi_{\ell}))$ and acting trivially on $Z^1(W_F, \pi_0(N_{\hat{G}}(\psi_{\ell})))$. Therefore, the stabilizer $C_{\hat{G}}(\psi_{\ell})_{\overline{\psi}}=C_{\hat{G}}(\psi_{\ell})$.
    	
    	Above all, we have 
        \begin{equation}
    		X_{\varphi} \cong (\hat{G} \times Z^1_{Ad(\psi)}(W_F, N_{\hat{G}}(\psi_{\ell})^0)_1)/C_{\hat{G}}(\psi_{\ell})_{\overline{\psi}} \cong (\hat{G} \times T \times \mu)/T.
        \end{equation}
    \end{proof}
\end{theorem}

\begin{eg}\label{Example: GL_2}
	
    Let $p=q=11, \ell=5, G=GL_2$.\footnote{They are chosen such that $\mu$ turns out to be non-trivial.} Let $F_2$ be the unique degree $2$ unramified extension of $F$. Then the Weil group of $F_2$ is $W_{F_2} \cong I_F \rtimes \left<\Fr^2\right>$.
    
    We define a tame character $\eta: W_{F_2}/P_F \to \overline{\mathbb{F}}_{\ell}^*$ as follows. It suffices to define $\eta$ on $I_F/P_F$ and $\left<\Fr^2\right>$ respectively.
    Let 
    $$\eta|_{I_F/P_F}: I_F/P_F \cong \prod_{p'\neq 11}\mathbb{Z}_{p'} \to \mathbb{Z}_3 \to \mathbb{Z}/3\mathbb{Z} \to \overline{\mathbb{F}_{5}}^*$$
    be the composition, where the last map is a non-trivial character $\chi: \mathbb{Z}/3\mathbb{Z} \to \overline{\mathbb{F}_{5}}^*$. Let $\eta(\Fr^2):=1$.
    
    Let $\varphi:=\Ind_{W_{F_2}}^{W_F}\eta$. $\varphi \in Z^1(W_F, \hat{G}(\overline{\mathbb{F}}_{\ell}))$ is an irreducible tame $L$-parameter, hence a TRSELP of $G=GL_2$. 
    
    To compute the connected component of $Z^1(W_F, \hat{G})$ containing $\varphi$ over $\overline{\mathbb{Z}}_{\ell}$, let us choose a lift $\psi$ of $\varphi$, as follows. First, let us define a lift $\tilde{\eta}: W_{F_2}/P_F \to \overline{\mathbb{Z}}_{\ell}^*$ of $\eta$, as follows. Let 
    $$\tilde{\eta}|_{I_F/P_F}: I_F/P_F \cong \prod_{p'\neq 11}\mathbb{Z}_{p'} \to \mathbb{Z}_3 \to \mathbb{Z}/3\mathbb{Z} \to \overline{\mathbb{Z}_{5}}^*$$
    be the composition, where the last map is a non-trivial character $\tilde{\chi}: \mathbb{Z}/3\mathbb{Z} \to \overline{\mathbb{Z}_{5}}^*$ lifting $\chi$. Let $\tilde{\eta}(\Fr^2):=1$. Next, define $\psi:=\Ind_{W_{F_2}}^{W_F}\tilde{\eta}$.
    
    Under a nice basis, we can express $\psi: W_F \to GL_2(\overline{\mathbb{Z}}_{\ell})$ in terms of matrices, as follows:
    $$\psi(s_0)=
    \begin{pmatrix}
    	\tilde{\chi}(1) & 0 \\
    	0 & \tilde{\chi}^q(1) \\
    \end{pmatrix}
    =
    \begin{pmatrix}
    	\zeta_3 & 0 \\
    	0 & \zeta_3^2 \\
    \end{pmatrix}\qquad 
    \psi(\Fr)=
    \begin{pmatrix}
    	0 & 1 \\
    	\tilde{\eta}(\Fr^2) & 0 \\
    \end{pmatrix}
    =
    \begin{pmatrix}
    	0 & 1 \\
    	1 & 0 \\
    \end{pmatrix},
    $$
    where $\zeta_3$ is a primitive $3$-rd root of unity of $\overline{\mathbb{Z}}_{\ell}$.
    
    Recall that
    $$X_{\varphi} \cong (\hat{G} \times Z^1_{Ad(\psi)}(W_F, N_{\hat{G}}(\psi_{\ell})^0)_1)/C_{\hat{G}}(\psi_{\ell})_{\overline{\psi}}.$$
    
    We see that 
    $$\psi(I_F^{\ell})=\left\{
    \begin{pmatrix}
    	1 & 0 \\
    	0 & 1 \\
    \end{pmatrix},
    \begin{pmatrix}
    	\zeta_3 & 0 \\
    	0 & \zeta_3^2 \\
    \end{pmatrix},
    \begin{pmatrix}
    	\zeta_3^2 & 0 \\
    	0 & \zeta_3 \\
    \end{pmatrix}
    \right\}.$$ 
    In this case, $T=C_{\hat{G}}(\psi_{\ell})$ is the diagonal torus of $GL_2$, $N_{\hat{G}}(\psi_{\ell})$ is the normalizer of $T$, $N_{\hat{G}}(\psi_{\ell})^0=T$, and $C_{\hat{G}}(\psi_{\ell})_{\overline{\psi}}=T$ since $T=C_{\hat{G}}(\psi_{\ell})$ fixes $\overline{\psi}$.
    
    It remains to compute $Z^1_{Ad(\psi)}(W_F, N_{\hat{G}}(\psi_{\ell})^0)_1 \cong Z^1_{Ad(\psi)}(W_F, T)_1$. We first compute $Z^1_{Ad(\psi)}(W_F, T)$ (without the subscript $1$). Indeed, by Equation \eqref{Equation: Z^1(W, T)}, 
    $$Z^1_{Ad(\psi)}(W_F, T) \cong T \times T^{\Fr=(-)^q},$$
    where $T^{\Fr=(-)^q}:=\{x \in T \;|\; \Fr.x=x^q\}$.
    In the case of $Z^1_{Ad(\psi)}(W_F, T)$, the $W_F$-action on $T$ is conjugation through $\psi$, so 
    $$T^{\Fr=(-)^q}=\{x \in T \;|\; \psi(\Fr)x\psi(\Fr)^{-1}=x^q\}
    =\left\{\begin{pmatrix}
    	t & 0 \\
    	0 & t^q \\
    \end{pmatrix} \;|\; t^{q^2-1}=1\right\}
    \cong \mu_{q^2-1}=\mu_{120}.$$
    
    $Z^1_{Ad(\psi)}(W_F, N_{\hat{G}}(\psi_{\ell})^0)_1$ turns out to be the connected component of $Z^1_{Ad(\psi)}(W_F, T) \cong T \times T^{\Fr=(-)^q}$ containing $1$. In our case,
    $$Z^1_{Ad(\psi)}(W_F, N_{\hat{G}}(\psi_{\ell})^0)_1 \cong (T \times \mu_{120})^0 \cong T \times \mu_5.$$
    
    Above all, 
    $$X_{\varphi} \cong (\hat{G} \times T \times \mu_5)/T,$$
    where $T$ is the diagonal torus of $\hat{G}=GL_2$.
    
    
     
\end{eg}

\subsection{The $T$-action on $(\hat{G} \times T \times \mu)$}\label{Subsection T-action}

We continue with the notations from the last subsection. Recall that $T:=C_{\hat{G}}(\psi|_{I_F^{\ell}})$ and $\mu:=\left(T^{\Fr=(-)^q}\right)^0$. The goal of this subsection is to specify the $T$-action on $(\hat{G} \times T \times \mu)$. Before that, let us record a lemma on several equivalent definitions of $T$.

\begin{lemma}\label{Lemma: T}
	$T:=C_{\hat{G}}(\psi|_{I_F^{\ell}}) = C_{\hat{G}}(\psi|_{I_F^{\ell}})^0 = C_{\hat{G}}(\psi|_{I_F}).$
\end{lemma}

\begin{proof}
	We have seen the first equality in Equation \eqref{Equation: T}. To see that $C_{\hat{G}}(\psi|_{I_F^{\ell}})=C_{\hat{G}}(\psi|_{I_F})$, we first note that $C_{\hat{G}}(\psi|_{I_F}) \subseteq C_{\hat{G}}(\psi|_{I_F^{\ell}})=:T$ is included in a commutative group scheme. Since $\psi|_{I_F}$ factors through the abelian group $I_F/P_F$, 
	$$\psi(I_F) \subseteq C_{\hat{G}}(\psi|_{I_F}) \subseteq T.$$
	Therefore, $C_{\hat{G}}(\psi|_{I_F}) \supseteq T$ since $T$ is commutative, and hence
	$$C_{\hat{G}}(\psi|_{I_F}) = T.$$
\end{proof}

Now let us make explicit the $T$-action on $(\hat{G} \times T \times \mu)$.

First, recall (see \cite[Subsection 4.6]{dat2022ihes}) that the component $X_{\varphi}=X_{\psi}$ consists of the $L$-parameters $\psi'$ such that $(\psi'_{\ell}, \overline{\psi'})$ is $\hat{G}$-conjugate to $(\psi_{\ell}, \overline{\psi})$. Hence $X_{\varphi}$ is isomorphic to 
$$(\hat{G} \times Z^1(W_F, N_{\hat{G}}(\psi_{\ell}))_{\psi_{\ell}, \overline{\psi}})/C_{\hat{G}}(\psi_{\ell})_{\overline{\psi}}$$
via $g\eta(-)g^{-1} \mapsfrom (g, \eta)$, with $C_{\hat{G}}(\psi_{\ell})_{\overline{\psi}}$ acting on $(\hat{G} \times Z^1(W_F, N_{\hat{G}}(\psi_{\ell}))_{\psi_{\ell}, \overline{\psi}})$ by 
$$(t, (g, \psi')) \mapsto (gt^{-1}, t\psi'(-)t^{-1}),$$
where $t \in C_{\hat{G}}(\psi_{\ell})_{\overline{\psi}} \cong T$ and $(g, \psi') \in (\hat{G} \times Z^1(W_F, N_{\hat{G}}(\psi_{\ell}))_{\psi_{\ell}, \overline{\psi}})$.

Next, recall that $\eta.\psi \mapsfrom \eta \mapsto (\eta(\Fr), \eta(s_0))$ induces isomorphisms
\begin{equation}\label{Equation: Z^1_Ad}
	Z^1(W_F, N_{\hat{G}}(\psi_{\ell}))_{\psi_{\ell}, \overline{\psi}} \cong Z^1_{Ad(\psi)}(W_F, N_{\hat{G}}(\psi_{\ell})^0)_1 \cong T \times \mu.
\end{equation}

For clarification, let us denote $T$ by $T_1$ when we consider $T$ as $C_{\hat{G}}(\psi_{\ell})_{\overline{\psi}}$, and denote $T$ by $T_2$ when we consider $T$ as a summand of $Z^1_{Ad(\psi)}(W_F, N_{\hat{G}}(\psi_{\ell})^0)_1$ (corresponding to the image of $\Fr$) via \eqref{Equation: Z^1_Ad}. We are going to make explicit the $T_1$-action on $(\hat{G} \times T_2 \times \mu)$.

Let us focus on the isomorphism $\eta.\psi \mapsfrom \eta$:
$$Z^1(W_F, N_{\hat{G}}(\psi_{\ell}))_{\psi_{\ell}, \overline{\psi}} \cong Z^1_{Ad(\psi)}(W_F, N_{\hat{G}}(\psi_{\ell})^0)_1.$$
Recall that $T_1 \subseteq \hat{G}$ acts on $Z^1(W_F, N_{\hat{G}}(\psi_{\ell}))_{\psi_{\ell}, \overline{\psi}}$ via conjugation. Hence, the above isomorphism induces an $T_1$-action on $Z^1_{Ad(\psi)}(W_F, N_{\hat{G}}(\psi_{\ell})^0)_1$, by
\begin{equation}\label{Equation: twisted conjugation}
	(t, \eta) \mapsto (t(\eta\psi(-)) t^{-1})\psi^{-1}.
\end{equation}

Hence in $(\hat{G} \times T_2 \times \mu)/T_1$, we compute by tracking the above isomorphisms that 
\begin{enumerate}
	\item $T_1$ acts on $\hat{G}$ via $(t, g) \mapsto gt^{-1}$.
	\item $T_1$ acts on $T_2 \subseteq (T_2 \times \mu)$ (corresponding to $\eta(\Fr)$) by twisted conjugacy (due to the isomorphisms $\eta.\psi \mapsfrom \eta \mapsto (\eta(\Fr), \eta(s_0))$), i.e., 
	$$(t, t') \mapsto \left(t(t'n)t^{-1}\right)n^{-1}=tt'(nt^{-1}n^{-1})=t(nt^{-1}n^{-1})t'=(tnt^{-1}n^{-1})t',$$
	where $n=\psi(\Fr)$; Note that $n$, a prior lies in $\hat{G}$, actually lies in $N_{\hat{G}}(T)$ (since $\Fr.s.\Fr^{-1}=s^q$ implies that $\psi(\Fr)$ normalizes $C_{\hat{G}}(\psi|_{I_F^{\ell}})=T$). To summarize, $t \in T_1$ acts on $T_2$ via multiplication by $tnt^{-1}n^{-1}$.
	\item $T_1$ acts trivially on $\mu \subseteq (T_2 \times \mu)$ (corresponding to $\eta(s_0)$). Indeed, $\eta(s_0) \in T$ and $\psi(s_0) \in T$. Therefore, the twisted (by $\psi$) conjugacy action \eqref{Equation: twisted conjugation} of $T_1$ on $\mu$ is trivial.
\end{enumerate}

On the other hand, recall that we have the natural $\hat{G}$-action on $Z^1(W_F, \hat{G})$ by conjugation, hence the $\hat{G}$-action on this component $X_{\varphi}$. Under the isomorphism $X_{\varphi} \cong (\hat{G} \times T_2 \times \mu)/T_1$, the $\hat{G}$-action becomes
$$(g', (g, t, m)) \mapsto  (g'g, t, m), \text{ for any } g' \in \hat{G} \text{ and } (g, t, m) \in (\hat{G} \times T_2 \times \mu)/T_1.$$

Note that the $T_1$-action and the $\hat{G}$-action on $(\hat{G} \times T_2 \times \mu)$ commute with each other; we thus have the following:\footnote{Since we have specified the action of $T_1=T$ on $T_2=T$, we go back to the notation $T$ in the statement of Proposition \ref{T times mu/T}.}

\begin{proposition}\label{T times mu/T}
	$$[X_{\varphi}/\hat{G}]=\left[\left((\hat{G} \times T \times \mu)/T\right)/\hat{G}\right] \cong \left[\left((\hat{G} \times T \times \mu)/\hat{G}\right)/T\right] \cong [(T \times \mu)/T],$$ 
	with $t \in T$ acting on $T$ via multiplication by $tnt^{-1}n^{-1}$, and $t \in T$ acting trivially on $\mu$. 
\end{proposition}

\subsection{Some lemmas}

\begin{lemma}\label{Lem generalizing}
	Let $\varphi' \in Z^1(W_t, \hat{G}(\overline{\mathbb{F}}_{\ell}))$. Then there exists $\psi' \in Z^1(W_t, \hat{G}(\overline{\mathbb{Z}}_{\ell}))$ such that $\psi'$ is a lift of $\varphi'$.
\end{lemma}

\begin{proof}
	In the statement, $Z^1(W_t, \hat{G})$ is the abbreviation for $Z^1(W_t, \hat{G})_{\overline{\mathbb{Z}}_{\ell}}$. Recall that $Z^1(W_t, \hat{G}) \to \overline{\mathbb{Z}}_{\ell}$ is flat (see \cite[Proposition 3.3]{dat2022ihes}), hence generalizing (see \cite[Stack, Tag 01U2]{stacks-project}). Therefore, given $\varphi' \in Z^1(W_t, \hat{G}(\overline{\mathbb{F}}_{\ell}))$, there exists $\xi \in Z^1(W_t, \hat{G}(\overline{\mathbb{Q}}_{\ell}))$ such that $\xi$ specializes to $\varphi'$. In other words, $\ker(\xi) \subseteq \ker(\varphi')$. We will show that $\xi: W_t \to \hat{G}(\overline{\mathbb{Q}}_{\ell})$ factors through  $\hat{G}(\overline{\mathbb{Z}}_{\ell})$.
	
	This is true by the following more general statement: Let $Y=\Spec(R)$ be an affine scheme over $\overline{\mathbb{Z}}_{\ell}$, let $y_{\eta} \in Y(\overline{\mathbb{Q}}_{\ell})$ specializing to $y_s \in Y(\overline{\mathbb{F}}_{\ell})$.  Then, $y_{\eta} \in Y(\overline{\mathbb{Q}}_{\ell})=\Hom(R, \overline{\mathbb{Q}}_{\ell})$ factors through $\overline{\mathbb{Z}}_{\ell}$.
	
    To prove the above statement, let $\mathfrak{p}:=\ker(y_\eta)$ and $\mathfrak{q}:=\ker(y_s)$ be the corresponding prime ideals. Then ``$y_{\eta}$ specializes to $y_s$" translates to ``$\mathfrak{p} \subseteq \mathfrak{q}$". Recall that we are going to show that $y_{\eta}: R \to \overline{\mathbb{Q}}_{\ell}$ factors through $\overline{\mathbb{Z}}_{\ell}$. We argue by contradiction. Otherwise there is some element $f \in R$ mapping to $\ell^{-m}u$ for some $m \in \mathbb{Z}_{\geq 1}$ and $u \in \overline{\mathbb{Z}}_{\ell}^*$. Hence 
    \begin{equation}\label{eq ell}
    	\ell^mu^{-1}f-1 \in \ker(y_{\eta}) \subseteq \ker(y_s).
    \end{equation}
    However, $\ell \in \ker(y_s)$ since $y_s$ lives on the special fiber. This together with equation \eqref{eq ell} implies that $1 \in \ker(y_s)$. Contradiction!
\end{proof}

\begin{lemma}\label{Lem gen red}
	The schematic centralizer $C_{\hat{G}}(\psi_{\ell})$ is a generalized reductive group scheme over $\overline{\mathbb{Z}}_{\ell}$.
\end{lemma}

\begin{proof}
	The idea is to invoke \cite[Lemma 3.2]{dat2022ihes}. We first note that $$C_{\hat{G}}(\psi_{\ell})=C_{\hat{G}}(\psi(I_F^{\ell})),$$
	where $C_{\hat{G}}(\psi(I_F^{\ell}))$ is the schematic centralizer of the subgroup $\psi(I_F^{\ell}) \subseteq \hat{G}(\overline{\mathbb{Z}}_{\ell})$ in $\hat{G}$. Indeed, this can be checked by the Yoneda Lemma on $R$-valued points for any $\overline{\mathbb{Z}}_{\ell}$-algebra $R$.
	
	Then, we can conclude by \cite[Lemma 3.2]{dat2022ihes}. Indeed, $\psi_{\ell}$ factors through some finite quotient $Q$ of $I_F^{\ell}$, which has order invertible in the base $\overline{\mathbb{Z}}_{\ell}$. Therefore, the assumptions of \cite[Lemma 3.2]{dat2022ihes} are satisfied (for details, see Remark \ref{Remark condition} below). 
\end{proof}

\begin{remark}\label{Remark condition}
	\begin{enumerate}
		\item While \cite[Lemma 3.2]{dat2022ihes} is phrased in the setting that $R$ is a normal subring of a number field, it still works for $\overline{\mathbb{Z}}_{\ell} \subseteq \overline{\mathbb{Q}}_{\ell}$ instead of $\mathbb{Z} \subseteq \mathbb{Q}$. Indeed, $\psi_{\ell}$ factors through some finite quotient $Q$ of $I_F^{\ell}$, say of order $|Q|=N$ (note that $N$ is coprime to $\ell$ since $Q$ is a quotient of $I_F^{\ell}$). Then we can use \cite[Lemma 3.2]{dat2022ihes} to conclude that $C_{\hat{G}}(\psi_{\ell})$ is generalized reductive over $\mathbb{Z}[1/pN]$. Hence $C_{\hat{G}}(\psi_{\ell})$ is also generalized reductive over $\overline{\mathbb{Z}}_{\ell}$ by base change.
		\item There is also a small issue that $\overline{\mathbb{Z}}_{\ell}$ is not finite over $\mathbb{Z}_{\ell}$, but this can be resolved since everything is already defined over some sufficiently large finite extension $\mathcal{O}$ of $\mathbb{Z}_{\ell}$.
	\end{enumerate}
\end{remark}

\begin{lemma}\label{Lem I_F^ell}
	$$C_{\hat{G}}(\psi_{\ell})(\overline{\mathbb{F}}_{\ell})=C_{\hat{G}(\overline{\mathbb{F}}_{\ell})}(\varphi(I_F^\ell))=C_{\hat{G}(\overline{\mathbb{F}}_{\ell})}(\varphi(I_F)).$$
\end{lemma}

\begin{proof}
	The first equation is by definition of the schematic centralizer and that $C_{\hat{G}}(\psi_{\ell})$ represents the set-theoretic centralizer. See Definition \ref{Definition: Schematic centralizer}.
	
	For the second equation, one way to conclude is using Lemma \ref{Lemma: T}. Alternatively, note that $\varphi|_{I_t}=\gamma_1 + ...+ \gamma_d$ is a direct sum of characters (since $I_t \cong \prod_{p'\neq p}\mathbb{Z}_{p'}$), so it suffices to show that each $\gamma_i$ is trivial on the summand $\mathbb{Z}_{\ell}$ of $I_t\cong \prod_{p'\neq p}\mathbb{Z}_{p'}$.
	Indeed,
	$$\Hom_{\Cont}(\mathbb{Z}_{\ell}, \overline{\mathbb{F}}_{\ell}^*)=\Hom_{\Cont}(\varprojlim\mathbb{Z}/\ell^n\mathbb{Z}, \overline{\mathbb{F}}_{\ell}^*)=\varinjlim\Hom(\mathbb{Z}/\ell^n\mathbb{Z}, \overline{\mathbb{F}}_{\ell}^*)=\{1\}.$$
\end{proof}

\begin{lemma}\label{Lem_Z^1()_1}
	$Z^1_{Ad(\psi)}(W_F, N_{\hat{G}}(\psi_{\ell})^0)_1 \cong (T \times T^{\Fr=(-)^q})^0.$
\end{lemma}

\begin{proof}
	We have omitted from the notations but here everything is over $\overline{\mathbb{Z}}_{\ell}$.
	Recall that $N_{\hat{G}}(\psi_{\ell})^0=C_{\hat{G}}(\psi_{\ell})^0=T$ and that
	$$Z^1_{Ad(\psi)}(W_F, N_{\hat{G}}(\psi_{\ell})^0) \cong T \times T^{\Fr=(-)^q}.$$
	
	By \cite[Section 5.4, 5.5]{dat2022ihes}, $Z^1_{Ad(\psi)}(W_F, N_{\hat{G}}(\psi_{\ell})^0)_1$ is connected (over $\overline{\mathbb{Z}}_{\ell}$). We need to check here that the assumptions of \cite[Section 5.4, 5.5]{dat2022ihes} are satisfied. Indeed, since $N_{\hat{G}}(\psi_{\ell})^0=T$ is a connected torus, the $W_t^0$-action on $T$ automatically fixes a Borel pair of $T$. Moreover, $s_0$ acts trivially on $N_{\hat{G}}(\psi_{\ell})^0=T$ via $\psi$, so in particular the action of $s_0$ (which is denoted by $s$ in \cite[Section 5.5]{dat2022ihes}) has order a power of $\ell$ (which is $1 = \ell^0$).
	
	Therefore, 
	$$Z^1_{Ad(\psi)}(W_F, N_{\hat{G}}(\psi_{\ell})^0)_1 \subseteq Z^1_{Ad(\psi)}(W_F, N_{\hat{G}}(\psi_{\ell})^0)^0 \cong (T \times T^{\Fr=(-)^q})^0.$$
	
	By \cite[Section 4.6]{dat2022ihes}, 
	$$Z^1_{Ad(\psi)}(W_F, N_{\hat{G}}(\psi_{\ell})^0)_1 \hookrightarrow Z^1_{Ad(\psi)}(W_F, N_{\hat{G}}(\psi_{\ell})^0)$$
	is open and closed. Indeed, this can be seen by considering the restriction to the prime-to-$\ell$ inertia $I_F^{\ell}$, and then using \cite[Theorem 4.2]{dat2022ihes}.
	
	Therefore, 
	$$Z^1_{Ad(\psi)}(W_F, N_{\hat{G}}(\psi_{\ell})^0)_1 = Z^1_{Ad(\psi)}(W_F, N_{\hat{G}}(\psi_{\ell})^0)^0 \cong (T \times T^{\Fr=(-)^q})^0.$$
	
\end{proof}




\section{Main Theorem: description of $[X_{\varphi}/\hat{G}]$}\label{Section X/hatG}

The goal of this section is to describe $[X_{\varphi}/\hat{G}]$ explicitly (see Theorem \ref{Thm X/G} for the precise statement).

Let $F$ be a non-archimedean local field, $G$ be a connected split reductive group over $F$. Let $\varphi \in Z^1(W_F, \hat{G}(\overline{\mathbb{F}}_{\ell}))$ be a TRSELP. Recall that this means that the centralizer 
$$C_{\hat{G}(\overline{\mathbb{F}}_{\ell})}(\varphi(I_F)) =: S \subseteq \hat{G}(\overline{\mathbb{F}}_{\ell})$$ 
is a maximal torus, and $\varphi(\Fr) \in N_{\hat{G}}(S)$ gives rise to an element $w=\overline{\varphi(\Fr)} \in N_{\hat{G}}(S)/S$ in the Weyl group (and $\varphi$ is tame and elliptic). 

%Assume that $Z(\hat{G})$ is finite.
%\begin{enumerate}
%	\item \label{assumption 1} The center $Z(\hat{G})$ is smooth over $\overline{\mathbb{Z}}_{\ell}$.
%	\item \label{assumption 2} $Z(\hat{G})$ is finite.
%%	\item \label{assumption 3} $\ell$ doesn't divide the order of $w=\overline{\varphi(\Fr)}$ in the Weyl group $N_{\hat{G}}(S)/S$.
%\end{enumerate}

Let $\psi \in Z^1(W_F, \hat{G}(\overline{\mathbb{Z}}_{\ell}))$ be any lifting of $\varphi$. Let $\psi_{\ell}$ denote the restriction $\psi|_{I_F^{\ell}}$, and $\overline{\psi}$ denote the image of $\psi$ in $Z^1(W_F, \pi_0(N_{\hat{G}}(\psi_{\ell})))$. Recall that the schematic centralizer $C_{\hat{G}}(\psi_{\ell})=T$ is a split torus over $\overline{\mathbb{Z}}_{\ell}$ with $\overline{\mathbb{F}}_{\ell}$-points $C_{\hat{G}(\overline{\mathbb{F}}_{\ell})}(\varphi(I_F)) = S$. 

For later use, we record the following lemma -- $w$ can also be defined in terms of $\psi$ instead of $\varphi$. This is helpful because we will reduce to a computation on the special fiber later. First, notice that since $T$ is a split torus over $\overline{\mathbb{Z}}_{\ell}$ with $\ell \neq 2$, we can identify
$$\left(N_{\hat{G}}(T)/T\right)(\overline{\mathbb{Z}}_{\ell}) \cong \left(N_{\hat{G}}(T)/T\right)(\overline{\mathbb{F}}_{\ell}),$$
and denote it by $\Omega$ (see Lemma \ref{Lem Wely} below).\footnote{Lemma \ref{Lem Wely} below shows that $N_{\hat{G}}(T)/T$ is representable by a constant group scheme over $\overline{\mathbb{Z}}_{\ell}$. Therefore, we will abuse notations and use 
	$\Omega, N_{\hat{G}}(T)/T, N_{\hat{G}}(S)/S$
	interchangeably.}


\begin{lemma}\label{Lemma w}
	The images of $\varphi(\Fr)$ and $\psi(\Fr)$ in the Weyl group $\Omega$ agree, hence giving a well defined element $w$ in the Weyl group $\Omega$.
\end{lemma}

\begin{proof}
	Let 
	$$\Omega=\left(N_{\hat{G}}(T)/T\right)(\overline{\mathbb{Z}}_{\ell}) = \left(N_{\hat{G}}(T)/T\right)(\overline{\mathbb{F}}_{\ell})$$ 
	as above and $\underline{\Omega}$ be the associated constant group scheme (see Lemma \ref{Lem Wely} below). Since $\psi$ is a lift of $\varphi$, $\psi(\Fr)$ specializes to $\varphi(\Fr)$ in $N_{\hat{G}}(T)$. Then the lemma follows since 
%	$$N_{\hat{G}}(T) \to N_{\hat{G}}(T)/T=\underline{\Omega}$$
%	is a morphism of schemes; hence, 
	the diagram
	$$
	\begin{tikzcd}
		{N_{\hat{G}}(T)(\overline{\mathbb{Z}}_{\ell})} && {N_{\hat{G}}(T)(\overline{\mathbb{F}}_{\ell})} \\
		\\
		{\underline{\Omega}(\overline{\mathbb{Z}}_{\ell})=\Omega} && {\underline{\Omega}(\overline{\mathbb{F}}_{\ell})=\Omega}
		\arrow[from=1-1, to=1-3]
		\arrow[from=1-1, to=3-1]
		\arrow[from=3-1, to=3-3]
		\arrow[from=1-3, to=3-3]
	\end{tikzcd}
    $$
    commutes.
\end{proof}

Our main theorem is the following.

\begin{theorem}\label{Thm X/G}
	Assume that $Z(\hat{G})$ is finite.
	Let $X_{\varphi}$($=X_{\psi}$) be the connected component of $Z^1(W_F, \hat{G})_{\overline{\mathbb{Z}}_{\ell}}$ containing $\varphi$ (hence also containing $\psi$). Then we have isomorphisms of quotient stacks
	\begin{equation}\label{Equation: X_phi}
		[X_{\varphi}/\hat{G}] \cong [(T \times \mu)/T] \cong [*/{C_T(n)}] \times \mu \cong [*/S_{\psi}] \time \mu,
	\end{equation}
	where $C_T(n)$ is the schematic centralizer of $n=\psi(\Fr)$ in $T=C_{\hat{G}}(\psi|_{I_F^{\ell}})$, $\mu=(T^{\Fr=(-)^q}) \cong \prod_{i=1}^m\mu_{\ell^{k_i}}$ for some $k_i \in \mathbb{Z}_{\geq 1}$, $m \in \mathbb{Z}_{\geq 0}$ is a product of group schemes of roots of unity, and $S_{\psi}:=C_{\hat{G}}(\psi)$ is the schematic centralizer of $\psi$ in $\hat{G}$. 
	
	If we moreover assume that
    $\ell$ does not divide the order of $w=\overline{\varphi(\Fr)}$ in the Weyl group $N_{\hat{G}}(S)/S$,
	then 
	$$[X_{\varphi}/\hat{G}] \cong [(T \times \mu)/T] \cong [*/\underline{S_{\varphi}(\overline{\mathbb{F}}_{\ell})}] \times \mu,$$
	where $S_{\varphi}(\overline{\mathbb{F}}_{\ell})=C_{\hat{G}(\overline{\mathbb{F}}_{\ell})}(\varphi(W_F))$, and $\underline{S_{\varphi}(\overline{\mathbb{F}}_{\ell})}$ is the corresponding constant group scheme.\footnote{By abuse of notation, we sometimes denote $S_{\varphi}(\overline{\mathbb{F}}_{\ell})$ simply by $S_{\varphi}$.}
\end{theorem}

\begin{proof}
	Recall that $X_{\varphi}$ is isomorphic to the contracted product 
	$$(\hat{G} \times Z^1(W_F, N_{\hat{G}}(\psi_{\ell}))_{\psi_{\ell}, \overline{\psi}})/C_{\hat{G}}(\psi_{\ell})_{\overline{\psi}},$$ 
	and that $\eta.\psi \mapsfrom \eta \mapsto (\eta(\Fr), \eta(s_0))$ induces isomorphisms
	$$Z^1(W_F, N_{\hat{G}}(\psi_{\ell}))_{\psi_{\ell}, \overline{\psi}} \cong Z^1_{Ad(\psi)}(W_F, N_{\hat{G}}(\psi_{\ell})^0)_1 \cong T \times \mu.$$
	
	This implies that $[X_{\varphi}/\hat{G}] \cong [(T \times \mu)/T]$ with $T$ acting on $T$ by twisted conjugacy:
	$$(t, t') \mapsto \left(t(t'n)t^{-1}\right)n^{-1}=tt'(nt^{-1}n^{-1})=t(nt^{-1}n^{-1})t'=(tnt^{-1}n^{-1})t',$$
	where $n=\psi(\Fr)$. In other words, $T$ acts on $T$ via multiplication by $tnt^{-1}n^{-1}$. In addition, $T$ acts trivially on $\mu$ (see Proposition \ref{T times mu/T}). Therefore, we are reduced to computing $[T/T]$ with respect to a nice action of the split torus $T$ on $T$.
	
    Consider the morphism
	$$f: T^{(1)} := T \longrightarrow T =: T^{(2)} \qquad s \longmapsto sns^{-1}n^{-1}.\footnote{For clarification, let us denote the source torus $T$ as $T^{(1)}$ and the target torus $T$ as $T^{(2)}$.}$$
	This is surjective on $\overline{\mathbb{F}}_{\ell}$-points by our assumption that $Z(\hat{G})$ is finite and $\varphi$ is elliptic (see Lemma \ref{Lem epic} below). Hence $f$ is an epimorphism in the category of diagonalizable $\overline{\mathbb{Z}}_{\ell}$-group schemes (see Lemma \ref{Lem epic} below). Therefore, $f$ induces an isomorphism 
	\begin{equation}\label{eq_T}
		T^{(1)}/\ker(f) \cong T^{(2)}
	\end{equation}
	as diagonalizable $\overline{\mathbb{Z}}_{\ell}$-group schemes. Moreover, if we let $t \in T$ act on $T^{(1)}$ by left multiplication by $t$, and on $T^{(2)}$ via multiplication by $(tnt^{-1}n^{-1})$, this isomorphism induced by $f$ is $T$-equivariant.
	
	Note that $T^{(1)}=T$ is commutative, so the $T$-action (via multiplication by $tnt^{-1}n^{-1}$) and the $\ker(f)$-action (via left multiplication) on $T$ commute with each other. Hence by the $T$-equivariant isomorphism \eqref{eq_T}, we have
	$$[T/T] = [T^{(2)}/T] \cong \left[\left(T^{(1)}/\ker(f)\right)/T\right] \cong \left[\left(T^{(1)}/T\right)/\ker(f)\right] \cong [*/\ker(f)] = [*/C_T(n)].$$ 
	
%	Finally, notice that 
%	$$\ker(f)=C_T(n)=C_{\hat{G}}(\psi),$$
%	since $T=C_{\hat{G}}(\psi|_{I_F^{\ell}})$ (\textcolor{red}{is this same as $C_{\hat{G}}(\psi|_{I_F})$ ?}) and $n=\psi(\Fr)$ (\textcolor{red}{see Lemma ? below}).
	
%	Now let's proof the last assertion: To show $\ker(f)=\underline{S_{\varphi}}$ under assumption \ref{assumption 3} (Need adjust) -- $\ell$ does not divide the order of $w$ in the Weyl group $N_{\hat{G}}(T)/T$ (\textcolor{red}{Use T or S?}). Then $\ker(f) \cong \underline{S_{\varphi}}$ is the constant group scheme of the finite abelian group $S_{\varphi}=C_{\hat{G}(\overline{\mathbb{F}}_{\ell})}(\varphi(W_F))$, \textcolor{red}{see Lemma below}. We win!

    Moreover, recall that we have $T:=C_{\hat{G}}(\psi|_{I_F^{\ell}})=C_{\hat{G}}(\psi|_{I_F})$ (see Lemma \ref{Lemma: T}). So 
    $$C_T(n) \cong C_{\hat{G}}(\psi(I_F), \psi(\Fr)) \cong C_{\hat{G}}(\psi) =: S_{\psi}.$$

    For the second part of the theorem, see Lemma \ref{Lem ker(f)} below.
	
\end{proof}



\begin{lemma}\label{Lem epic}
	The morphism 
	$$f: T^{(1)} = T \longrightarrow T = T^{(2)} \qquad s \longmapsto sns^{-1}n^{-1}$$
	is epimorphic in the category of diagonalizable $\overline{\mathbb{Z}}_{\ell}$-group schemes. Moreover, $f$ induces an isomorphism $T^{(1)}/\ker(f) \cong T^{(2)}$ as diagonalizable $\overline{\mathbb{Z}}_{\ell}$-group schemes.
\end{lemma}

\begin{proof}
	Recall that $T$ is a split torus over $\overline{\mathbb{Z}}_{\ell}$, hence a diagonalizable $\overline{\mathbb{Z}}_{\ell}$-group scheme. Note that $f$ is a morphism of $\overline{\mathbb{Z}}_{\ell}$-group schemes and hence a morphism of diagonalizable $\overline{\mathbb{Z}}_{\ell}$-group schemes. Recall that the category of diagonalizable $\overline{\mathbb{Z}}_{\ell}$-group schemes is equivalent to the category of abelian groups (see \cite[p70, Section 5]{brochard2014autour} or \cite{conrad2014reductive}) via
	$$D \mapsto \Hom_{\overline{\mathbb{Z}}_{\ell}-GrpSch}(D, \mathbb{G}_m),$$
	and the inverse is given by 
	$$\overline{\mathbb{Z}}_{\ell}[M] \mapsfrom M,$$
	where $\overline{\mathbb{Z}}_{\ell}[M]$ is the group algebra of $M$ with $\overline{\mathbb{Z}}_{\ell}$-coefficients.
	
	Therefore, we can argue in the category of abelian groups via the above equivalence of categories: $f$ is epimorphic if and only if the map $f^*$ in the category of abelian groups is monomorphic. Since $\varphi$ is elliptic and $Z(\hat{G})$ is finite, $S_{\varphi}$ is finite; hence, 
	$$\ker(f)(\overline{\mathbb{F}}_{\ell})=C_T(n)(\overline{\mathbb{F}}_{\ell})=S_{\varphi}(\overline{\mathbb{F}}_{\ell})$$
	is finite (where the first equality is by definition of $f$, and the second equality holds because $T(\overline{\mathbb{F}}_{\ell})=C_{\hat{G}(\overline{\mathbb{F}}_{\ell})}(\varphi(I_F))$ and $n = \psi(\Fr)$ maps to $\varphi(\Fr) \in \hat{G}(\overline{\mathbb{F}}_{\ell})$ by Lemma \ref{Lemma w}). Accordingly, $\coker(f^*)$ is finite. Therefore, 
	$$f^*:\Hom(T^{(2)}, \mathbb{G}_m) \to \Hom(T^{(1)}, \mathbb{G}_m)$$
	is injective (i.e., monomorphism). Indeed, otherwise $\ker(f^*)$ would be a nonzero sub-$\mathbb{Z}$-module of the finite free $\mathbb{Z}$-module $\Hom(T^{(2)}, \mathbb{G}_m)$, hence a free $\mathbb{Z}$-module of positive rank, which contradicts $\coker(f^*)$ being finite.
	
	The statement on the quotient follows from the corresponding result in the category of abelian groups: $f^*$ induces an isomorphism
	$$\Hom(T^{(1)}, \mathbb{G}_m)/\Hom(T^{(2)}, \mathbb{G}_m) \cong \coker(f^*)$$
	(see \cite[p71, Subsection 5.3]{brochard2014autour}).
\end{proof}

\begin{lemma}\label{Lem ker(f)}
	Assume that $\ell$ does not divide the order of $w$ in the Weyl group $\Omega$. Then, $\ker(f) \cong \underline{S_{\varphi}(\overline{\mathbb{F}}_{\ell})}$ is the constant group scheme of the finite abelian group $S_{\varphi}(\overline{\mathbb{F}}_{\ell})=C_{\hat{G}(\overline{\mathbb{F}}_{\ell})}(\varphi(W_F))$.
\end{lemma}

\begin{proof}
	We recall the following fact: Let $H$ be a smooth affine group scheme over some ring $R$, let $\Gamma$ be a finite group whose order is invertible in $R$. Then the fixed point functor $H^{\Gamma}$ is representable by a scheme which is smooth over $R$.
	
	For a proof of the above fact, see \cite[Proposition 3.4]{edixhoven1992neron} or \cite[Lemma A.1, A.13]{dhkm2020moduli}.
	
	In our case, let $H=T$, $\Gamma=\left<w\right>$ be the subgroup of the Weyl group $N_{\hat{G}}(T)/T$ generated by $w$. Hence $$\ker(f)=C_T(n)=H^{\Gamma}$$
    is smooth over $\overline{\mathbb{Z}}_{\ell}$. Therefore, $\ker(f)$ is finite étale over $\overline{\mathbb{Z}}_{\ell}$ (because it is smooth of relative dimension $0$ over $\overline{\mathbb{Z}}_{\ell}$, which can be checked on $\overline{\mathbb{F}}_{\ell}$-points). Hence, $\ker(f)$ is a constant group scheme over $\overline{\mathbb{Z}}_{\ell}$, since $\overline{\mathbb{Z}}_{\ell}$ has no non-trivial finite étale covers.
	
	Since $\ker(f)$ is constant, we can determine it by computing its $\overline{\mathbb{F}}_{\ell}$-points:
	\begin{equation}\label{Eq ker(f)}
	\ker(f)(\overline{\mathbb{F}}_{\ell})=C_{T(\overline{\mathbb{F}}_{\ell})}(n)=C_{\hat{G}(\overline{\mathbb{F}}_{\ell})}(\varphi(W_F)),
	\end{equation}
	where the last equality follows by noticing that $T(\overline{\mathbb{F}}_{\ell})=C_{\hat{G}(\overline{\mathbb{F}}_{\ell})}(\varphi(I_F))$ and $n=\psi(\Fr)$ maps to $\varphi(\Fr) \in \hat{G}(\overline{\mathbb{F}}_{\ell})$ (by Lemma \ref{Lemma w}).
	
	Finally, note that by our TRSELP assumption, $C_{\hat{G}(\overline{\mathbb{F}}_{\ell})}(\varphi(I_F))$ is (the $\overline{\mathbb{F}}_{\ell}$-points of) a torus. Hence $S_{\varphi}(\overline{\mathbb{F}}_{\ell})=C_{\hat{G}(\overline{\mathbb{F}}_{\ell})}(\varphi(W_F))$ is abelian, hence finite abelian, as we noticed in the proof of the previous lemma that $S_\varphi(\overline{\mathbb{F}}_{\ell})$ is finite (since $\varphi$ is elliptic and $Z(\hat{G})$ is finite).
\end{proof}

\begin{lemma}\label{Lem Wely}
	Let $\hat{G}$ be a connected reductive group scheme over $\overline{\mathbb{Z}}_{\ell}$, and let $T$ be a maximal torus of $\hat{G}$. Then $N_{\hat{G}}(T)/T$ is split over $\overline{\mathbb{Z}}_{\ell}$.
\end{lemma}

\begin{proof}
	By \cite[Proposition 3.2.8]{conrad2014reductive}, $N_{\hat{G}}(T)/C_{\hat{G}}(T)$ is finite étale over $\overline{\mathbb{Z}}_{\ell}$ and hence split over $\overline{\mathbb{Z}}_{\ell}$. In our case, $C_{\hat{G}}(T)=T$ since $\hat{G}$ is connected (for example, use the third paragraph of the proof of \cite[Proposition 3.1.12]{conrad2014reductive}).
\end{proof}
	


	\documentclass{article}

\special{dvipdfmx:config z 0}

\usepackage{amsmath,amssymb,amsfonts,amsthm,extarrows}
\usepackage{mathtools}
\usepackage{enumitem}
\usepackage{stmaryrd}
\usepackage{tikz-cd} 

\usepackage{color}
\newcommand{\red}[1]{\textcolor{red}{#1}}
\newcommand{\blue}[1]{\textcolor{blue}{#1}}

\usepackage{nameref}

\usepackage{graphicx}
\graphicspath{ {./images/} }

\usepackage{soul}

%%%% todo notes %%%%
\usepackage[colorinlistoftodos,textsize=footnotesize]{todonotes}
\setlength{\marginparwidth}{2.5cm}
\newcommand{\leftnote}[1]{\reversemarginpar\marginnote{\footnotesize #1}}
\newcommand{\rightnote}[1]{\normalmarginpar\marginnote{\footnotesize #1}\reversemarginpar}


\usepackage[colorlinks]{hyperref}

\newtheorem*{remark}{Remark}
\newtheorem{theorem}{Theorem}
\newtheorem{lemma}{Lemma}
\newtheorem{question}{Question}
\newtheorem{answer}{Answer}
\newtheorem{proposition}{Proposition}
\newtheorem{definition}{Definition}
\newtheorem{exer}{Exercise}
\newtheorem{corollary}{Corollary}
\newtheorem{example}{Example}
\newtheorem{warning}{Warning}

\DeclareMathOperator{\cInd}{\operatorname{c-Ind}}
\DeclareMathOperator{\Ind}{\operatorname{Ind}}
\newcommand{\Res}{\operatorname{Res}}
\newcommand{\Hom}{\operatorname{Hom}}
\newcommand{\Rep}{\operatorname{Rep}}
\newcommand{\End}{\operatorname{End}}
\newcommand{\GL}{\operatorname{GL}}
\newcommand{\diag}{\operatorname{diag}}
\newcommand{\Mod}{\operatorname{Mod}}
\newcommand{\Irr}{\operatorname{Irr}}
\newcommand{\Modr}{\operatorname{Mod-}}
\newcommand{\Modl}{\operatorname{-Mod}}
\newcommand{\Ob}{\operatorname{Ob}}


\begin{document}
	
	\section{Proof of the Main Theorem \ref{Thm Main} modulo Theorem \ref{Thm Cusp Red} \ref{Thm Hom} \ref{Thm Proj}}
	Let $\mathcal{G}$ be a split reductive group scheme over $\mathbb{Z}$, which is simply connected. Let $G:=\mathcal{G}(\mathbb{Q}_p)$. For simplicity, I assume that $p$ is greater than the Coxeter number of $\mathcal{G}$ (See Theorem \ref{Thm Broue} for reason).
	
	Let $x$ be a vertex of the Bruhat-Tits building $\mathcal{B}(\mathcal{G}, \mathbb{Q}_p)$, $G_x$ the parahoric subgroup associated to $x$, $G_x^+$ its pro-unipotent radical. Recall that $\overline{G_x}:=G_x/G_x^+$ is a generalized Levi subgroup of $\mathcal{G}(\mathbb{F}_p)$ with root system $\Phi_x$, see \cite[Theorem 3.17]{rabinoff2003bruhat}. 
	
	Let $\Lambda=\overline{\mathbb{Z}_\ell}$, with $\ell \neq p$. Let $\rho \in \Rep_{\Lambda}(G_x)$ be an irreducible representation of $G_x$, which is trivial on $G_x^+$ and whose reduction to the finite group of Lie type $\overline{G_x}=G_x/G_x^+$ is  
	regular cuspidal. Here \textbf{regular cuspidal} (See Definition \ref{Def regular cuspidal} for precise definition.) means $\rho$ is cuspidal (Which I think follows from regularity? \blue{No, it doesn't. For example, the irreducible principal series $\Ind_B^G\chi$ for $G=GL_2$.}) and lies in a \textbf{regular block} of $\Rep_{\Lambda}(\overline{G_x})$, in the sense of \cite{broue1990isometries}. The reason we want the regularity assumption is that we want to work with a block of $\Rep_{\Lambda}(\overline{G_x})$ which consists purely of cuspidal representations. See Section \ref{Sec Reg Cusp} for details. We make this a definition for later use.
	
	\begin{definition}
		Let $\rho \in \Rep_{\Lambda}(G_x)$. We say $\rho$ \textbf{has cuspidal reduction} (resp. \textbf{has regualr cuspidal reduction}), if $\rho$ is trivial on $G_x^+$ and whose reduction to the finite group of Lie type $\overline{G_x}=G_x/G_x^+$ is cuspidal (resp. regular cuspidal). Let's denote the reduction of $\rho$ modulo $G_x^+$ by $\overline{\rho} \in \Rep_{\Lambda}(\overline{G_x})$.
	\end{definition}
	
	Let $\mathcal{B}_{x,1}$ be the block of $\Rep_{\Lambda}(G_x)$ containing $\rho$. Let $\mathcal{C}_{x,1}$ be the block of $\Rep_{\Lambda}(G)$ containing $\pi:=\cInd_{G_x}^G\rho$. Now I can state the Main Theorem of this paper.
	
	\begin{theorem}[Main Theorem]\label{Thm Main}
		Let $x$ be a vertex of the Bruhat-Tits building $\mathcal{B}(\mathcal{G}, \mathbb{Q}_p)$. Let $\rho \in \Rep_{\Lambda}(G_x)$ which has regular cuspidal reduction. Let $\mathcal{B}_{x,1}$ be the block of $\Rep_{\Lambda}(G_x)$ containing $\rho$. Let $\mathcal{C}_{x,1}$ be the block of $\Rep_{\Lambda}(G)$ containing $\pi:=\cInd_{G_x}^G\rho$. Then the compact induction $\cInd_{G_x}^G$ induces an equivalence of categories $\mathcal{B}_{x,1} \simeq \mathcal{C}_{x,1}$. 
	\end{theorem}
	
	As mentioned before, the reason we want the regular cuspidal assumption is the following Theorem. 
	
	\begin{theorem}\label{Thm Cusp Red}
		Let $\rho \in \Rep_{\Lambda}(G_x)$ be an irreducible representation of $G_x$, which has regular cuspidal reduction. Let $\mathcal{B}_{x,1}$ be the block of $\Rep_{\Lambda}(G_x)$ containing $\rho$. Then any $\rho' \in \mathcal{B}_{x,1}$ has cuspidal reduction.
	\end{theorem}
	
	The proof of the Main Theorem \ref{Thm Main} basically splits into two parts -- fully faithfulness and essentially surjectivity. It is convenient to have the following Theorem available at an early stage, which implies fully faithfulness immediately and is also used in the proof of essentially surjectivity.
	
	\begin{theorem}\label{Thm Hom}
		Let $x, y$ be two vertices of the Bruhat-Tits building of $G$. Let $\rho_1$ be a representation of the parahoric $G_x$ which is trivial on the pro-unipotent radical $G_x^+$. Let $\rho_2$ be a representation of $G_y$ which is trivial on $G_y^+$. Assume one of them has cuspidal reduction. Then exactly one of the following happens:
		\begin{enumerate}
			\item If there exists an element $g \in G$ such that $g.x=y$, then
			$$\Hom_G(\cInd_{G_x}^G\rho_1, \cInd_{G_y}^G\rho_2)=\Hom_{G_x}(\rho_1, {^g\rho_2}).$$
			\item If there is no elements $g \in G$ such that $g.x=y$, then
			$$\Hom_G(\cInd_{G_x}^G\rho_1, \cInd_{G_y}^G\rho_2)=0.$$
		\end{enumerate}
	\end{theorem}
	
	The proof of the above Theorem is basically a computation using Mackey's formula. see Section \ref{Sec Pf Thm Hom}.
	
	Now we proceed by steps towards our goal: The compact induction $\cInd_{G_x}^G$ induces an equivalence of categories $\mathcal{B}_{x,1} \simeq \mathcal{C}_{x,1}$. 
	
	First, we show that $\cInd_{G_x}^G: \mathcal{B}_{x,1} \to \mathcal{C}_{x,1}$ is well-defined. We need to show that the image of $\mathcal{B}_{x,1}$ under $\cInd_{G_x}^G$ lies in $\mathcal{C}_{x,1}$. By Theorem \ref{Thm Cusp Red} and Theorem \ref{Thm Hom} above, $$\cInd_{G_x}^G|_{\mathcal{B}_{x,1}}: \mathcal{B}_{x,1} \to \Rep_{\Lambda}(G)$$
	is fully faithful (See Lemma \ref{Lem Thm Hom implies fully faithful}, note here we used Theorem \ref{Thm Cusp Red} that any representation in $\mathcal{B}_{x,1}$ has cuspidal reduction, so that we can apply Theorem \ref{Thm Hom}), hence an equivalence onto the essential image. Since $\mathcal{B}_{x,1}$ is indecomposable as an abelian category, so is its essential image (See Lemma \ref{Lem Indec}), hence its essential image is contained in a single block of $\Rep_{\Lambda}(G)$. But such a block must be $\mathcal{C}_{x,1}$ since $\cInd_{G_x}^G$ maps $\rho$ to $\pi \in \mathcal{C}_{x,1}$. Therefore, $\cInd_{G_x}^G: \mathcal{B}_{x,1} \to \mathcal{C}_{x,1}$ is well-defined.
	
	Second, we show that $\cInd_{G_x}^G: \mathcal{B}_{x,1} \to \mathcal{C}_{x,1}$ is fully faithful. This is already noticed in the proof of "well-defined" in the last paragraph. Indeed, 
	$$\Hom_G(\cInd_{G_x}^G\rho_1, \cInd_{G_x}^G\rho_2)=\Hom_{G_x}(\rho_1, \rho_2)$$
	by Theorem \ref{Thm Cusp Red} and Theorem \ref{Thm Hom} (See Lemma \ref{Lem Thm Hom implies fully faithful}.). Therefore, $\cInd_{G_x}^G: \mathcal{B}_{x,1} \to \mathcal{C}_{x,1}$ is fully faithful.
	
	Finally, we show that $\cInd_{G_x}^G: \mathcal{B}_{x,1} \to \mathcal{C}_{x,1}$ is essentially surjective. This will occupy the rest of this section. 
	
	The idea is to find a projective generator of $\mathcal{C}_{x,1}$ and show that it is in the essential image. Fix a vertex $x$ of the Bruhat-Tits building $\mathcal{B}(\mathcal{G}, \mathbb{Q}_p)$ as before. Let $V$ be the set of equivalence classes of vertices of the Bruhat-Tits building $\mathcal{B}(\mathcal{G}, \mathbb{Q}_p)$ up to $G$-action. For $y \in V$, let $\sigma_y:=\cInd_{G_y^+}^{G_y}\Lambda$. Let $\Pi:=\bigoplus_{y \in V}\Pi_y$ where $\Pi_y:=\cInd_{G_y^+}^G\Lambda$. Then $\Pi$ is a projective generator of the category of depth-zero representations $\Rep_{\Lambda}(G)_0$, see \cite[Appendix]{dat2009finitude}. Let $\sigma_{x,1}:=(\sigma_x)|_{\mathcal{B}_{x,1}} \in \mathcal{B}_{x,1} \xhookrightarrow{summand} \Rep_{\Lambda}(G_x)$ be the $\mathcal{B}_{x,1}$-summand of $\sigma_x$. And let $\Pi_{x,1}:=\cInd_{G_x}^G\sigma_{x,1}$. Note $\Pi_{x,1}$ is a summand of $\Pi_x=\cInd_{G_x}^G\sigma_x$, hence a summand of $\Pi$. Using Theorem \ref{Thm Hom}, one can show that the rest of the summands of $\Pi$ don't interfere with $\Pi_{x,1}$ (See Lemma \ref{Lem Ortho} and Lemma \ref{Lem Gen} for precise meaning), hence $\Pi_{x,1}$ is a projective generator of $\mathcal{C}_{x,1}$. Let us state it as a Theorem, see Section 2 for details.
	
	\begin{theorem}\label{Thm Proj}
		$\Pi_{x,1}=\cInd_{G_x}^G\sigma_{x,1}$ is a projective generator of $\mathcal{C}_{x,1}$.
	\end{theorem}
	
	Now we've found a projective generator $\Pi_{x,1}=\cInd_{G_x}^G\sigma_{x,1}$ of $\mathcal{C}_{x,1}$, and it is clear that $\Pi_{x,1}$ is in the essential image of $\cInd_{G_x}^G$. We now deduce from this that $\cInd_{G_x}^G: \mathcal{B}_{x,1} \to \mathcal{C}_{x,1}$ is essentially surjective. Indeed, for any $\pi' \in \mathcal{C}_{x,1}$, we can resolve $\pi'$ by some copies of $\Pi_{x,1}$:
	$$\Pi_{x,1}^{\oplus I} \xrightarrow{f} \Pi_{x,1}^{\oplus J} \to \pi' \to 0.$$
	Using Theorem \ref{Thm Hom} and $\cInd_{G_x}^G$ commutes with arbitrary direct sums (See Lemma \ref{Lem Sum}) we see that $f \in \Hom_G(\Pi_{x,1}^{\oplus I}, \Pi_{x,1}^{\oplus J})$ comes from a morphism $g \in \Hom_{G_x}(\sigma_{x,1}^{\oplus I}, \sigma_{x,1}^{\oplus J})$. Using $\cInd_{G_x}^G$ is exact we see that $\pi'$ is the image of $coker(g) \in \mathcal{B}_{x,1}$ under $\cInd_{G_x}^G$. Therefore, $\cInd_{G_x}^G: \mathcal{B}_{x,1} \to \mathcal{C}_{x,1}$ is essentially surjective.
	
	\subsection{Lemmas}
	
	In this subsection I collect some Lemmas used in the proof of the Main Theorem.
	
	\begin{lemma}\label{Lem Thm Hom implies fully faithful}
		$\cInd_{G_x}^G|_{\mathcal{B}_{x,1}}: \mathcal{B}_{x,1} \to \Rep_{\Lambda}(G)$ is fully faithful.
	\end{lemma}
	
	\begin{proof}
		Let $\rho_1, \rho_2 \in \mathcal{B}_{x,1}$. By the regular cuspidal assumption and Theorem \ref{Thm Cusp Red}, $\rho_1, \rho_2$ has cuspidal reduction. Hence the assumption of Theorem \ref{Thm Hom} is satisfied and we compute using the first case of Theorem \ref{Thm Hom} that
		$$\Hom_G(\cInd_{G_x}^G\rho_1, \cInd_{G_x}^G\rho_2) \simeq \Hom_{G_x}(\rho_1, \rho_2).$$
		In other words, $\cInd_{G_x}^G|_{\mathcal{B}_{x,1}}: \mathcal{B}_{x,1} \to \Rep_{\Lambda}(G)$ is fully faithful.
	\end{proof}
	
	\begin{lemma}\label{Lem Indec}
		The image of $\mathcal{B}_{x,1}$ under $\cInd_{G_x}^G$ is indecomposable as an abelian category.
	\end{lemma}
	
	\begin{proof}
		The point is that $\cInd_{G_x}^G|_{\mathcal{B}_{x,1}}: \mathcal{B}_{x,1} \to \Rep_{\Lambda}(G)$ is not only fully faithful, i.e., an equivalence of categories onto the essential image, but also an equivalence of \textbf{abelian} categories onto the essential image. Indeed, it suffices to show that $\cInd_{G_x}^G|_{\mathcal{B}_{x,1}}: \mathcal{B}_{x,1} \to \Rep_{\Lambda}(G)$ preserves kernels, cokernels, and finite (bi-)products. But this follows from the next Lemma \ref{Lem Sum}.
		
		Assume otherwise that the essential image of $\mathcal{B}_{x,1}$ under $\cInd_{G_x}^G$ is decomposable, then so is $\mathcal{B}_{x,1}$. But $\mathcal{B}_{x,1}$ is a block, hence indecomposable, contradiction!
	\end{proof}
	
	\begin{lemma}\label{Lem Sum}
		$\cInd_{G_x}^G$ is exact and commutes with arbitrary direct sums.
	\end{lemma}
	
	\begin{proof}
		For $\cInd_{G_x}^G$ is exact, we refer to \cite[I.5.10]{vigneras1996representations}.
		
		We show that $\cInd_{G_x}^G$ commutes with arbitrary direct sums. Indeed, $\cInd_{G_x}^G$ is a left adjoint (See \cite[I.5.7]{vigneras1996representations}), hence commutes with arbitrary colimits. In particular, it commutes with arbitrary direct sums.
	\end{proof}
	
	
	
	
	\section{Proof of Theorem \ref{Thm Cusp Red}}\label{Sec Reg Cusp}
	
	The goal of this section is to define regular blocks and regular cuspidal representations with $\Lambda=\overline{\mathbb{Z}_{\ell}}$-coefficients of a finite group of Lie type, and to show that a regular block consists purely of cuspidal representations.
	
	Let $\Lambda:=\overline{\mathbb{Z}_{\ell}}$ be the coefficients of representations. Fix a prime number $p$. Let $\ell$ be a prime number different from $p$. For simplicity, let $q=p$.
	
	\begin{definition}[{\cite[I.4.1]{vigneras1996representations}}]
		\begin{enumerate}Let $\Lambda'$ be any ring.
			\item Let $G$ be a profinite group, a \textbf{representation of $G$ with $\Lambda'$-coefficients} $(\pi, V)$ is a $\Lambda'$-module $V$, together with a $G$-action $\pi: G \to GL_{\Lambda'}(V)$.
			\item A representation of $G$ with $\Lambda'$-coefficients is called \textbf{smooth} if for any $v \in V$, the stablizer $Stab_G(v) \subset G$ is open.
		\end{enumerate}
	\end{definition}
	
	Throughout the article, all representations are assumed to be smooth. The category of smooth representations of $G$ with $\Lambda'$-coefficients is denoted by $\Rep_{\Lambda'}(G)$.
	
	\subsection{Regular blocks and regular cuspidal representations of a finite group of Lie type}
	
	\red{The following notations are used in this subsection only.} Let $\mathcal{G}$ be a split reductive group scheme over $\mathbb{Z}$. Let $\mathbb{G}:=\mathcal{G}(\overline{\mathbb{F}_p})$, $G:=\mathbb{G}^F=\mathcal{G}(\mathbb{F}_p)$, where $F$ is the Frobenius. By abuse of notation, I sometimes identify the group scheme $\mathcal{G}_{\overline{\mathbb{F}_p}}$ with its $\overline{\mathbb{F}_p}$-points $\mathbb{G}$. Let $\mathbb{G}^*$ be the dual group (over $\overline{\mathbb{F}_p}$) of $\mathbb{G}$, and $F^*$ the dual Frobenius (See \cite[Section 4.2]{carter1985finite}). Fix an isomorphism $\overline{\mathbb{Q}_{\ell}} \simeq \mathbb{C}$. 
	
	The definition of regular blocks and regular cuspidal representations of a finite group of Lie type $\Gamma$ involves modular Deligne-Lusztig theory and block theory. We refer to \cite{deligne1976representations}, \cite{carter1985finite}, and \cite{digne2020representations} for Deligne-Lusztig theory, \cite{michel1989bloc} and \cite{broue1990isometries} for modular Deligne-Lusztig theory, and \cite[Appendix B]{bonnafe2010representations} for generalities on blocks. 
	
	First, let's recall a result in Deligne-Lusztig theory (See \cite[Proposition 11.1.5]{digne2020representations}). 
	
	\begin{proposition}\label{Prop dual torus}
		The set of $\mathbb{G}^F$-conjugacy classes of pairs $(\mathbb{T}, \theta)$, where  $\mathbb{T}$ is a $F$-stable maximal torus of  $\mathbb{G}$ and $\theta \in \widehat{\mathbb{T}^F}$, is in non-canonical bijection to the set of $\mathbb{G^*}^{F^*}$-conjugacy classes of pairs $(\mathbb{T}^*, s)$, where $s$ is a semisimple element of $\mathbb{G}^*$ and $\mathbb{T}^*$ is a $F^*$-stable maximal torus of $\mathbb{G}^*$ such that $s \in {\mathbb{T}^*}^{F^*}$.  \blue{Moreover, we could and will fix a compatible system of isomorphisms $\mathbb{F}_{q^n}^* \simeq \mathbb{Z}/(q^n-1)\mathbb{Z}$ to pin down this bijection}.
	\end{proposition}
	
	Now let $s$ be a \textbf{strongly regular semisimple} 
	%(\red{Is this the standard terminology?}) 
	element of $G^*={\mathbb{G}^*}^{F^*}$ (note we require $s$ to be fixed by $F^*$ here), i.e., the centralizer $C_{\mathbb{G}^*}(s)$ is a $F^*$-stable maximal torus, denoted $\mathbb{T}^*$. Let $\mathbb{T}$ be the dual torus of $\mathbb{T}^*$. Let $T=\mathbb{T}^F$ and $T^*={\mathbb{T}^*}^{F^*}$. Let $T_\ell$ denote the $\ell$-part of $T$.
	
	Recall for $s$ strongly regular semisimple, the (rational) Lusztig series $\mathcal{E}(G, (s))$ consists of only one element, namely, $R_T^G(\hat{s})$, where $\hat{s}=\theta$ is such that $(\mathbb{T}, \theta)$ corresponds to $(\mathbb{T}^*, s)$ via the previous bijection \blue{in Proposition \ref{Prop dual torus}}. (This follows from, for example, Broué's equivalence. See Theorem \ref{Thm Broue} below.
	% Better explanation?
	)
	
	\textbf{From now on, assume moreover that $s \in {\mathbb{G}^*}^{F^*}$ has order prime to $\ell$.} In other words, assume $s \in G^*={\mathbb{G}^*}^{F^*}$ is a \textbf{strongly regular semisimple $\ell'$-element}. We are going to define regular blocks, we refer to \cite[Appendix B]{bonnafe2010representations} for generalities on blocks.
	
	Define the \textbf{$\ell$-Lusztig series} 
	$$\mathcal{E}_\ell(G, (s)):=\{R_T^G(\hat{s}\eta)| \eta \in \widehat{T_\ell}\}.$$ Note the notation $\mathcal{E}_\ell(T, (s))$ also makes sense by putting $G=T$.
	
	By \cite{michel1989bloc}, $\mathcal{E}_\ell(G, (s))$ is a union of $\ell$-blocks of $\Rep_{\overline{\mathbb{Q}_\ell}}(G)$. Such a block (or more precisely, a union of blocks) is called a \textbf{($\ell$-)regular block}. Let $e_s^G \in \overline{\mathbb{Z}_\ell}G$ denote the corresponding central idempotent. Note $e_s^T$ also makes sense by putting $G=T$. We shall see later that a regular block is indeed a block, i.e., indecomposible. (This follows from, for example, Broué's equivalence. See Theorem \ref{Thm Broue} below.)
	
	\begin{definition}[Regular blocks]\label{Def Regular Block}
		Let $s \in G^*={\mathbb{G}^*}^{F^*}$ be a strongly regular semisimple $\ell'$-element.
		We call the block $\overline{\mathbb{Z}_\ell}Ge_s^G$ of the group algebra $\overline{\mathbb{Z}_\ell}G$ corresponding to the central idempotent $e_s^G$ the \textbf{regular $\overline{\mathbb{Z}_\ell}$-block} associated to $s$. Let $\mathcal{A}_s:=\overline{\mathbb{Z}_\ell}Ge_s^G\Modl$ be the corresponding category of modules, this is also referred to as a regular block, by abuse of notation.
		
		Similarly, the block $\overline{\mathbb{F}_\ell}Ge_s^G$ is called a $\overline{\mathbb{F}_{\ell}}$-block. (However, this notion won't be used later.)
	\end{definition}
	
	\begin{remark}
		Above all, "a block" could have three different meanings: $\ell$-block, $\overline{\mathbb{Z}_{\ell}}$-block, and $\overline{\mathbb{F}_{\ell}}$-block. But they are in one-one correspondence to each other, so I often abuse the notation and simply call it "a block".
	\end{remark}
	
	Thanks to \cite{broue1990isometries}, we understand the category $\mathcal{A}_s=\overline{\mathbb{Z}_\ell}Ge_s^G\Modl$ quite well. Roughly speaking, it is equivalent to the category of representations of a torus, via Deligne-Lusztig induction. This is what I'm going to explain now.
	
	Let $\mathbb{B} \subset \mathbb{G}$ be a Borel subgroup containing our torus $\mathbb{T}$, let $\mathbb{U}$ be the unipotent radical of $\mathbb{B}$. Let $X_{\mathbb{U}}$ be the Deligne-Lusztig variety defined by
	$$X_{\mathbb{U}}:=\{g \in \mathbb{G} | g^{-1}F(g) \in \mathbb{U}\}.$$
	
	The main result of \cite{broue1990isometries} is the following: The Deligne-Lusztig induction 
	$$R_T^G=R\Gamma_c(X_{\mathbb{U}}, \overline{\mathbb{Z}_\ell})\otimes_{\overline{\mathbb{Z}_\ell}T}-: \overline{\mathbb{Z}_\ell}T\Modl \to \overline{\mathbb{Z}_\ell}G\Modl$$ induces an equivalence of categories between $\overline{\mathbb{Z}_\ell}Te_s^T\Modl$ and $\overline{\mathbb{Z}_\ell}Ge_s^G\Modl$. In particular, one deduce that the irreducible objects in $\overline{\mathbb{F}_\ell}Ge_s^G\Modl$ lifts to $\overline{\mathbb{Z}_\ell}$. More precisely, let us state it as the following theorem.
	
	\begin{theorem}[Broué's equivalence, {\cite[Theorem 3.3]{broue1990isometries}}]\label{Thm Broue}
		With the previous assumptions and notations, assume $X_{\mathbb{U}}$ is affine of dimension $d$ (which is the case if $q$ is greater than the Coxeter number of $\mathbb{G}$.). The cohomology complex $R\Gamma_c(X_{\mathbb{U}}, \overline{\mathbb{Z}_\ell})=R\Gamma_c(X_{\mathbb{U}}, {\mathbb{Z}_\ell}) \otimes_{\mathbb{Z}_\ell}$$\overline{\mathbb{Z}_\ell}$ is concentrated in degree $d=dimX_{\mathbb{U}}$. And the $(\overline{\mathbb{Z}_\ell}Ge_s^G, \overline{\mathbb{Z}_\ell}Te_s^T)$-bimodule $e_s^GH_c^d(X_{\mathbb{U}}, \overline{\mathbb{Z}_\ell})e_s^T$ induces an equivalence of categories
		$$e_s^GH_c^d(X_{\mathbb{U}}, \overline{\mathbb{Z}_\ell})e_s^T \otimes_{\overline{\mathbb{Z}_\ell}Te_s^T}-: \overline{\mathbb{Z}_\ell}Te_s^T\Modl \to \overline{\mathbb{Z}_\ell}Ge_s^G\Modl.$$
	\end{theorem}
	
	\textbf{From now on, we assume the above Theorem holds for all finite groups of Lie type we encountered in this paper.} I hope this is not a severe restriction. This is the case at least when $p$ (or rather, $q$. But I assumed $p=q$ for simplicity in this paper.) is greater than the Coxeter number of $\mathbb{G}$.
	
	We now define regular cuspidal representations as those representations that occur in some regular block. The term "cuspidal" in the name "regular cuspidal" shall be justified later by Theorem \ref{Pure Cuspidality}.
	
	\begin{definition}\label{Def regular cuspidal}
		Let $G$ be a finite group of Lie type. Let $\Lambda=\overline{\mathbb{Z}_{\ell}}$. Let $\rho \in \Rep_{\Lambda}(G)$. Then $\rho$ is called \textbf{regular cuspidal} if each of its irreducible subquotient $\rho_i$ \blue{are cuspidal (See Definition \ref{Def Cuspidal})} and lies in a regular $\overline{\mathbb{Z}_{\ell}}$-block $\mathcal{A}_{s_i}$.
	\end{definition}
	
	
	
	
	\subsection{Pure Cuspidality}
	
	\subsubsection{A digression on cuspidality}
	
	Before stating the theorem of pure cuspidality, let's define cuspidality for representations with arbitrary coefficients. Let $\Lambda'$ be any ring. For example, $\Lambda'$ can be $\overline{\mathbb{Q}_{\ell}}$, $\overline{\mathbb{Z}_{\ell}}$, or $\overline{\mathbb{F}_{\ell}}$.
	
	First, we define two functors.
	
	\begin{definition}[Parabolic induction and restriction] 
		Let $G$ be a finite group of Lie type. Let $P$ be a parabolic subgroup and $M$ the corresponding Levi subgroup.
		\begin{enumerate}
			\item The \textbf{parabolic induction functor} is defined to be the composition 
			$$i_M^G:= \Ind_P^G \circ f^*,$$ where 
			$$f^*: \Rep_{\Lambda'}(M) \to \Rep_{\Lambda'}(P)$$
			is the inflation along the natural projection $f: P \to M$. 
			\item The \textbf{parabolic restriction functor} is defined to be the composition 
			$$r_M^G:= (-)_U \circ \Res_P^G,$$ where 
			$$(-)_U: \Rep_{\Lambda'}(P) \to \Rep_{\Lambda'}(M), V \mapsto V/<\{u.v-v | u \in U, v \in V\}>_{\Lambda'U\Modl}$$
			is the functor of taking coinvariance.
		\end{enumerate}
	\end{definition}
	
	We recall that $r_M^G$ is left adjoint to $i_M^G$ and they are both exact under our assumption $\ell \neq p$ (See \cite[II.2.1]{vigneras1996representations}).
	
	\begin{definition}[Cuspidal]\label{Def Cuspidal}
		Let $G$ be a finite group of Lie type. Let $\rho \in \Rep_{\Lambda'}(G)$ be a representation of $G$. Then $\rho$ is called \textbf{($\Lambda'$-)cuspidal} if $\rho$ is not a subrepresentation of any proper parabolic induction, i.e., 
		$$\Hom_{G}(\rho, i_P^G(\sigma))=0$$ 
		for any proper parabolic subgroup $P$ of $G$ and any representation $\sigma \in \Rep_{\Lambda'}(M)$, where $M$ is the Levi subgroup corresponding to $P$.
	\end{definition}
	
	For example, let $s \in G^*$ strongly regular semisimple, then 
	$$R_T^G(\hat{s})=R\Gamma_c(X_{\mathbb{U}}, \overline{\mathbb{Q}_\ell})\otimes \hat{s}$$ 
	is cuspidal in $\Rep_{\overline{\mathbb{Q}_{\ell}}}(G)$ (See \cite[Theorem 8.3]{deligne1976representations}). 
%	Moreover, reduction modulo $\ell$ preserves cuspidality (\red{See ?}), hence each irreducible component of the reduction $$r_{\ell}(R_T^G(\hat{s})):=R\Gamma_c(X_{\mathbb{U}}, \overline{\mathbb{F}_\ell})\otimes \hat{s}$$
%	is cuspidal in $\Rep_{\overline{\mathbb{F}_{\ell}}}(G)$. 
	
	I record the following equivalent definition of cuspidality for later use.
	
	\begin{lemma}\cite[II.2.3]{vigneras1996representations}\label{Lemma Cuspidal}
		$\rho \in \Rep_{\Lambda'}(G)$ is cuspidal if and only if $r_M^G\rho=0$, for any proper Levi subgroup $M$ of $G$.
	\end{lemma}
	
%	Can cuspidality be checked on irreducible subquotients? Let's define subquotient first.
%	
%	\begin{definition}[subquotient]
%		Let $\pi$ be a representation of $G$, a \textbf{subquotient} of $\pi$ is a representation of the form $\pi_1/\pi_2$ for some chain of subrepresentations $\pi_2 \subset \pi_1 \subset \pi$.
%	\end{definition}
	
%	\begin{question}
%		\red{Let $(\pi, V) \in \Rep_{\Lambda}(G)$. If all irreducible subquotients of $\pi$ are cuspidal, is $\pi$ cuspidal?}
%	\end{question}
	

	
	
	\subsubsection{The theorem of pure cuspidality}
	
	We can now state the theorem of pure cuspidality. 
	
	As in Broué's paper \cite{broue1990isometries}, we fix a finite integral extension $\mathcal{O}$ of $\mathbb{Z}_{\ell}$, which is big enough. One good thing to work with $\mathcal{O}$ instead of $\overline{\mathbb{Z}_{\ell}}$ is that $\mathcal{O}$ is a discrete valuation ring, while $\overline{\mathbb{Z}_{\ell}}$ is not (even not Noetherian). We assume $\mathcal{O}$ to be big enough (for example, $\mathcal{O}$ contains all roots of unity we encounter) so that all things we need to do representation theory are available without change.
	
	\begin{theorem}[Pure Cuspidality]\label{Pure Cuspidality}
		Let $G$ be a finite group of Lie type. Let $s \in G^*=\mathbb{G^*}^{F^*}$ be a strongly regular semisimple $\ell'$-element, \blue{with corresponding torus $T=\mathbb{T}^F$ and character $\hat{s} \in \hat{T}$ as in Proposition \ref{Prop dual torus}}. \blue{Assume that $R_T^G(\hat{s})$ is $\overline{\mathbb{Q}_{\ell}}$-cuspildal.} Then the $\overline{\mathbb{Z}_{\ell}}$-block $\mathcal{A}_s=\overline{\mathbb{Z}_{\ell}}Ge_s^G\Modl$ consists purely of cuspidal representations.
	\end{theorem}
	
	\begin{proof}
		%		\red{Minor technical issue: Broué's paper works with a Dedekind ring $\mathcal{O}$, but $\overline{\mathbb{Z}_{\ell}}$ is not Dedekind. So you need to be careful about the results cited from Broué's paper, for example, Lemma 3.4 of Broué's paper.}
		%    	Let $V:=\overline{\mathbb{Z}_{\ell}}Ge_s^G \in \overline{\mathbb{Z}_{\ell}}G\Modl=\Rep_{\overline{\mathbb{Z}_{\ell}}}(G)$. Let's first show that $V$ is $\overline{\mathbb{Z}_{\ell}}$-cuspidal.
		
		Recall Broué's equivalence: For $\mathcal{O}$ a finite integral extension of $\mathbb{Z}_{\ell}$, big enough, we have
		$$F:=e_s^GH^d_c(X_{\mathbb{U}}, \mathcal{O})e_s^T\otimes_{\mathcal{O}Te_s^T}-: \mathcal{O}Te_s^T\Modl \to \mathcal{O}Ge_s^G\Modl$$ is an equivalence of categories. This is moreover an equivalence of abelian categories (See Lemma \ref{Lem abelian}). Let $V:=F(\mathcal{O}Te_s^T)=e_s^GH^d_c(X_{\mathbb{U}}, \mathcal{O})e_s^T$ %(\red{$=\mathcal{O}Ge_s^G$?} \blue{No, not true. Otherwise this is contained in the Harish-Chandler induction.})
		. Then $V$ is a projective generator of $\mathcal{A}_s$, since $\mathcal{O}Te_s^T$ is a projective generator of $\mathcal{O}Te_s^T\Modl$. We first show that $V$ is $\mathcal{O}$-cuspidal.
		
		By classical Deligne-Lusztig theory, $\overline{\mathbb{Q}_{\ell}}V \simeq \bigoplus_{\eta \in \hat{T_{\ell}}}R_T^G(\hat{s}\eta)$ 
		%(\red{Is this right?} \blue{Yes.}) 
		is $\overline{\mathbb{Q}_{\ell}}$-cuspidal \blue{(For details, see Lemma below.)}. 
		%    In other words, 
		%    $$dim_{\overline{\mathbb{Q}_{\ell}}}\overline{\mathbb{Q}_{\ell}}V=dim_{\overline{\mathbb{Q}_{\ell}}}(\overline{\mathbb{Q}_{\ell}}V)(U)=dim_{\overline{\mathbb{Q}_{\ell}}}\overline{\mathbb{Q}_{\ell}}V(U).$$ 
		%    Since $V$ is free $\overline{\mathbb{Z}_{\ell}}$-module, we thus have
		%    $$rank_{\overline{\mathbb{Z}_{\ell}}}V=rank_{\overline{\mathbb{Z}_{\ell}}}V(U).$$
		In other words, 
		$$r^G_{M, \overline{\mathbb{Q}_{\ell}}}(\overline{\mathbb{Q}_{\ell}}V):=\overline{\mathbb{Q}_{\ell}}V/<\{u.v-v | u \in U, v \in \overline{\mathbb{Q}_{\ell}}V\}>_{\overline{\mathbb{Q}_{\ell}}U\Modl}=0.$$
		However, note  
		$$<\{u.v-v | u \in U, v \in \overline{\mathbb{Q}_{\ell}}V\}>_{\overline{\mathbb{Q}_{\ell}}U\Modl}=<\{u.v-v | u \in U, v \in \overline{\mathbb{Q}_{\ell}}V\}>_{\mathcal{O}U\Modl}.$$
		So we have 
		$$r^G_{M, \mathcal{O}}(\overline{\mathbb{Q}_{\ell}}V):=\overline{\mathbb{Q}_{\ell}}V/<\{u.v-v | u \in U, v \in \overline{\mathbb{Q}_{\ell}}V\}>_{\mathcal{O}U\Modl}=0.$$
		
		Note $V$ is finitely presented and projective over $\mathcal{O}Te_s^T$ (See \cite[Proof of Theorem 3.3]{broue1990isometries}), hence projective over $\mathcal{O}$ (because the restriction functor $\mathcal{O}T\Modl \to \mathcal{O}\Modl$ preserves projectivity, since it's left adjoint to an exact functor, the induction functor), which is a local ring 
		%(\blue{See Lemma 3 from last manuscript "Week 24-25"})
		, hence $V$ is free over $\mathcal{O}$ (See \cite[Theorem 24.4.5]{vakil2017rising}). We thus have an inclusion
		$$V \xhookrightarrow[]{} \overline{\mathbb{Q}_{\ell}}V:=\overline{\mathbb{Q}_{\ell}}\otimes_{\mathcal{O}}V %\text{(\red{Is this true?} \blue{Yes.})}
		$$
		as $\mathcal{O}G$-modules.
		Recall that the parabolic restriction $r^G_{M, \mathcal{O}}$ is exact (See \cite[II.2.1]{vigneras1996representations}), hence 
		$$r^G_{M, \mathcal{O}}(\overline{\mathbb{Q}_{\ell}}V)=0$$
		implies that 
		$$r^G_{M, \mathcal{O}}(V)=0,$$
		i.e., $V$ is $\mathcal{O}$-cuspidal. 
		%(\red{Did we use $U$ pro-$p$ somewhere?} \blue{No. But one can also argue using invariance instead of coinvariance, and for "invariance = coinvariance" we need $U$ to be a $p$-group.})
		
		Moreover, base change to $\overline{\mathbb{Z}_{\ell}}$ we see that $\overline{\mathbb{Z}_{\ell}}V$ is $\overline{\mathbb{Z}_{\ell}}$-cuspidal. Indeed, 
		$$r^G_{M, \overline{\mathbb{Z}_{\ell}}}(\overline{\mathbb{Z}_{\ell}}V)=\overline{\mathbb{Z}_{\ell}}V/\overline{\mathbb{Z}_{\ell}}V(U)=\overline{\mathbb{Z}_{\ell}}\otimes_{\mathcal{O}}(V/V(U))=\overline{\mathbb{Z}_{\ell}}\otimes_{\mathcal{O}}r^G_{M, \mathcal{O}}(V)=0.$$
		
		For general $V' \in \mathcal{A}_s$, we can resolve it by some direct sum of $V$'s, and we see that
		$$r^G_{M, \overline{\mathbb{Z}_{\ell}}}(V')=0,$$
		(using $r^G_{M, \overline{\mathbb{Z}_{\ell}}}$ is exact and commutes with arbitrary direct sum) i.e., $V'$ is $\overline{\mathbb{Z}_{\ell}}$-cuspidal.
	\end{proof}
	

	

	
	\begin{lemma}\label{Lem abelian}
		$$F:=e_s^GH^d_c(X_{\mathbb{U}}, \mathcal{O})e_s^T\otimes_{\mathcal{O}Te_s^T}-: \mathcal{O}Te_s^T\Modl \to \mathcal{O}Ge_s^G\Modl$$ is an equivalence of abelian categories.
	\end{lemma}
	
	\begin{proof}
		We already know that $F$ is an equivalence of categories. It remains to show that $F$ is exact and commutes with product.
		
		Now $e_s^GH^d_c(X_{\mathbb{U}}, \mathcal{O})e_s^T$ is projective over ${\mathcal{O}Te_s^T}$ (See \cite[Proof of Theorem 3.3]{broue1990isometries}), hence flat over ${\mathcal{O}Te_s^T}$. Hence $F:=e_s^GH^d_c(X_{\mathbb{U}}, \mathcal{O})e_s^T\otimes_{\mathcal{O}Te_s^T}-$ is exact.
		
		It is clear that $F:=e_s^GH^d_c(X_{\mathbb{U}}, \mathcal{O})e_s^T\otimes_{\mathcal{O}Te_s^T}-$ commutes with product.
	\end{proof}

    \blue{
    \begin{lemma}\label{Lem Q_l-bar cuspidal}
    	Let $G$ be a finite group of Lie type. Let $s \in G^*=\mathbb{G^*}^{F^*}$ be a strongly regular semisimple $\ell'$-element, with corresponding torus $T=\mathbb{T}^F$ and character $\hat{s} \in \hat{T}$ as before. Assume that $R_T^G(\hat{s})$ is $\overline{\mathbb{Q}_{\ell}}$-cuspildal. Then $R_T^G(\hat{s}\eta)$ is $\overline{\mathbb{Q}_{\ell}}$-cuspidal for any $\eta \in \hat{T_{\ell}}$.
    \end{lemma}
    }
	
	

	
	
	
	\subsection{Proof of Theorem \ref{Thm Cusp Red}}
	
	We now apply the previous results on finite group of Lie types to representations of the parahoric subgroups of a $p$-adic group. The notation, therefore, is different from before.
	
	Let $\mathcal{G}$ be a split, simply connected reductive group scheme over $\mathbb{Z}$. Let $G:=\mathcal{G}(\mathbb{Q}_p)$. For simplicity, I assume $p=q$ is greater than the Coxeter number of $G$ (See Theorem \ref{Thm Broue} for reason).
	
	Let $x$ be a vertex of the Bruhat-Tits building $\mathcal{B}(\mathcal{G}, \mathbb{Q}_p)$, $G_x$ the parahoric subgroup associated to $x$, $G_x^+$ its pro-unipotent radical. Recall that $\overline{G_x}:=G_x/G_x^+$ is a generalized Levi subgroup of $\mathcal{G}(\mathbb{F}_p)$ (in particular, a finite group of Lie type) with root system $\Phi_x$, see \cite[Theorem 3.17]{rabinoff2003bruhat}. 
	
	Let $\Lambda=\overline{\mathbb{Z}_\ell}$, with $\ell \neq p$. Let $\rho \in \Rep_{\Lambda}(G_x)$ be an irreducible representation of $G_x$, which is trivial on $G_x^+$ and whose reduction to the finite group of Lie type $\overline{G_x}=G_x/G_x^+$ is regular cuspidal. 
	%We make this a definition for later use.
	
%	\begin{definition}
%		Let $\rho \in \Rep_{\Lambda}(G_x)$. We say $\rho$ \textbf{has cuspidal reduction} (resp. \textbf{has regualr cuspidal reduction}), if $\rho$ is trivial on $G_x^+$ and whose reduction to the finite group of Lie type $\overline{G_x}=G_x/G_x^+$ is cuspidal (resp. regular cuspidal). Let's denote the reduction of $\rho$ modulo $G_x^+$ by $\overline{\rho} \in \Rep_{\Lambda}(\overline{G_x})$.
%	\end{definition}
	
	In other words, we start with an irreducible representation $\rho \in \Rep_{\Lambda}(G_x)$ which has regular cuspidal reduction. Let $\mathcal{B}_{x,1}$ be the ($\overline{\mathbb{Z}_{\ell}}$-)block of $\Rep_{\Lambda}(G_x)$ containing $\rho$. We can now prove Theorem \ref{Thm Cusp Red}, which we restate as follows.
	
	\begin{theorem}
		Let $\rho \in \Rep_{\Lambda}(G_x)$ be an irreducible representation of $G_x$, which has regular cuspidal reduction. Let $\mathcal{B}_{x,1}$ be the $\overline{\mathbb{Z}_{\ell}}$-block of $\Rep_{\Lambda}(G_x)$ containing $\rho$. Then any $\rho' \in \mathcal{B}_{x,1}$ has cuspidal reduction.
	\end{theorem}
	
	\begin{proof}
		Let $\overline{\rho} \in \Rep_{\Lambda}(\overline{G_x})$ be the reduction of $\rho$ modulo $G_x^+$. $\overline{\rho}$ is irreducible (since $\rho$ is) and regular cuspidal by assumption, so it is of the form $r_{\ell}(R_T^G(\hat{s}))$ (i.e., the $\ell$-reduction of $R_T^G(\hat{s})$), for some strongly regular semisimple $\ell'$-element $s$ of $\overline{G_x}^*$ (See Definition \ref{Def regular cuspidal}.). 
		
		Let $\Rep_{\Lambda}(G_x)_0$ be the full subcategory of $\Rep_{\Lambda}(G_x)$ consists of representations of $G_x$ that are trivial on $G_x^+$. The key observation is that $\Rep_{\Lambda}(G_x)_0$ is a summand (as abelian category) of $\Rep_{\Lambda}(G_x)$ (See Lemma \ref{Lem Summand}).
		
		Then since $\rho \in \Rep_{\Lambda}(G_x)_0$, its block $\mathcal{B}_{x,1}$ is a summand of $\Rep_{\Lambda}(G_x)_0$.
		
		On the other hand, notice the inflation induces an equivalence of categories between $\Rep_{\Lambda}(\overline{G_x})$ and $\Rep_{\Lambda}(G_x)_0$, with inverse the reduction modulo $G_x^+$.
		
		So the blocks of $\Rep_{\Lambda}(\overline{G_x})$ and $\Rep_{\Lambda}(G_x)_0$ should agree. Let $\mathcal{A}_{x,1}$ be the corresponding block of $\Rep_{\Lambda}(\overline{G_x})$ to $\mathcal{B}_{x,1}$. Then $\mathcal{A}_{x,1}$ is contained in the regular block $\mathcal{A}_s$ corresponding to $s$ (recall $\overline{\rho}=r_{\ell}(R_T^G(\hat{s}))$). By Theorem \ref{Pure Cuspidality}, $\mathcal{A}_s$ consists purely of cuspidal representation. Therefore, $\mathcal{B}_{x,1}$ consists purely of representations that have cuspidal reductions. 
	\end{proof}
	
	\begin{lemma}\label{Lem Summand}
		Let $\Rep_{\Lambda}(G_x)_0$ be the full subcategory of $\Rep_{\Lambda}(G_x)$ consists of representations of $G_x$ that are trivial on $G_x^+$. Then $\Rep_{\Lambda}(G_x)_0$ is a summand as abelian category of $\Rep_{\Lambda}(G_x)$.
	\end{lemma}
	
	\begin{proof}
		Note $G_x^+$ is pro-$p$ (See \cite[II.5.2.(b)]{vigneras1996representations}), in particular, it has pro-order invertible in $\Lambda$. So we have a normalized Haar measure $\mu$ on $G_x$ such that $\mu(G_x^+)=1$ (See \cite[I.2.4]{vigneras1996representations}). The characteristic function $e:=1_{G_x^+}$ is an idempotent of the Hecke algebra $\mathcal{H}_{\Lambda}(G_x)$ under convolution with respect to the Haar measure $\mu$. We shall show that $e=1_{G_x^+}$ cuts out $\Rep_{\Lambda}(G_x)_0$ as a summand of $\Rep_{\Lambda}(G_x) \simeq \mathcal{H}_{\Lambda}(G_x)\Modl$.
		
		Let's first check that $e=1_{G_x^+}$ is central. This can be done by an explicit computation. Recall that we have a descending filtration $\{G_{x,r} | r\in \mathbb{R}_{>0}\}$ of $G_x$ such that 
		\begin{enumerate}
			\item $\forall r \in \mathbb{R}_{>0}, G_{x,r}$ is an open compact pro-$p$ subgroup of $G_x$.
			\item $\forall r \in \mathbb{R}_{>0}, G_{x,r}$ is a normal subgroup of $G_x$.
			\item $G_{x,r}$ form a neighborhood basis of $1$ inside $G_x$. 
		\end{enumerate}
		(See \cite[II.5.1]{vigneras1996representations}.) Therefore, to check $e \ast f=f \ast e$, for all $f \in \mathcal{H}_{\Lambda}(G_x)$, it suffices to check for all $f$ of the form $1_{gG_{x,r}}$, the characteristic function of the (both left and right) coset $gG_{x,r}$($=G_{x,r}g$, by normality) for some $g \in G$ and $r \in \mathbb{R}_{>0}$. Indeed, one can compute that $(e \ast 1_{gG_{x,r}})(y)=\mu(G_x^+\cap G_{x,r}yg^{-1})$ and that $(1_{gG_{x,r}} \ast e)(y)=\mu(gG_{x,r}\cap yG_x^+)$, for any $y \in G_x$. Note that $G_{x,r} \subset G_x^+$, we get that $\mu(G_x^+\cap G_{x,r}yg^{-1})=\mu(G_{x,r})$ if $yg^{-1} \in G_x^+$ and $0$ otherwise. Same for $\mu(gG_{x,r}\cap yG_x^+)$. Therefore, $e$ is central.
		%(\red{See ? for details.})
		
		
		Finally, under the isomorphism $\Rep_{\Lambda}(G_x) \simeq \mathcal{H}_{\Lambda}(G_x)\Modl$, $\Rep_{\Lambda}(G_x)_0$ corresponds to the summand $\mathcal{H}_{\Lambda}(G_x, G_x^+)\Modl=e\mathcal{H}_{\Lambda}(G_x)e\Modl$ corresponding to the central idempotent $e:=1_{G_x^+} \in \mathcal{H}_{\Lambda}(G_x)$ of $\mathcal{H}_{\Lambda}(G_x)\Modl$, hence $\Rep_{\Lambda}(G_x)_0$ is a summand of $\Rep_{\Lambda}(G_x)$. 
	\end{proof}
	
	




	
	
	
	
	\section{Proof of Theorem \ref{Thm Hom}}\label{Sec Pf Thm Hom}
	
	Let's now prove Theorem \ref{Thm Hom}.
	
	\begin{proof}[Proof of Theorem \ref{Thm Hom}]
		\begin{equation*}
			\begin{aligned}
				&\Hom_G(\cInd_{G_x}^G\rho_1, \cInd_{G_y}^G\rho_2)\\
				=\;&\Hom_{G_x}\left(\rho_1,(\cInd_{G_y}^G\rho_2)|_{G_x}\right)\\
				=\;& \Hom_{G_x}\left(\rho_1, \bigoplus_{g \in {G_y\backslash G/G_x}}\cInd_{G_x \cap g^{-1}G_yg}^{G_x}\rho_2(g-g^{-1})\right)
			\end{aligned}
		\end{equation*}
		
		Recall that $g^{-1}G_yg=G_{g^{-1}.y}$. So it suffices to show that for $g \in G$ with $G_x \cap g^{-1}G_yg \neq G_x$, or equivalently, for $g 
		\in G$ with $g.x \neq y$ (since $x$ and $y$ are vertices), it holds that
		$$\Hom_{G_x}\left(\rho_1, \cInd_{G_x \cap g^{-1}G_yg}^{G_x}\rho_2(g-g^{-1})\right)=0.$$
		
		Note $G_x/(G_x \cap g^{-1}G_yg)$ is compact, hence $\cInd_{G_x \cap g^{-1}G_yg}^{G_x}=\operatorname{Ind}_{G_x \cap g^{-1}G_yg}^{G_x}$, and we have Frobenius reciprocity in the other direction
		$$\Hom_{G_x}\left(\rho_1, \cInd_{G_x \cap g^{-1}G_yg}^{G_x}\rho_2(g-g^{-1})\right) \simeq \Hom_{G_x \cap g^{-1}G_yg}\left(\rho_1, \rho_2(g-g^{-1})\right).$$
		
		So it suffices to show that for $g \in G$ with $g.x \neq y$,
		$$\Hom_{G_x \cap g^{-1}G_yg}\left(\rho_1, \rho_2(g-g^{-1})\right)=0.$$
		Note now this expression is symmetric with respect to $\rho_1$ and $\rho_2$, so is the following argument.
		
		First, if $\rho_2$ has cuspidal reduction (denoted $\overline{\rho_2}$),
		\begin{align*}    	
			& \Hom_{G_x \cap g^{-1}G_yg}\left(\rho_1, \rho_2(g-g^{-1})\right) \\
			=\;& \Hom_{G_x \cap G_{g^{-1}.y}}\left(\rho_1, \rho_2(g-g^{-1})\right) \\
			\subseteq\;& \Hom_{G_x^+ \cap G_{g^{-1}.y}}\left(\rho_1, \rho_2(g-g^{-1})\right) && %\text{By \eqref{eq:1}}
			\\
			=\;& \Hom_{G_x^+ \cap G_{g^{-1}.y}}(1^{\oplus d_1}, \rho_2(g-g^{-1})) && \text{$\rho_1$ is trivial on $G_x^+$ }\\
			=\;& \Hom_{G_{g.x}^+ \cap G_y}(1^{\oplus d_1}, \rho_2) && \text{Conjugate by $g^{-1}$}\\
			=\;& \Hom_{U_y(g.x)}(1^{\oplus d_1}, \overline{\rho_2}) && \text{Reduction modulo $G_y^+$. See below.}\\
			=\;& 0 && \text{$\overline{\rho_2}$ is cuspidal. See below.}
		\end{align*}
		
		The last two equations need some explanation. 
		
		The former one uses the following consequence from Bruhat-Tits theory: If $x_1$ and $x_2$ are two different vertices of the Bruhat-Tits building, then $\overline{G_{x_i}}:=G_{x_i}/G_{x_i}^+$ is a generalized Levi subgroup of $\overline{G}=G(\mathbb{F}_p)$, for $i=1, 2$. Moreover, $G_{x_1} \cap G_{x_2}$ projects onto a proper parabolic subgroup $P_{x_1}(x_2)$ of $\overline{G_{x_1}}$ under the reduction map $G_{x_1} \to \overline{G_{x_1}}$. And $G_{x_1} \cap G_{x_2}^+$ projects onto $U_{x_1}(x_2)$, the unipotent radical of $P_{x_1}(x_2)$, under the reduction map $G_{x_1} \to \overline{G_{x_1}}$. For details, see Lemma \ref{Lem Passage to Residue Field} below. Note that the assumption of Lemma \ref{Lem Passage to Residue Field} is satisfied since without loss of generality we may assume $x_1=x$ and $x_2=y$ lies in the closure of a common alcove (since $G$ acts simply transitively on the set of alcoves).
		
		The latter one uses that for a cuspidal representation $\rho$ of a finite group of Lie type $\Gamma$, 
		$$\Hom_U(1, \rho|_U)=\Hom_U(\rho|_U, 1)=0,$$
		for the unipotent radical $U$ of $P$, where $P$ is any proper parabolic subgroup of $\Gamma$. For details, see Lemma \ref{Lem Hom_U(1_U, cusp)} below.
		
		Symmetrically, a similar argument works if $\rho_1$ has cuspidal reduction. Indeed, if $\rho_1$ has cuspidal reduction (denoted $\overline{\rho_1}$),
		\begin{align*}    	
			& \Hom_{G_x \cap g^{-1}G_yg}\left(\rho_1, \rho_2(g-g^{-1})\right) \\
			=\;& \Hom_{gG_xg^{-1} \cap G_y}\left(\rho_1(g^{-1}-g), \rho_2\right) && \text{Conjugate by $g^{-1}$}\\ 
			\subseteq\;& \Hom_{gG_xg^{-1} \cap G_y^+}\left(\rho_1(g^{-1}-g), \rho_2\right) && %\text{By \eqref{eq:1}}
			\\
			=\;& \Hom_{gG_xg^{-1} \cap G_y^+}(\rho_1(g^{-1}-g), 1^{\oplus d_2}) && \text{$\rho_2$ is trivial on $G_y^+$ }\\
			=\;& \Hom_{G_x \cap g^{-1}G_y^+g}(\rho_1, 1^{\oplus d_2}) && \text{Conjugate by $g$}\\
			=\;& \Hom_{G_x \cap G_{g^{-1}.y}^+}(\rho_1, 1^{\oplus d_2}) && \\
			=\;& \Hom_{U_x(g^{-1}.y)}(\overline{\rho_1}, 1^{\oplus d_2}) && \text{Reduction modulo $G_x^+$}\\
			=\;& 0 && \text{$\overline{\rho_1}$ is cuspidal. }
		\end{align*}
		
	\end{proof}
	
	\subsection{Lemmas}
	
	\begin{lemma}\label{Lem Passage to Residue Field}
		Let $x_1$ and $x_2$ be two points of the Bruhat-Tits building $\mathcal{B}(\mathcal{G}, \mathbb{Q}_p)$. Assume they lie in the closure of a same alcove.
		\begin{enumerate}
			\item[(i)]   The image of $G_{x_1} \cap G_{x_2}$ in $\overline{G_{x_1}}$ is a parabolic subgroup of $\overline{G_{x_1}}$. Let's denote it by $P_{x_1}(x_2)$. Moreover, the image of $G_{x_1} \cap G_{x_2}^+$ in $\overline{G_{x_1}}$ is the unipotent radical of $P_{x_1}(x_2)$. Let's denote it by $U_{x_1}(x_2)$.
			\item[(ii)] 	Assume moreover that $x_1$ and $x_2$ are two different vertices of the building. Then $P_{x_1}(x_2)$ is a proper parabolic subgroup of $\overline{G_{x_1}}$.
		\end{enumerate}
	\end{lemma}
	
	\begin{proof}
		(i) is \cite[II.5.1.(k)]{vigneras1996representations}.
		
		Let's prove (ii). It suffices to show that $G_{x_1} \neq G_{x_2}$. Assume otherwise that $G_{x_1}=G_{x_2}$, then $x_1$ and $x_2$ lie in the same facet, which contradicts with the assumption that $x_1$ and $x_2$ are two different vertices.
	\end{proof}
	
	\begin{lemma}\label{Lem Hom_U(1_U, cusp)}
		Let $\overline{\rho}$ be a cuspidal representation of a finite group of Lie type $\Gamma$. Let $P$ be a proper parabolic subgroup of $\Gamma$, with unipotent radical $U$. Then
		$$Hom_U(1_U, \overline{\rho})=Hom_U(\overline{\rho}, 1_U)=0.$$
	\end{lemma}
	
	\begin{proof}
		$\Hom_U(\overline{\rho}|_U, 1_U)=\Hom_{\Gamma}(\overline{\rho}, Ind_P^{\Gamma}(\sigma))=0$, where $\sigma=Ind_U^P(1_U)$. The last equality holds because $\overline{\rho}$ is assumed to be cuspidal (Recall Definition \ref{Def Cuspidal}). Similar for $\Hom_U(1_U, \rho|_U)$.
	\end{proof}
	
	
	
	
	\section{Proof of Theorem \ref{Thm Proj}}
	
	In this subsection, I prove that $\Pi_{x,1}$ is a projective generator of $\mathcal{C}_{x,1}$. Before doing this, let's recall the setting. Fix a vertex $x$ of the building of $G$. Let $\rho \in \Rep_{\Lambda}(G_x)$ which is trivial on $G_x^+$ and whose reduction to $\overline{G_x}=G_x/G_x^+$ is regular cuspidal, $\pi=\cInd_{G_x}^G\rho$ as before. Let $\mathcal{B}_{x,1}$ be the block of $\Rep_{\Lambda}(G_x)$ containing $\rho$, and $\mathcal{C}_{x,1}$ the block of $\Rep_{\Lambda}(G)$ containing $\pi$. 
	
	Let $V$ be the set of equivalence classes of vertices of the Bruhat-Tits building $\mathcal{B}(\mathcal{G}, \mathbb{Q}_p)$ up to $G$-action. For $y \in V$, let $\sigma_y:=\cInd_{G_y^+}^{G_y}\Lambda$. Let $\Pi:=\bigoplus_{y \in V}\Pi_y$ where $\Pi_y:=\cInd_{G_y^+}^G\Lambda$. Then $\Pi$ is a projective generator of the category of depth-zero representations $\Rep_{\Lambda}(G)_0$, see \cite[Appendix]{dat2009finitude}. Let $\sigma_{x,1}:=(\sigma_x)|_{\mathcal{B}_{x,1}} \in \mathcal{B}_{x,1} \xhookrightarrow{summand} \Rep_{\Lambda}(G_x)$ be the $\mathcal{B}_{x,1}$-summand of $\sigma_x$. And let $\Pi_{x,1}:=\cInd_{G_x}^G\sigma_{x,1}$.
	
	Let's summarize the setting in the following diagram.
	
	\begin{tikzcd}
		{\Rep_{\Lambda}(G_x)} & {\Rep_{\Lambda}(G)} \\
		{\Rep_{\Lambda}(G_x)_0} & {\Rep_{\Lambda}(G)_0} \\
		{\mathcal{B}_{x,1}} & {\mathcal{C}_{x,1}} \\
		{\text{block of } \rho} & {\text{block of }\pi}
		\arrow[from=2-1, to=2-2]
		\arrow["{\cInd_{G_x}^{G}}", from=1-1, to=1-2]
		\arrow["\subset"{description}, sloped, draw=none, from=2-1, to=1-1]
		\arrow["\subset"{description}, sloped, draw=none, from=3-1, to=2-1]
		\arrow["\subset"{description}, sloped, draw=none, from=3-2, to=2-2]
		\arrow["\subset"{description}, sloped, draw=none, from=2-2, to=1-2]
		\arrow["{=:}"{description}, sloped, draw=none, from=4-1, to=3-1]
		\arrow["{:=}"{description}, sloped, draw=none, from=3-2, to=4-2]
		\arrow[from=3-1, to=3-2]
	\end{tikzcd}
	
	\begin{theorem}
		$\Pi_{x,1}=\cInd_{G_x}^G\sigma_{x,1}$ is a projective generator of $\mathcal{C}_{x,1}$.
	\end{theorem}
	
	\begin{proof}
		First, let $\Rep_{\Lambda}(G_x)_0$ be the full subcategory of $\Rep_{\Lambda}(G_x)$ consisting of representations that are trivial on $G_x^+$ (Don't confuse with $\Rep_{\Lambda}(G)_0$, the depth-zero category of $G$). Note $\Rep_{\Lambda}(G_x)_0$ is a summand of $\Rep_{\Lambda}(G_x)$ (see Lemma \ref{Lem Summand}).
		
		Second, note that $\Rep_{\Lambda}(G_x)_0 \simeq \Rep_{\Lambda}(\overline{G_x})$. We may assume $$\Rep_{\Lambda}(G_x)_0=\mathcal{B}_{x,1} \oplus ... \oplus \mathcal{B}_{x,m}$$
		is its block decomposition. So that $\sigma_x=\sigma_{x,1}\oplus...\oplus\sigma_{x,m}$ accordingly. Write $\sigma_x^1:=\sigma_{x,2}\oplus...\oplus\sigma_{x,m}$. Then $\sigma_x=\sigma_{x,1} \oplus \sigma_x^1$, and $\Pi_x=\Pi_{x,1} \oplus \Pi_x^1$ accordingly, where $\Pi_x^1:=\cInd_{G_x}^G\sigma_x^1$. And
		$$\Pi=\Pi_{x,1}\oplus \Pi_x^1 \oplus \Pi^x,$$
		where $\Pi^x:=\bigoplus_{y \neq x}\Pi_y$. Let $\Pi^{x,1}:=\Pi_x^1 \oplus \Pi^x$, then we have
		$$\Pi=\Pi_{x,1} \oplus \Pi^{x,1}.$$
		
		Recall that $\Pi$ is a projective generator of the category of depth-zero representations $\Rep_{\Lambda}(G)_0$. This implies that 
		$$\Hom_G(\Pi, -): \Rep_{\Lambda}(G)_0 \to \Modr\End_G(\Pi)$$
		is an equivalence of categories. See \cite[Lemma 22]{bernsteindraft}.
		
		Next, it is not hard to see that Theorem \ref{Thm Hom} implies that 
		$$\Hom_G(\Pi_{x,1}, \Pi^{x,1})=\Hom_G(\Pi^{x,1}, \Pi_{x,1})=0,$$
		see Lemma \ref{Lem Ortho}. This implies that $$\Modr\End_G(\Pi) \simeq \Modr\End_G(\Pi_{x,1}) \oplus \Modr\End_G(\Pi^{x,1})$$ is an equivalence of categories.
		
		Now we can combine the above to show that $\Pi^{x,1}$ does not interfere with $\Pi_{x,1}$, i.e.,
		$$\Hom_G(\Pi^{x,1}, X)=0,$$
		for any object $X \in \mathcal{C}_{x,1}$ (see Importent Lemma \ref{Lem Gen}).
		
		However, since $\Pi$ is a projective generator of $\Rep_{\Lambda}(G)_0$, we have
		$$\Hom_G(\Pi, X) \neq 0,$$
		for any $X \in \mathcal{C}_{x,1}$. This together with the last paragraph implies that 
		$$\Hom_G(\Pi_{x,1}, X) \neq 0,$$
		for any $X \in \mathcal{C}_{x,1}$, i.e. $\Pi_{x,1}$ is a generator of $\mathcal{C}_{x,1}$.
		
		Finally, note $\Pi_{x,1}$ is projective in $\Rep_{\Lambda}(G)_0$ since it is a summand of the projective object $\Pi$. Hence $\Pi_{x,1}$ is projective in $\mathcal{C}_{x,1}$. This together with the last paragraph implies that $\Pi_{x,1}$ is a projective generator of $\mathcal{C}_{x,1}$.
	
		
	\end{proof}
	
	\subsection{Lemmas}
	
	In this subsection, I collect some lemmas used in the proof of Theorem \ref{Thm Proj}.
	

	
	\begin{lemma}\label{Lem Ortho}
		$$\Hom_G(\Pi_{x,1}, \Pi^{x,1})=\Hom_G(\Pi^{x,1}, \Pi_{x,1})=0.$$
	\end{lemma}
	
	\begin{proof}
		Recall that
		$\Pi^{x,1}:=\Pi_x^1 \oplus \Pi^x$.
		
		First, we compute
		$$\Hom_G(\Pi_{x,1}, \Pi_x^1)=\Hom_{G_x}(\sigma_{x,1}, \sigma_x^1)=0,$$
		where the first equality is the first case of Theorem \ref{Thm Hom} (note $\sigma_{x,1} \in \mathcal{B}_{x,1}$, hence has cuspidal reduction by Theorem \ref{Thm Cusp Red}, and hence the condition of Theorem \ref{Thm Hom} is satisfied), and the second equality is because $\sigma_{x,1}$ and $\sigma_x^1$ lies in different blocks of $\Rep_{\Lambda}(G_x)$ by definition.
		
		Second, recall that $\Pi_{x,1}=\cInd_{G_x}\sigma_{x,1}$ with $\sigma_{x,1}$ having cuspidal reduction, and $\Pi_y=\cInd_{G_y}\sigma_y$. We compute 
		$$\Hom_G(\Pi_{x,1}, \Pi^x)=\bigoplus_{y \neq x}\Hom_G(\Pi_{x,1}, \Pi_y)=0,$$
		by the second case of Theorem \ref{Thm Hom}.
		
		Combining the above three paragraphs, we get $\Hom_G(\Pi_{x,1}, \Pi^{x,1})=0$.
		
		A same argument shows that $\Hom_G(\Pi^{x,1}, \Pi_{x,1})=0$.
	\end{proof}
	
	\begin{lemma}[Important Lemma]\label{Lem Gen}
		$\Hom_G(\Pi^{x,1}, X)=0,$
		for any object $X \in \mathcal{C}_{x,1}$.
	\end{lemma}
	
	\begin{proof}
		Recall that 
		$$\Hom_G(\Pi, -): \Rep_{\Lambda}(G)_0 \to \Modr\End_G(\Pi) \simeq \Modr\End_G(\Pi_{x,1}) \oplus \Modr\End_G(\Pi^{x,1})$$ 
		is an equivalence of categories. It is even an equivalence of abelian categories since $\Hom_G(\Pi, -)$ is exact and commutes with direct product. Hence the image of $\mathcal{C}_{x,1}$ must be indecomposable as $\mathcal{C}_{x,1}$ is indecomposable, i.e., 
		$$\Hom_G(\Pi, -)=\Hom_G(\Pi_{x,1}, -) \oplus \Hom_G(\Pi^{x,1}, -)$$
		can map $\mathcal{C}_{x,1}$ nonzeroly to only one of $\Modr\End_G(\Pi_{x,1})$ and $\Modr\End_G(\Pi^{x,1})$ (See the diagram below). 
		
		\begin{tikzcd}
			{\Rep_{\Lambda}(G)_0} &&&& {\Modr \End_G(\Pi)} \\
			\\
			{\mathcal{C}_{x,1}} &&&& {\Modr \End_G(\Pi_{x,1}) \oplus \Modr \End_G(\Pi^{x,1})}
			\arrow["{\Hom_G(\Pi, -)}", from=1-1, to=1-5]
			\arrow["{\Hom_G(\Pi_{x,1}, -) \oplus \Hom_G(\Pi^{x,1}, -)}", from=3-1, to=3-5]
			\arrow["\subset", sloped, from=3-1, to=1-1]
			\arrow["\simeq", sloped, from=3-5, to=1-5]
		\end{tikzcd}
		
		Then it must be $\Modr\End_G(\Pi_{x,1})$ (that $\Hom_G(\Pi, -)$ maps $\mathcal{C}_{x,1}$ nonzeroly to) since 
		$$\Hom_G(\Pi_{x,1}, \pi)=\Hom_G(\sigma_{x,1}, \rho)=\Hom_G(\sigma_x, \rho) \neq 0.$$
		In other words, $\Hom_G(\Pi^{x,1}, -)$ is zero on $\mathcal{C}_{x,1}$.
		
	\end{proof}
	\bibliographystyle{plain}
	\bibliography{reference}
\end{document}
	\documentclass{article}

\special{dvipdfmx:config z 0}

\usepackage{amsmath,amssymb,amsfonts,amsthm,extarrows}
\usepackage{mathtools}
\usepackage{enumitem}
\usepackage{stmaryrd}
\usepackage{tikz-cd} 
\usepackage{bbm}

\usepackage{color}
\newcommand{\red}[1]{\textcolor{red}{#1}}
\newcommand{\blue}[1]{\textcolor{blue}{#1}}

\usepackage{nameref}

\usepackage{graphicx}
\graphicspath{ {./images/} }

\usepackage{soul}

%%%% todo notes %%%%
\usepackage[colorinlistoftodos,textsize=footnotesize]{todonotes}
\setlength{\marginparwidth}{2.5cm}
\newcommand{\leftnote}[1]{\reversemarginpar\marginnote{\footnotesize #1}}
\newcommand{\rightnote}[1]{\normalmarginpar\marginnote{\footnotesize #1}\reversemarginpar}


\usepackage[colorlinks]{hyperref}

\newtheorem*{remark}{Remark}
\newtheorem{theorem}{Theorem}
\newtheorem{lemma}{Lemma}
\newtheorem{question}{Question}
\newtheorem{answer}{Answer}
\newtheorem{proposition}{Proposition}
\newtheorem{definition}{Definition}
\newtheorem{exer}{Exercise}
\newtheorem{corollary}{Corollary}
\newtheorem{example}{Example}
\newtheorem{warning}{Warning}

\DeclareMathOperator{\cInd}{\operatorname{c-Ind}}
\DeclareMathOperator{\Ind}{\operatorname{Ind}}
\newcommand{\Res}{\operatorname{Res}}
\newcommand{\Hom}{\operatorname{Hom}}
\newcommand{\Rep}{\operatorname{Rep}}
\newcommand{\End}{\operatorname{End}}
\newcommand{\GL}{\operatorname{GL}}
\newcommand{\diag}{\operatorname{diag}}
\newcommand{\Mod}{\operatorname{Mod}}
\newcommand{\Irr}{\operatorname{Irr}}
\newcommand{\Modr}{\operatorname{Mod-}}
\newcommand{\Modl}{\operatorname{-Mod}}
\newcommand{\Perf}{\operatorname{Perf}}
\newcommand{\Spec}{\operatorname{Spec}}
\newcommand{\Ob}{\operatorname{Ob}}
\newcommand{\Fr}{\operatorname{Fr}}
\newcommand{\coker}{\operatorname{coker}}
\newcommand{\Cont}{\operatorname{Cont}}
\newcommand{\QCoh}{\operatorname{QCoh}}
\newcommand{\Coh}{\operatorname{Coh}}


\begin{document}
%	In this file, I prove the categorical local Langlands conjecture for depth-zero supercuspidal part of $G=GL_n$.
%	
%	\section{$\Lambda=\overline{\mathbb{Q}_{\ell}}$}
%	
%	\red{Let's first do the $\overline{\mathbb{Q}_{\ell}}$-case, and then see what should be modified to get the general case.}
%	
%	Let $\varphi \in Z^1(W_E, \hat{G}(\overline{\mathbb{Q}_{\ell}}))$ be an irreducible tame $L$-parameter. Let $C_{\varphi}$ be the connected component of $Z^1(W_E, \hat{G})_{\overline{\mathbb{Q}_{\ell}}}$ containing $\varphi$. 
%	
%	The goal is to show that there is an equivalence
%	$$D_{lis}^{C_{\varphi}}(Bun_G, \overline{\mathbb{Q}_{\ell}})^{\omega} \cong D^{b, qc}_{Coh, Nilp}(C_{\varphi})$$
%	of derived categories (it is even expected to be an equivalence as stable $\infty$-categories).
%	
%	Let's spell out both sides of the correspondence explicitly.
%	
%	Let's unravel the left hand side. By \cite[Section X.2]{fargues2021geometrization},
%	$$D_{lis}^{C_{\varphi}}(Bun_G, \overline{\mathbb{Q}_{\ell}})^{\omega} \cong \bigoplus_{b \in B(G)_{basic}}D^{C_{\varphi}}(G_b(F), \overline{\mathbb{Q}_{\ell}})^{\omega}.$$
%	For $G=GL_n$, $B(G)_{basic} \simeq \mathbb{Z}$, and $G_b(F)=GL_n(F)$ for $b=1$ (corresponds to $0 \in \mathbb{Z}$). Moreover, for $b=1$, 
%	$$D^{C_{\varphi}}(G_b(F), \overline{\mathbb{Q}_{\ell}})^{\omega}=D^{C_{\varphi}}(GL_n(F), \overline{\mathbb{Q}_{\ell}})^{\omega}=D(\Rep_{\overline{\mathbb{Q}_{\ell}}}\left(GL_n(F)\right)_{[\pi]})^{\omega},$$
%	where $\pi$ is any irreducible representation with $L$-parameter $\varphi_{\pi}=\varphi$, and $\Rep_{\overline{\mathbb{Q}_{\ell}}}(GL_n(F))_{[\pi]}$ is the block of $\Rep_{\overline{\mathbb{Q}_{\ell}}}\left(GL_n(F)\right)$ containing $\pi$. And we've computed (\red{See ?}) that
%	$$\Rep_{\overline{\mathbb{Q}_{\ell}}}(GL_n(F))_{[\pi]} \simeq \overline{\mathbb{Q}_{\ell}}[t, t^{-1}]\Modl.$$ So we have
%	$$D^{C_{\varphi}}(GL_n(F), \overline{\mathbb{Q}_{\ell}})^{\omega} \simeq \Perf(\overline{\mathbb{Q}_{\ell}}[t, t^{-1}]).$$ For $b \neq 1$, we could take care of it using the spectral action (\red{See ?}).
%	
%	Now let's unravel the right hand side. We first notice that the decorations $qc$ and $Nilp$ goes away in our case. Since we are focusing on one component, the quasi-compact support condition goes away. (\red{Need explain Nilp.}) So 
%	$$D^{b, qc}_{Coh, Nilp}(C_{\varphi}) \simeq D^b_{Coh, \{0\}}(C_{\varphi}) \simeq \Perf(C_{\varphi}).$$
%	By our computation before,
%	$$C_{\varphi} \simeq [\mathbb{G}_m/\mathbb{G}_m] \simeq \mathbb{G}_m \times [*/\mathbb{G}_m]$$
%	where $\mathbb{G}_m$ acts on $\mathbb{G}_m$ via the trivial action. So
%	$$\Perf(C_{\varphi}) \simeq \Perf(\mathbb{G}_m \times [*/\mathbb{G}_m]) \simeq \Perf(\mathbb{G}_m) \otimes \Perf([*/\mathbb{G}_m]).$$
%	Here 
%	$$\Perf([*/\mathbb{G}_m]) \simeq \bigoplus_{\chi}\Perf(\overline{\mathbb{Q}_{\ell}})\chi \simeq \bigoplus_{\chi}\Perf(\overline{\mathbb{Q}_{\ell}}),$$
%	where $\chi \in \{t \mapsto t^n | n \in \mathbb{Z}\}$ runs over all (algebraic) characters of $\mathbb{G}_m$.
%	
%	In conclusion, both sides are isomorphic to $\mathbb{Z}$ copies of $\Perf(\mathbb{G}_m)$, where $\mathbb{Z}$ corresponds to $B(G)_{basic}$ for left hand side and $\mathbb{Z}$ corresponds to the set of algebraic character's of $\mathbb{G}_m$ in the right hand side.
%	
%	The $\mathbb{Z}$-grading on both sides match in the following sense. (\red{Need explain})
%	
%	Therefore, we are reduced to the degree zero case. But this we know from compatibility of Spectral action with the maps between Bernstein centers, and that the maps between Bernstein centers are isomorphism for $GL_n$.
%	
%	
%	\section{$\Lambda=\overline{\mathbb{Z}_{\ell}}$}
%	\begin{enumerate}
%		\item Step 1: Both sides are isomorphic abstractly, as $\mathbb{Z}$ copies of $\Perf(C_{\varphi})$. Here for the $Bun_G$ side, we could argue using the spectral action.
%	    \item Step 2: By compatibility with central character, we are reduced to the degree $0$ part.
%	    \item Step 3: The degree $0$ part follows from compatibility of Spectral action with the maps between Bernstein centers and Helm-Moss (we could even avoid using Helm Moss).
%	\end{enumerate}
%
%    A technical point: the Nilp condition. I claim again $Nilp=\{0\}$. This boils down to compote 
%    $$H^0(W_E, Ad(\varphi)) \cap Nilp(\mathfrak{g})=\{0\}.$$




%\chapter{Example: $GL_n(F)$}

Let's apply the theories in the previous chapters to the example of $GL_n(F)$. Throughout this chapter, $G:=GL_n$.

That said, there is a little mismatch between the theories before and the example here, namely, we assumed for simplicity in the theories that $G$ is simply connected (and in particular, semisimple), while this is not the case for $G=GL_n$. However, there is only some minor difference due to the center $\mathbb{G}_m$ of $GL_n$. I leave it as an exercise for the readers to figure out the details.

\section{$L$-parameter side}
Let $\varphi \in Z^1(W_F, \hat{G}(\overline{\mathbb{F}_{\ell}}))$ be an irreducible tame $L$-parameter. Let $\psi \in Z^1(W_F, \hat{G}(\overline{\mathbb{Z}_{\ell}}))$ be any lift of $\varphi$. Let $C_{\varphi}$ be the connected component of $Z^1(W_F, \hat{G})_{\overline{\mathbb{Z}_{\ell}}}$ containing $\varphi$. By \red{Need ref?}, we compute that
%$$C_{\varphi} \cong \left(\hat{G} \times Z^1(W_F, N_{\hat{G}}(\psi_{\ell}))_{\psi_{\ell}, \overline{\psi}}\right)/C_{\hat{G}}(\psi_{\ell})_{\overline{\psi}}.$$
%Here 
%$$Z^1(W_F, N_{\hat{G}}(\psi_{\ell}))_{\psi_{\ell}, \overline{\psi}} \cong Z^1_{Ad(\psi)}(W_F, N_{\hat{G}}(\psi_{\ell})^0)_{1_{I_F^{\ell}}}.$$
%In our case, $N_{\hat{G}}(\psi_{\ell})^0$ is the diagonal torus $T$ of $GL_n$.
$$C_{\varphi} \cong [T/T] \times \mu,$$
where $T=C_{\hat{G}}(\psi_{\ell})$ is a maximal torus of $GL_n$, and $\mu=(T^{Fr=(-)^q})^0$, and the $T$-action on $T$ is specified in \red{Need ref?}. To go further, let's choose a nice basis of the Weil group representations $\varphi$ and $\psi$.

Indeed, every irreducible tame $L$-parameter with $\overline{\mathbb{F}_{\ell}}$-coefficients $\varphi$ of $GL_n$ are of the form $\varphi=Ind_{W_E}^{W_F}\eta$, where $E$ is a degree $n$ unramified extension of $F$, $W_E \cong I_F \rtimes \left<\Fr^n\right>$ is the Weil group of $E$, and $\eta: W_E \to \overline{\mathbb{F}_{\ell}}^*$ is a tame (i.e., trivial on $P_E=P_F$) character of $W_E$ such that $\{\eta, \eta^q, ..., \eta^{q^{n-1}}\}$ are distinct. To find a lift of it with $\overline{\mathbb{Z}_{\ell}}$-coefficients, we let $\tilde{\eta}: W_E \to \overline{\mathbb{Z}_{\ell}}^*$, and let $\psi:=Ind_{W_E}^{W_F}\tilde{\eta}$. Then under a nice basis, we could specify the matrices corresponds to the topological generater $s_0$ and $Fr$:
$$\psi(s_0)=
\begin{bmatrix}
	\tilde{\eta}(s_0) & 0                   & 0      & \dots  & 0 \\
	0                 & \tilde{\eta}(s_0)^q & 0      & \dots  & 0 \\
	\vdots            & \vdots              & \vdots & \ddots & \vdots \\
	0                 & 0                   & 0      & \dots   & \tilde{\eta}(s_0)^{q^{n-1}}
\end{bmatrix}$$
and 
$$\psi(\Fr)=
\begin{bmatrix}
	0                   & 1      & 0      & \dots  & 0 \\
	0                   & 0      & 1      & \dots  & 0 \\
	\vdots              & \vdots & \vdots & \ddots & \vdots \\
	0                   & 0      & 0      & \dots  & 1 \\
	\tilde{\eta}(\Fr^n) & 0      & 0      & \dots  & 0
\end{bmatrix}
.$$
Under this basis, $T=C_{\hat{G}}(\psi_{\ell})$ is the diagonal torus of $GL_n$, with $\Fr$ acting by conjugacy via $\psi$, i.e., 
$$\Fr. \diag(t_1, t_1, ..., t_{n-1}, t_{n}) = \diag(t_{2}, t_{3}, ..., t_{n}, t_{1}).$$
So one could compute that 
$$T^{\Fr=(-)^q}\cong \mu_{q^n-1},$$
and that
$$(T^{\Fr=(-)^q})^0 \cong \mu_{\ell^k},$$
where $k \in \mathbb{Z}$ is maximal such that $\ell^k$ divides $q^n-1$.

To compute the quotient $[T/T]$, we note that $T$ acts on $T$ via twisted conjugacy
$$(t, t') \mapsto (tnt^{-1}n^{-1})t',$$
where $n$ is same as $\psi(Fr)$ in effect. So in our case, this action is 
$$(t_1, t_2, ..., t_n).(t'_1, t'_2, ..., t'_n)=(t_n^{-1}t_1t'_1, t_1^{-1}t_2t'_2, ..., t_{n-1}^{-1}t_nt'_n).$$ 
We see that the orbits of this action are determined by the determinants (hence are in bijection with $\mathbb{G}_m$), and the center $\mathbb{G}_m \cong Z \subset T$ acts trivially. Therefore,
$$[T/T] \cong [\mathbb{G}_m/\mathbb{G}_m],$$
where $\mathbb{G}_m$ acts trivially on $\mathbb{G}_m$.

In conclusion, we have that the connected component of $Z^1(W_F, \hat{G})_{\overline{\mathbb{Z}_{\ell}}}$ containing $\varphi$ is
$$C_{\varphi} \cong [\mathbb{G}_m/\mathbb{G}_m] \times \mu_{\ell^k},$$
where $\mathbb{G}_m$ acts trivially on $\mathbb{G}_m$, and $k \in \mathbb{Z}$ is maximal such that $\ell^k$ divides $q^n-1$.


\section{Representation side}

By modular Deligne-Lusztig theory, the block $\mathcal{A}_{x,1}$ of $GL_n(\mathbb{F}_q)$ containing a cuspidal representation $\sigma$ is equivalent to the block of an elliptic torus, which is isomorphic to $\mathbb{F}_{q^n}^*$. So this block is equivalent to $\overline{\mathbb{Z}_{\ell}}[s]/(s^{\ell^k}-1)$, where $k \in \mathbb{Z}$ is maximal such that $\ell^k$ divides $q^n-1$.

$\mathcal{A}_{x,1}$ inflats to a block of $K:=GL_n(\mathcal{O}_F)$ containing the inflation $\tilde{\sigma}$ of $\sigma$, and further corresponds to a block $\mathcal{B}_{x,1}$ of $KZ$ containing $\rho$, a extension of $\tilde{\sigma}$ to $KZ$, where $Z$ is the center of $GL_n(F)$. We have
$$\mathcal{B}_{x,1} \cong \mathcal{A}_{x,1} \otimes \Rep_{\overline{\mathbb{Z}_{\ell}}}(\mathbb{Z}) \cong \overline{\mathbb{Z}_{\ell}}[s]/(s^{\ell^k}-1) \otimes \overline{\mathbb{Z}_{\ell}}[t, t^{-1}]\Modl,$$
because
$$KZ \cong K \times \{\diag(\pi^m, ..., \pi^m | m \in \mathbb{Z})\} \cong K \times \mathbb{Z}.$$
Argue as before (\red{See ?}) we see that the compact induction $\cInd_{KZ}^G$ induces an equivalence of categories
$$\mathcal{B}_{x,1} \cong \mathcal{C}_{x,1},$$
where $\mathcal{C}_{x,1}$ is the block of $\Rep_{\overline{\mathbb{Z}_{\ell}}}(G(F))$ containing $\pi:=\cInd_{KZ}^G\rho$.

Since every depth-zero supercuspidal representation $\pi$ arises as above, we have that the block containing $\pi$ satisfies
$$\Rep_{\overline{\mathbb{Z}_{\ell}}}(G(F))_{[\pi]} \cong \mathcal{C}_{x,1} \cong \overline{\mathbb{Z}_{\ell}}[s]/(s^{\ell^k}-1) \otimes \overline{\mathbb{Z}_{\ell}}[t, t^{-1}]\Modl.$$



\pagebreak
%\chapter{The categorical local Langlands conjecture}

In this chapter, I prove the categorical local Langlands conjecture for depth-zero supercuspidal part of $G=GL_n$ with coefficients $\Lambda=\overline{\mathbb{Z}_{\ell}}$.

Let $\varphi \in Z^1(W_E, \hat{G}(\overline{\mathbb{F}_{\ell}}))$ be an irreducible tame $L$-parameter. Let $C_{\varphi}$ be the connected component of $Z^1(W_E, \hat{G})_{\overline{\mathbb{Z}_{\ell}}}$ containing $\varphi$. 

The goal is to show that there is an equivalence
$$D_{lis}^{C_{\varphi}}(Bun_G, \overline{\mathbb{Z}_{\ell}})^{\omega} \cong D^{b, qc}_{Coh, Nilp}(C_{\varphi})$$
of derived (\red{?}) categories.

As a first step, let's unravel the definition of both sides and describe them explicitly.

\section{Unraveling definitions}

\subsection{$L$-parameter side}

Let's first state a lemma that makes the decorations in $D^{b, qc}_{Coh, Nilp}(C_{\varphi})$ go away. We postpone its proof to a later subsection.

\begin{lemma} \label{Lemma 1}
	$D^{b, qc}_{Coh, Nilp}(C_{\varphi}) \cong D^b_{Coh, \{0\}}(C_{\varphi}) \cong \Perf(C_{\varphi}).$
\end{lemma} 
	
Let's assume the lemma for the moment and continue. By our computation before,
$$C_{\varphi} \cong [\mathbb{G}_m/\mathbb{G}_m] \times \mu_{\ell^k} \cong \mathbb{G}_m \times [*/\mathbb{G}_m] \times \mu_{\ell^k},$$
where $k \in \mathbb{Z}_{\geq 0}$ is maximal such that $\ell^k$ divides $q^n-1$. So
$$\Perf(C_{\varphi}) \cong \Perf(\mathbb{G}_m \times [*/\mathbb{G}_m] \times \mu_{\ell^k}) \simeq \Perf(\mathbb{G}_m) \otimes \Perf([*/\mathbb{G}_m]) \otimes \Perf(\mu_{\ell^k}).$$
Here,
$$\Perf([*/\mathbb{G}_m]) \cong \bigoplus_{\chi}\Perf(\overline{\mathbb{Z}_{\ell}})\chi \cong \bigoplus_{\chi}\Perf(\overline{\mathbb{Z}_{\ell}}),$$
where $\chi$ runs over characters of $\mathbb{G}_m$ 
$$X^*(\mathbb{G}_m)=\{t \mapsto t^m | m \in \mathbb{Z}\} \cong \mathbb{Z}.$$

In conclusion, we have 
$$\Perf(C_{\varphi}) \cong \bigoplus_{\chi}\Perf(\mathbb{G}_m \times \mu_{\ell^k}),$$
where $\chi$ runs over characters of $\mathbb{G}_m$ 
$$X^*(\mathbb{G}_m)=\{t \mapsto t^m | m \in \mathbb{Z}\} \cong \mathbb{Z}.$$


\subsection{$Bun_G$ side}

Since $\varphi$ is irreducible, 
$$D^{C_{\varphi}}_{lis}(Bun_G, \overline{\mathbb{Z}_{\ell}})^{\omega} \cong D^{C_{\varphi}}_{lis}(Bun_G^{ss}, \overline{\mathbb{Z}_{\ell}})^{\omega}.$$

Since
$$Bun_G^{ss}=\sqcup_{b \in B(G)_{basic}}[*/G_b(F)],$$
we have 
$$D^{C_{\varphi}}_{lis}(Bun_G^{ss}, \overline{\mathbb{Z}_{\ell}})^{\omega} \cong \bigoplus_{b \in B(G)_{basic}}D^{C_{\varphi}}(G_b(F), \overline{\mathbb{Z}_{\ell}})^{\omega}.$$

Let's look closer into each direct summand. In our case $G=GL_n$, 
$$B(G)_{basic} \cong \pi_1(G)_{\Gamma} \cong \mathbb{Z}$$. 

Let's first look at the summand for $b=1$ (corresponding to $0 \in \mathbb{Z} \cong B(G)_{basic}$). For $b=1$, $G_b \cong GL_n$, and 
$$D^{C_{\varphi}}(G_b(F), \overline{\mathbb{Z}_{\ell}})^{\omega} \cong D^{C_{\varphi}}(GL_n(F), \overline{\mathbb{Z}_{\ell}})^{\omega} \cong D(\Rep_{\overline{\mathbb{Z}_{\ell}}}(GL_n(F))_{[\pi]})^{\omega},$$
where $\pi \in \Rep_{\overline{\mathbb{F}_{\ell}}}(GL_n(F))$ is the representation with $L$-parameter $\varphi$, and $\Rep_{\overline{\mathbb{Z}_{\ell}}}(GL_n(F))_{[\pi]}$ is the block of $\Rep_{\overline{\mathbb{Z}_{\ell}}}(GL_n(F))$ containing $\pi$.
And we've computed that
$$\Rep_{\overline{\mathbb{Z}_{\ell}}}(GL_n(F))_{[\pi]} \cong \overline{\mathbb{Z}_{\ell}}[t, t^{-1}] \otimes \overline{\mathbb{Z}_{\ell}}[s]/(s^{\ell^k}-1)\Modl \cong \QCoh(\mathbb{G}_m \times \mu_{\ell^k}),$$
where $k \in \mathbb{Z}_{\geq 0}$ is again maximal such that $\ell^k$ divides $p^n-1$. So we have
$$D^{C_{\varphi}}(GL_n(F), \overline{\mathbb{Z}_{\ell}})^{\omega} \cong D(\QCoh(\mathbb{G}_m \times \mu_{\ell^k}))^{\omega} \cong \Perf(\mathbb{G}_m \times \mu_{\ell^k}).$$

We could get a similar description of $D^{C_{\varphi}}(G_b(F), \overline{\mathbb{Z}_{\ell}})$ for free by the spectral action and the compatibility of Fargues-Scholze with $\pi_1(G)_{\Gamma}$-grading. For this, we consider the composition
$$q: C_{\varphi} \cong \mathbb{G}_m \times [*/\mathbb{G}_m] \times \mu_{\ell^k} \to [*/\mathbb{G}_m].$$
Recall that 
$$\Perf([*/\mathbb{G}_m]) \cong \bigoplus_{\chi}\Perf(\overline{\mathbb{Z}_{\ell}})\chi,$$
we denote by $\mathcal{M}_{\chi}$ the corresponding simple object in $\Perf([*/\mathbb{G}_m])$. Moreover, $\mathcal{M}_{\chi}$ pullbacks to a line bundle
$$\mathcal{L}_{\chi}:=q^*\mathcal{M}_{\chi}.$$
We could now state the key proposition that allows us to get to arbitrary $b \in B(G)_{basic}$ from the $b=1$ case, using the spectral action.
\begin{proposition}\label{Prop Spectral action}
	\begin{enumerate}
		\item The restriction of the spectral action by $\mathcal{L}_{\chi}$ to $D(G_b(F), \overline{\mathbb{Z}_{\ell}})$ factors through $D(G_{b-\chi}(F), \overline{\mathbb{Z}_{\ell}})$.
		
		\begin{tikzcd}
			{\mathcal{L}_{\chi}*-:} & {D_{lis}(Bun_G, \overline{\mathbb{Z}_{\ell}})} && {D_{lis}(Bun_G, \overline{\mathbb{Z}_{\ell}})} \\
			\\
			& {D(G_b(F), \overline{\mathbb{Z}_{\ell}})} && {D(G_{b-\chi}(F), \overline{\mathbb{Z}_{\ell}})}
			\arrow[from=1-2, to=1-4]
			\arrow[from=3-2, to=3-4]
			\arrow["\subset", from=3-2, to=1-2]
			\arrow["\subset", from=3-4, to=1-4]
		\end{tikzcd}
		\item $\mathcal{L}_{\chi}*-: D(G_b(F), \overline{\mathbb{Z}_{\ell}}) \to D(G_{b-\chi}(F), \overline{\mathbb{Z}_{\ell}})$ is an equivalence of categories, with inverse $\mathcal{L}_{\chi^{-1}}*-$.
	\end{enumerate}
\end{proposition}

\begin{proof}
	For the first assertion, see \cite[Lemma 5.3.2]{zou2022categorical}. For the second assertion, note that $\mathcal{L}_{\chi}$ and $\mathcal{L}_{\chi^{-1}}$ are clearly inverse to each other once they are well-defined, since $q^*$ preserves tensor product.
\end{proof}
So we have 
$$D^{C_{\varphi}}(Bun_G, \overline{\mathbb{Z}_{\ell}})^{\omega} \cong \bigoplus_{b \in B(G)_{basic}}D^{C_{\varphi}}(G_b(F), \overline{\mathbb{Z}_{\ell}}) \cong \bigoplus_{b \in B(G)_{basic}}\Perf(\mathbb{G}_m \times \mu_{\ell^k}).$$

\subsection{Proof of Lemma \ref{Lemma 1}}
Now we prove Lemma \ref{Lemma 1}. 

The first isomorphism is because $C_{\varphi}$ is connected, hence the quasicompact support condition $qc$ is automatic. 

The second isomorphism needs some computation. For the definition and properties of the nilpotent singular support condition $Nilp$, I refer to \cite[Section VIII.2]{fargues2021geometrization}. At the end of the day, it boils to the fact that
$$H^0(W_F, \hat{\mathfrak{g}}^*\otimes_{\mathbb{Z}_{\ell}}\Lambda(1)) \cap Nilp(\hat{\mathfrak{g}}^*)=\{0\}.$$ (\red{Maybe elaborate more.})


\section{The spectral action induces an equivalence of categories}
To summarize, we have (abstract) equivalence of categories
$$D^{b, qc}_{Coh, Nilp}(C_{\varphi}) \cong \bigoplus_{\chi \in \mathbb{Z}}\Perf(\mathbb{G}_m \times \mu_{\ell^k}) \cong \bigoplus_{b \in \mathbb{Z}}\Perf(\mathbb{G}_m \times \mu_{\ell^k}) \cong D^{C_{\varphi}}_{lis}(Bun_G, \overline{\mathbb{Z}_{\ell}})^{\omega},$$
where I identified both $X^*(\mathbb{G}_m) \cong X^*(Z(\hat{G}))$ and $B(G)_{basic} \cong \pi_1(G)_{\Gamma}$ with $\mathbb{Z}$. The next goal is to show that the spectral action induces an equivalence of categories
\begin{equation}\label{Equiv}
	D_{lis}^{C_{\varphi}}(Bun_G, \overline{\mathbb{Z}_{\ell}})^{\omega} \cong D^{b, qc}_{Coh, Nilp}(C_{\varphi}).
\end{equation}

%\subsection{Equivalence on degree $0$ part}
%By compatibility of the spectral action with the map 
%$$\psi_G: \mathcal{O}(Z^1(W_F, \hat{G})/\hat{G}) \to \mathcal{Z}(\Rep(G(E)))$$
%between Bernstein centers (\red{?}), we reduce to show that the restriction 
%$$\psi_G|_{\mathcal{O}(C_{\varphi})}: \mathcal{O}(C_{\varphi}) \to \mathcal{Z}(\Rep(G(E))_{[\pi]})$$
%is an equivalence of categories. For this, we could refer to \cite{helm2018converse} (\red{?}). 
%
%\subsection{The full equivalence}
%Now we could use the compatibility of the spectral action with the $\pi_1(G)_{\Gamma}$-grading to get the full equivalence \ref{Equiv}. For this, I refer to \cite{zou2022categorical}.

%For this, we argue as in \cite[Section 5, 6]{zou2022categorical}.
%
%Let's first define the functor. For this, let's fix a Whittaker datum
%
%\subsection{Equivalence for the degree $0$ part}
%\begin{proposition}
%	内容...
%\end{proposition}
%
%\subsection{The full equivalence}

%Let's first define the functor. Recall the notation from the previous Chapter \red{?} that $\mathcal{C}_{x, 1}$ is the block of $\Rep_{\overline{\mathbb{Z}_{\ell}}}(G(F))$ containing $\pi$, and we have a projective generator $\Pi_{x, 1}=\cInd_{G_x}^G\sigma_{x, 1}$ of it. We define the functor by spectral acting on $\Pi_{x, 1}$:
%$$\Theta: D^{b, qc}_{Coh, Nilp}(C_{\varphi}) \cong \Perf(C_{\varphi}) \longrightarrow D_{lis}^{C_{\varphi}}(Bun_G, \overline{\mathbb{Z}_{\ell}})^{\omega}, \qquad A \mapsto A*\Pi_{x, 1},$$
%where I abuse the notation and see $\Pi_{x, 1}$ as an element in $D_{lis}^{C_{\varphi}}(Bun_G, \overline{\mathbb{Z}_{\ell}})^{\omega}$ via 
%$$(i_1)_*: D(GL_n(F), \overline{\mathbb{Z}_{\ell}}) \to D_{lis}(Bun_G, \overline{\mathbb{Z}_{\ell}})^{\omega}.$$
%
%\begin{remark}
%	\red{Need to check: the structure sheaf goes to the Whittaker sheaf.}
%\end{remark} 
%
%Let's first show that $\Theta$ is an equivalence on degree zero part. It suffices to show that the composition
%$$\Perf(C_{\varphi})_{\chi=0} \cong \Perf(\mathbb{G}_m \times \mu_{\ell^k}) \to D(\Rep_{\overline{\mathbb{Z}_{\ell}}}(GL_n(F))_{[\pi]})^{\omega} \cong D()$$

\subsection{Definition of the functor}

Let's first define the functor. For this, let's choose a Whittaker datum consisting of a Borel $B \subset G$ and a generic character $\vartheta: U(F) \to \overline{\mathbb{Z}_{\ell}}^*$. Let $\mathcal{W}_{\vartheta}$ be the sheaf concentrated on $Bun_G^1$ corresponding to the representation $W_{\vartheta}:=\cInd_{U(F)}^{G(F)}\vartheta$. Let $W_{\vartheta, [\pi]}$ be the restriction of $W_{\vartheta}$ to the block $\Rep_{\overline{\mathbb{Z}_{\ell}}}(G(F))_{[\pi]}$, and $\mathcal{W}_{\vartheta, [\pi]}$ the corresponding sheaf.

We define our desired functor by spectral acting on $\mathcal{W}_{\vartheta, [\pi]}$:
$$\Theta: D^{b, qc}_{Coh, Nilp}(C_{\varphi}) \cong \Perf(C_{\varphi}) \longrightarrow D_{lis}^{C_{\varphi}}(Bun_G, \overline{\mathbb{Z}_{\ell}})^{\omega}, \qquad A \mapsto A*\mathcal{W}_{\vartheta, [\pi]}.$$

\subsection{Equivalence on degree zero part}

We now show that $\Theta$ induces an equivalence on degree zero part. At the end of the day, this is similar to the following fact: If I have a functor $F: R\Modl \to R\Modl$, which is $(R\Modl)$-linear and sends $R$ to $R$, then $F$ is an equivalence of category. 

By compatibility with $\pi_1(G)_{\Gamma}$-grading, $\Theta$ restricts to a map
$$\Theta_0:=\Theta|_{\Perf(C_{\varphi})_{\chi=0}}: \Perf(C_{\varphi})_{\chi=0} \longrightarrow D_{lis}^{C_{\varphi}}(Bun_G, \overline{\mathbb{Z}_{\ell}})^{\omega}_{b=0},$$
where $\Perf(C_{\varphi})_{\chi=0} \cong \Perf(\mathbb{G}_m \times \mu_{\ell^k})$ and 
$$D_{lis}^{C_{\varphi}}(Bun_G, \overline{\mathbb{Z}_{\ell}})^{\omega}_{b=0} \cong D(\Rep_{\overline{\mathbb{Z}_{\ell}}}(G(F))_{[\pi]})^{\omega} \cong D(\End(W_{\vartheta, [\pi]})\Modl)^{\omega}.$$
By tracking the definition, the structure sheaf $\mathcal{O} \in \Perf(\mathbb{G}_m \times \mu_{\ell^k})$ goes to the Whittaker representation $W_{\vartheta, [\pi]} \in D(\Rep_{\overline{\mathbb{Z}_{\ell}}}(G(F))_{[\pi]})^{\omega}$, and further goes to $\End(W_{\vartheta, [\pi]}) \in D(\End(W_{\vartheta, [\pi]})\Modl)$. Moreover, by local Langlands in family (\red{See ?}), 
$$\End(W_{\vartheta, [\pi]}) \cong \mathcal{Z}(G)_{[\pi]} \cong \mathcal{O}(C_{\varphi}) \cong \mathcal{O}(\mathbb{G}_m \times \mu_{\ell^k}).$$ Therefore, we have a functor $\Theta_0: \Perf(\mathbb{G}_m \times \mu_{\ell^k}) \to \Perf(\mathbb{G}_m \times \mu_{\ell^k})$ which is $\Perf(\mathbb{G}_m \times \mu_{\ell^k})$-linear and sends the structure sheaf to the structure sheaf, hence an equivalence of categories.

\subsection{The full equivalence}	

Finally, we use the spectral action to get the full equivalence. Indeed, on the $L$-parameter side, for any character $\chi' \in X^*(\mathbb{G}_m)$, tensoring with $\mathcal{\mathcal{L}_{\chi'}}$ induces an equivalence
$$\mathcal{\mathcal{L}_{\chi'}} \otimes -: \Perf(C_{\varphi})_{\chi=0} \cong \Perf(C_{\varphi})_{\chi=\chi'}.$$
Similarly, on the $Bun_G$ side, by Proposition \ref{Prop Spectral action}, spectral acting by $\mathcal{\mathcal{L}_{\chi'}}$ induces an equivalence
$$\mathcal{\mathcal{L}_{\chi'}}*-: D_{lis}^{C_{\varphi}}(Bun_G, \overline{\mathbb{Z}_{\ell}})^{\omega}_{b=0} \cong D_{lis}^{C_{\varphi}}(Bun_G, \overline{\mathbb{Z}_{\ell}})^{\omega}_{b=-\chi'}.$$ Therefore, we get the full equivalence via the spectral action.
	
\bibliographystyle{plain}
\bibliography{reference}
\end{document}
	
	
	%\section{Introduction}
	%This is a document for beginning with ease. Sometimes I felt disturbed by the structures of the \LaTeX\;document. I don't know how to reset the arranges among paragraphs, and some environments crash with each other.
	%
	%
	%
	%
	%The structure of documents:
	%\begin{enumerate}
	%\item document class;
	%\item packages;
	%\item symbols, containing math operators and other symbols;
	%\item global settings;
	%\item blocks for special features;
	%\end{enumerate}
	%\begingroup
	%\renewcommand{\arraystretch}{1.2}
	%
	%% https://q.uiver.app/?q=WzAsMTAsWzAsMCwiXFxSZXBfe1xcTGFtYmRhfShLWikiXSxbMCwxLCJcXFJlcF97XFxMYW1iZGF9KEtaKV8wIl0sWzAsMiwiXFxtYXRoY2Fse0J9Il0sWzEsMiwiXFxtYXRoY2Fse0N9Il0sWzAsMywiXFxtYXRoY2Fse0J9XzEiXSxbMSwzLCJcXG1hdGhjYWx7Q31fMSciXSxbMSwwLCJcXFJlcF97XFxMYW1iZGF9KEcpIl0sWzEsMSwiXFxSZXBfe1xcTGFtYmRhfShHKV8wIl0sWzIsMywiXFxNb2QoXFxHYW1tYV97XFxwaV8xfSkiXSxbMywzLCJcXG1hdGhjYWx7Q31fMSJdLFs0LDUsIlxcc2ltIl0sWzIsMywiXFxzaW0gXFx0ZXh0e2ZvciBmLmwufSJdLFsxLDddLFswLDYsIlxcY0luZF97S1p9XntHfSJdLFsxLDAsIlxcc3Vic2V0IiwxLHsic3R5bGUiOnsiYm9keSI6eyJuYW1lIjoibm9uZSJ9LCJoZWFkIjp7Im5hbWUiOiJub25lIn19fV0sWzIsMSwiXFxzdWJzZXQiLDEseyJzdHlsZSI6eyJib2R5Ijp7Im5hbWUiOiJub25lIn0sImhlYWQiOnsibmFtZSI6Im5vbmUifX19XSxbMyw3LCJcXHN1YnNldCIsMSx7InN0eWxlIjp7ImJvZHkiOnsibmFtZSI6Im5vbmUifSwiaGVhZCI6eyJuYW1lIjoibm9uZSJ9fX1dLFs3LDYsIlxcc3Vic2V0IiwxLHsic3R5bGUiOnsiYm9keSI6eyJuYW1lIjoibm9uZSJ9LCJoZWFkIjp7Im5hbWUiOiJub25lIn19fV0sWzQsMiwiXFxzdWJzZXQiLDEseyJzdHlsZSI6eyJib2R5Ijp7Im5hbWUiOiJub25lIn0sImhlYWQiOnsibmFtZSI6Im5vbmUifX19XSxbNSwzLCJcXHN1YnNldCIsMSx7InN0eWxlIjp7ImJvZHkiOnsibmFtZSI6Im5vbmUifSwiaGVhZCI6eyJuYW1lIjoibm9uZSJ9fX1dLFs1LDgsIlxcY29uZyIsMSx7InN0eWxlIjp7ImJvZHkiOnsibmFtZSI6Im5vbmUifSwiaGVhZCI6eyJuYW1lIjoibm9uZSJ9fX1dLFs4LDksIlxcY29uZyIsMSx7InN0eWxlIjp7ImJvZHkiOnsibmFtZSI6Im5vbmUifSwiaGVhZCI6eyJuYW1lIjoibm9uZSJ9fX1dLFs5LDcsIlxcc3Vwc2V0IiwyLHsic3R5bGUiOnsidGFpbCI6eyJuYW1lIjoiaG9vayIsInNpZGUiOiJib3R0b20ifX19XV0=
	%\[\begin{tikzcd}[column sep={between origins, 12mm},row sep=small]
		%	{\Rep_{\Lambda}(KZ)} & [37mm]{\Rep_{\Lambda}(G)} \\
		%	{\Rep_{\Lambda}(KZ)_0} & {\Rep_{\Lambda}(G)_0} &&[7mm]\\
		%	{\mathcal{B}} & {\mathcal{C}} \\
		%	{\mathcal{B}_1} & {\mathcal{C}_1'} & {\!\!\End_G(\Pi_1)\Modr} & {\mathcal{C}_1}
		%	\arrow["\sim", from=4-1, to=4-2]
		%	\arrow["{\sim \text{ for f.l.}}", from=3-1, to=3-2]
		%	\arrow[from=2-1, to=2-2]
		%	\arrow["{\cInd_{KZ}^{G}}", from=1-1, to=1-2]
		%	\arrow["\subset"{description}, sloped, draw=none, from=2-1, to=1-1]
		%	\arrow["\subset"{description}, sloped, draw=none, from=3-1, to=2-1]
		%	\arrow["\subset"{description}, sloped, draw=none, from=3-2, to=2-2]
		%	\arrow["\subset"{description}, sloped, draw=none, from=2-2, to=1-2]
		%	\arrow["\subset"{description}, sloped, draw=none, from=4-1, to=3-1]
		%	\arrow["\subset"{description}, sloped, draw=none, from=4-2, to=3-2]
		%	\arrow["\cong"{description}, draw=none, from=4-2, to=4-3]
		%	\arrow["\cong"{description}, draw=none, from=4-3, to=4-4]
		%	\arrow["\supset"{sloped}, hook', from=4-4, to=2-2]
		%\end{tikzcd}\]
		%
		%\begin{table}[ht]
		%\centering
		%\[
		%\begin{array}{c|c|c|c|c|c|c|c}
		%\hline
		%M                      & M(1) & M(2) & M(3) & M(4) & M(5) & M(6)  & \href{http://oeis.org}{OEIS} \\ \hline
		%\Com                   & 1    & 1    & 1    & 1    & 1    & 1     &      \\ \hline
		%\Ass                   & 1    & 2    & 6    & 24   & 120  & 720   &      \\ \hline
		%\Lie                   & 1    & 1    & 2    & 6    & 24   & 120   &      \\ \hline
		%\mathcal{T}(E_{\Com})  & 1    & 1    & 3    & 15   & 105  & 945   &  \href{http://oeis.org/A001147}{A001147}    \\ \hline
		%\mathcal{T}(E_{\Ass})  & 1    & 2    & 12   & 120  & 1680 & 30240 & \href{http://oeis.org/A001813}{A001813}     \\ \hline
		%\mathcal{T}(E_{\Lie})  & 1    & 1    & 3    & 15   & 105  & 945   & \href{http://oeis.org/A001147}{A001147}     \\ \hline
		%\left(R_{\Com}\right)  & 0    & 0    & 2    & 14   & 104  & 944   &      \\ \hline
		%\left(R_{\Ass}\right)  & 0    & 0    & 6    & 96   & 1560 & 29520 &      \\ \hline
		%\left(R_{\Ass}\right)  & 0    & 0    & 1    & 9    & 81   & 825   &      \\ \hline
		%\calEnd_{\mathbb{C}^k} & k^2  & 2k^2 & 3k^2 & 4k^2 & 5k^2 & 6k^2  &      \\ \hline
		%\Com \circ \Lie        &      &      &      &      &      &       &      \\ \hline
		%\vdots                 &      &      &      &      &      &       &      \\ \hline
		%                       &      &      &      &      &      &       &      \\ \hline
		%                       &      &      &      &      &      &       &      \\ \hline
		%\end{array}
		%\]
		%\end{table}
		%\endgroup
		%\section{Examples}
		%\subsection{Theorem environment}
		%\begin{theorem}[{see \cite[Theorem 18.5.1]{vakil2017rising}}]
		%内容...
		%\end{theorem}
		%
		%\begin{setting}
		%内容...
		%\end{setting}
		%
		%\begin{definition}
		%内容...
		%\end{definition}
		%
		%\begin{lemma}
		%内容...
		%\end{lemma}
		%
		%\begin{proposition}
		%内容...
		%\end{proposition}
		%
		%\begin{corollary}
		%内容...
		%\end{corollary}
		%\begin{conjecture}
		%内容...
		%\end{conjecture}
		%\begin{claim}
		%内容...
		%\end{claim}
		%\begin{eg}
		%内容...
		%\end{eg}
		%\begin{ex}
		%内容...
		%\end{ex}
		%\begin{fact}
		%内容...
		%\end{fact}
		%\begin{ques}
		%内容...
		%\end{ques}
		%\begin{warning}
		%内容...
		%\end{warning}
		%\begin{bbox}
		%内容...
		%\end{bbox}
		%\begin{notation}
		%内容...
		%\end{notation}
		%
		%\begin{remark}
		%内容...
		%\end{remark}
		%
		%\begin{remarks}\
		%\begin{enumerate}[1.]
		%\item ...
		%\item ...
		%\end{enumerate}
		%\end{remarks}
		
		%\include{chapters/chapter9}
		
		%\nocite{Eberhardt2022Koszul}	% cite articles which are not cited in the document yet
		
		% Remember to protect the uppercase of people's name and LaTeX symbols
		
		\chapter{Depth-zero regular supercuspidal blocks} \label{Chapter Rep}
		
		The goal of this chapter is to describe the block $\Rep_{\overline{\mathbb{Z}_{\ell}}}(G(F))_{[\pi]}$ (denoted $\mathcal{C}_{x,1}$ later) of $\Rep_{\overline{\mathbb{Z}_{\ell}}}(G(F))$ containing a  depth-zero regular supercuspidal representation $\pi$.
		
		Recall that a depth-zero regular supercuspidal representation $\pi$ is of the form
		$$\pi=\cInd_{G_x}^{G(F)}\rho,$$
		where $\rho$ is a representation of $G_x$ whose reduction $\overline{\rho}$ to the finite reductive group
		$\overline{G_x}=G_x/G_x^+$ is supercuspidal.
		
		In the end, assuming that $G$ is simply connected, the block $\Rep_{\overline{\mathbb{Z}_{\ell}}}(G(F))_{[\pi]}$ would be equivalent to the block $\Rep_{\overline{\mathbb{Z}_{\ell}}}(\overline{G_x})_{[\overline{\rho}]}$ (denoted $\mathcal{A}_{x,1}$ later) of $\Rep_{\overline{\mathbb{Z}_{\ell}}}(\overline{G_x})$ containing $\overline{\rho}$. And $\mathcal{A}_{x,1}$ has an explicit description via the Broué equivalence \ref{Thm Broué}.
		
		Indeed, let $\Rep_{\overline{\mathbb{Z}_{\ell}}}(G_x)_{[\rho]}$ (denoted $\mathcal{B}_{x,1}$ later) be the block of $\Rep_{\overline{\mathbb{Z}_{\ell}}}(G_x)$ containing $\rho$. It is not hard to see that the inflation along $G_x \to \overline{G_x}$ induces an equivalence of categories 
		$\mathcal{A}_{x,1} \cong \mathcal{B}_{x,1}$. The main theorem we prove in this chapter is that the compact induction induces an equivalence of categories
		$$\cInd_{G_x}^{G(F)}: \mathcal{B}_{x,1} \cong \mathcal{C}_{x,1}.$$
		The proof of this main theorem \ref{Thm Main} would occupy most of this chapter, from Section \ref{Section cInd} to \ref{Section projective generator}. The proof relies on three theorems. In Section \ref{Section cInd}, we prove the main theorem modulo the three theorems. And the proofs of the three theorems are given in Sections \ref{Sec Reg Cusp}, \ref{Sec Pf Thm Hom}, \ref{Section projective generator}, respectively.
		
		
		\section{The compact induction induces an equivalence}\label{Section cInd}
		In this section, we prove the Main Theorem \ref{Thm Main} modulo Theorem \ref{Thm SC Red} \ref{Thm Hom} \ref{Thm Proj}.
		
		Let $G$ be a split reductive group scheme over $\mathbb{Z}$, which is simply connected. Let $F$ be a non-archimedean local field, with ring of integers $\mathcal{O}_F$ and residue field $k_F \cong \mathbb{F}_q$ of characteristic $p$. For simplicity, we assume that $q$ is greater than the Coxeter number of $G$ (See Theorem \ref{Thm Broué} for reason).
		
		Let $x$ be a vertex of the Bruhat-Tits building $\mathcal{B}(G, F)$. Let $G_x$ be the parahoric subgroup associated to $x$, and $G_x^+$ be its pro-unipotent radical. Recall that $\overline{G_x}:=G_x/G_x^+$ is a generalized Levi subgroup of $G(k_F)$ with root system $\Phi_x$, see \cite[Theorem 3.17]{rabinoff2003bruhat}. 
		
		Let $\Lambda=\overline{\mathbb{Z}_\ell}$, with $\ell \neq p$. Let $\rho \in \Rep_{\Lambda}(G_x)$ be an irreducible representation of $G_x$, which is trivial on $G_x^+$ and whose reduction to the finite group of Lie type $\overline{G_x}=G_x/G_x^+$ is  
		regular supercuspidal. Here \textbf{regular supercuspidal} (See Definition \ref{Def regular supercuspidal} for precise definition.) means $\rho$ is supercuspidal and lies in a \textbf{regular block} of $\Rep_{\Lambda}(\overline{G_x})$, in the sense of \cite{broue1990isometries}. The reason we want the regularity assumption is that we want to work with a block of $\Rep_{\Lambda}(\overline{G_x})$ which consists purely of supercuspidal representations. See Section \ref{Sec Reg Cusp} for details. We make this a definition for later use.
		
		\begin{definition}
			Let $\rho \in \Rep_{\Lambda}(G_x)$. We say $\rho$ \textbf{has supercuspidal reduction} (resp. \textbf{has regular supercuspidal reduction}), if $\rho$ is trivial on $G_x^+$ and whose reduction to the finite group of Lie type $\overline{G_x}=G_x/G_x^+$ is supercuspidal (resp. regular supercuspidal). Let's denote the reduction of $\rho$ modulo $G_x^+$ by $\overline{\rho} \in \Rep_{\Lambda}(\overline{G_x})$.
		\end{definition}
		
		Let $\mathcal{B}_{x,1}$ be the block of $\Rep_{\Lambda}(G_x)$ containing $\rho$. Let $\mathcal{C}_{x,1}$ be the block of $\Rep_{\Lambda}(G(F))$ containing $\pi:=\cInd_{G_x}^{G(F)}\rho$. Now we can state the Main Theorem of this chapter.
		
		\begin{theorem}[Main Theorem]\label{Thm Main}
			Let $x$ be a vertex of the Bruhat-Tits building $\mathcal{B}(G, F)$. Let $\rho \in \Rep_{\Lambda}(G_x)$ which has regular supercuspidal reduction. Let $\mathcal{B}_{x,1}$ be the block of $\Rep_{\Lambda}(G_x)$ containing $\rho$. Let $\mathcal{C}_{x,1}$ be the block of $\Rep_{\Lambda}(G(F))$ containing $\pi:=\cInd_{G_x}^{G(F)}\rho$. Then the compact induction $\cInd_{G_x}^{G(F)}$ induces an equivalence of categories $\mathcal{B}_{x,1} \cong \mathcal{C}_{x,1}$. 
		\end{theorem}
		
		As mentioned before, the reason we want the regular supercuspidal assumption is the following Theorem. 
		
		\begin{theorem}\label{Thm SC Red}
			Let $\rho \in \Rep_{\Lambda}(G_x)$ be an irreducible representation of $G_x$, which has regular supercuspidal reduction. Let $\mathcal{B}_{x,1}$ be the block of $\Rep_{\Lambda}(G_x)$ containing $\rho$. Then any $\rho' \in \mathcal{B}_{x,1}$ has supercuspidal reduction.
		\end{theorem}
		
		The proof of the Main Theorem \ref{Thm Main} basically splits into two parts -- fully faithfulness and essentially surjectivity. It is convenient to have the following theorem available at an early stage, which implies fully faithfulness immediately and is also used in the proof of essentially surjectivity.
		
		\begin{theorem}\label{Thm Hom}
			Let $x, y$ be two vertices of the Bruhat-Tits building $\mathcal{B}(G, F)$. Let $\rho_1$ be a representation of the parahoric $G_x$ which is trivial on the pro-unipotent radical $G_x^+$. Let $\rho_2$ be a representation of $G_y$ which is trivial on $G_y^+$. Assume one of them has supercuspidal reduction. Then exactly one of the following happens:
			\begin{enumerate}
				\item If there exists an element $g \in G(F)$ such that $g.x=y$, then
				$$\Hom_G(\cInd_{G_x}^{G(F)}\rho_1, \cInd_{G_y}^{G(F)}\rho_2)=\Hom_{G_x}(\rho_1, {^g\rho_2}).$$
				\item If there is no elements $g \in G(F)$ such that $g.x=y$, then
				$$\Hom_G(\cInd_{G_x}^{G(F)}\rho_1, \cInd_{G_y}^{G(F)}\rho_2)=0.$$
			\end{enumerate}
		\end{theorem}
		
		The proof of the above Theorem is basically a computation using Mackey's formula. See Section \ref{Sec Pf Thm Hom}.
		
		\begin{proof}[Proof of Theorem \ref{Thm Main}]
			
			Now we proceed by steps towards our goal: The compact induction $\cInd_{G_x}^{G(F)}$ induces an equivalence of categories $\mathcal{B}_{x,1} \cong \mathcal{C}_{x,1}$. 
			
			First, we show that $\cInd_{G_x}^{G(F)}: \mathcal{B}_{x,1} \to \mathcal{C}_{x,1}$ is well-defined. We need to show that the image of $\mathcal{B}_{x,1}$ under $\cInd_{G_x}^{G(F)}$ lies in $\mathcal{C}_{x,1}$. By Theorem \ref{Thm SC Red} and Theorem \ref{Thm Hom} above, $$\cInd_{G_x}^{G(F)}|_{\mathcal{B}_{x,1}}: \mathcal{B}_{x,1} \to \Rep_{\Lambda}(G(F))$$
			is fully faithful (See Lemma \ref{Lem Thm Hom implies fully faithful}, note here we used Theorem \ref{Thm SC Red} that any representation in $\mathcal{B}_{x,1}$ has supercuspidal reduction, so that we can apply Theorem \ref{Thm Hom}), hence an equivalence onto the essential image. Since $\mathcal{B}_{x,1}$ is indecomposable as an abelian category, so is its essential image (See Lemma \ref{Lem Indec}), hence its essential image is contained in a single block of $\Rep_{\Lambda}(G(F))$. But such a block must be $\mathcal{C}_{x,1}$ since $\cInd_{G_x}^{G(F)}$ maps $\rho$ to $\pi \in \mathcal{C}_{x,1}$. Therefore, $\cInd_{G_x}^{G(F)}: \mathcal{B}_{x,1} \to \mathcal{C}_{x,1}$ is well-defined.
			
			Second, we show that $\cInd_{G_x}^{G(F)}: \mathcal{B}_{x,1} \to \mathcal{C}_{x,1}$ is fully faithful. This is already noticed in the proof of ``well-defined" in the last paragraph. Indeed, 
			$$\Hom_G(\cInd_{G_x}^{G(F)}\rho_1, \cInd_{G_x}^{G(F)}\rho_2)=\Hom_{G_x}(\rho_1, \rho_2)$$
			by Theorem \ref{Thm SC Red} and Theorem \ref{Thm Hom} (See Lemma \ref{Lem Thm Hom implies fully faithful}.). Therefore, $\cInd_{G_x}^{G(F)}: \mathcal{B}_{x,1} \to \mathcal{C}_{x,1}$ is fully faithful.
			
			Finally, we show that $\cInd_{G_x}^{G(F)}: \mathcal{B}_{x,1} \to \mathcal{C}_{x,1}$ is essentially surjective. This will occupy the rest of this section. 
			
			The idea is to find a projective generator of $\mathcal{C}_{x,1}$ and show that it is in the essential image. Fix a vertex $x$ of the Bruhat-Tits building $\mathcal{B}(G, F)$ as before. Let $V$ be the set of equivalence classes of vertices of the Bruhat-Tits building $\mathcal{B}(G, F)$ up to $G(F)$-action. For $y \in V$, let $\sigma_y:=\cInd_{G_y^+}^{G_y}\Lambda$. Let $\Pi:=\bigoplus_{y \in V}\Pi_y$ where $\Pi_y:=\cInd_{G_y^+}^{G(F)}\Lambda$. Then $\Pi$ is a projective generator of the category of depth-zero representations $\Rep_{\Lambda}(G(F))_0$, see \cite[Appendix]{dat2009finitude}. Let $\sigma_{x,1}:=(\sigma_x)|_{\mathcal{B}_{x,1}} \in \mathcal{B}_{x,1} \xhookrightarrow{summand} \Rep_{\Lambda}(G_x)$ be the $\mathcal{B}_{x,1}$-summand of $\sigma_x$. And let $\Pi_{x,1}:=\cInd_{G_x}^{G(F)}\sigma_{x,1}$. Note $\Pi_{x,1}$ is a summand of $\Pi_x=\cInd_{G_x}^{G(F)}\sigma_x$, hence a summand of $\Pi$. Using Theorem \ref{Thm Hom}, one can show that the rest of the summands of $\Pi$ don't interfere with $\Pi_{x,1}$ (See Lemma \ref{Lem Ortho} and Lemma \ref{Lem Gen} for precise meaning), hence $\Pi_{x,1}$ is a projective generator of $\mathcal{C}_{x,1}$. Let us state it as a Theorem, see Section 2 for details.
			
			\begin{theorem}\label{Thm Proj}
				$\Pi_{x,1}=\cInd_{G_x}^{G(F)}\sigma_{x,1}$ is a projective generator of $\mathcal{C}_{x,1}$.
			\end{theorem}
			
			Now we've found a projective generator $\Pi_{x,1}=\cInd_{G_x}^{G(F)}\sigma_{x,1}$ of $\mathcal{C}_{x,1}$, and it is clear that $\Pi_{x,1}$ is in the essential image of $\cInd_{G_x}^{G(F)}$. We now deduce from this that $\cInd_{G_x}^{G(F)}: \mathcal{B}_{x,1} \to \mathcal{C}_{x,1}$ is essentially surjective. Indeed, for any $\pi' \in \mathcal{C}_{x,1}$, we can resolve $\pi'$ by some copies of $\Pi_{x,1}$:
			$$\Pi_{x,1}^{\oplus I} \xrightarrow{f} \Pi_{x,1}^{\oplus J} \to \pi' \to 0.$$
			Using Theorem \ref{Thm Hom} and $\cInd_{G_x}^{G(F)}$ commutes with arbitrary direct sums (See Lemma \ref{Lem Sum}) we see that $f \in \Hom_G(\Pi_{x,1}^{\oplus I}, \Pi_{x,1}^{\oplus J})$ comes from a morphism $g \in \Hom_{G_x}(\sigma_{x,1}^{\oplus I}, \sigma_{x,1}^{\oplus J})$. Using $\cInd_{G_x}^{G(F)}$ is exact we see that $\pi'$ is the image of $\coker(g) \in \mathcal{B}_{x,1}$ under $\cInd_{G_x}^{G(F)}$. Therefore, $\cInd_{G_x}^{G(F)}: \mathcal{B}_{x,1} \to \mathcal{C}_{x,1}$ is essentially surjective.
			
		\end{proof}
		
		\begin{lemma}\label{Lem Thm Hom implies fully faithful}
			$\cInd_{G_x}^{G(F)}|_{\mathcal{B}_{x,1}}: \mathcal{B}_{x,1} \to \Rep_{\Lambda}(G(F))$ is fully faithful.
		\end{lemma}
		
		\begin{proof}
			Let $\rho_1, \rho_2 \in \mathcal{B}_{x,1}$. By the regular supercuspidal assumption and Theorem \ref{Thm SC Red}, $\rho_1, \rho_2$ has supercuspidal reduction. Hence the assumption of Theorem \ref{Thm Hom} is satisfied and we compute using the first case of Theorem \ref{Thm Hom} that
			$$\Hom_G(\cInd_{G_x}^{G(F)}\rho_1, \cInd_{G_x}^{G(F)}\rho_2) \cong \Hom_{G_x}(\rho_1, \rho_2).$$
			In other words, $\cInd_{G_x}^{G(F)}|_{\mathcal{B}_{x,1}}: \mathcal{B}_{x,1} \to \Rep_{\Lambda}(G(F))$ is fully faithful.
		\end{proof}
		
		\begin{lemma}\label{Lem Indec}
			The image of $\mathcal{B}_{x,1}$ under $\cInd_{G_x}^{G(F)}$ is indecomposable as an abelian category.
		\end{lemma}
		
		\begin{proof}
			The point is that $\cInd_{G_x}^{G(F)}|_{\mathcal{B}_{x,1}}: \mathcal{B}_{x,1} \to \Rep_{\Lambda}(G(F))$ is not only fully faithful, i.e., an equivalence of categories onto the essential image, but also an equivalence of \textbf{abelian} categories onto the essential image. Indeed, it suffices to show that $\cInd_{G_x}^{G(F)}|_{\mathcal{B}_{x,1}}: \mathcal{B}_{x,1} \to \Rep_{\Lambda}(G(F))$ preserves kernels, cokernels, and finite (bi-)products. But this follows from the next Lemma \ref{Lem Sum}.
			
			Assume otherwise that the essential image of $\mathcal{B}_{x,1}$ under $\cInd_{G_x}^{G(F)}$ is decomposable, then so is $\mathcal{B}_{x,1}$. But $\mathcal{B}_{x,1}$ is a block, hence indecomposable, contradiction!
		\end{proof}
		
		\begin{lemma}\label{Lem Sum}
			$\cInd_{G_x}^{G(F)}$ is exact and commutes with arbitrary direct sums.
		\end{lemma}
		
		\begin{proof}
			For $\cInd_{G_x}^{G(F)}$ is exact, we refer to \cite[I.5.10]{vigneras1996representations}.
			
			We show that $\cInd_{G_x}^{G(F)}$ commutes with arbitrary direct sums. Indeed, $\cInd_{G_x}^{G(F)}$ is a left adjoint (See \cite[I.5.7]{vigneras1996representations}), hence commutes with arbitrary colimits. In particular, it commutes with arbitrary direct sums.
		\end{proof}
		
		
		
		
		\section{Regular supercuspidal blocks for finite groups of Lie type}\label{Sec Reg Cusp}
		
		In this section, we prove Theorem \ref{Thm SC Red}. As mentioned before, we made the \textbf{regular} assumption in order that the conclusion of Theorem \ref{Thm SC Red} -- all representations in such a block have supercuspidal reduction -- is true. So the readers are welcome to skip this section for a first reading and pretend that we begin with a block in which all representations have supercuspidal reduction.
		
		The main body of this section is to define regular supercuspidal blocks with $\Lambda=\overline{\mathbb{Z}_{\ell}}$-coefficients of a finite group of Lie type, and to show that a regular supercuspidal block consists purely of supercuspidal representations.
		
		Let $\Lambda:=\overline{\mathbb{Z}_{\ell}}$ be the coefficients of representations. Fix a prime number $p$. Let $\ell$ be a prime number different from $p$. Let $q$ be a power of $p$.
		
		\begin{definition}[{\cite[I.4.1]{vigneras1996representations}}]
			\begin{enumerate}Let $\Lambda'$ be any ring.
				\item Let $H$ be a profinite group, a \textbf{representation of $H$ with $\Lambda'$-coefficients} $(\pi, V)$ is a $\Lambda'$-module $V$, together with a $H$-action $\pi: H \to GL_{\Lambda'}(V)$.
				\item A representation of $H$ with $\Lambda'$-coefficients is called \textbf{smooth} if for any $v \in V$, the stabilizer $Stab_H(v) \subseteq H$ is open.
			\end{enumerate}
		\end{definition}
		
		From now on, all representations are assumed to be smooth. The category of smooth representations of $H$ with $\Lambda'$-coefficients is denoted by $\Rep_{\Lambda'}(H)$.
		
		\subsection{Regular blocks}
		
		\textbf{The following notations are used in this subsection only.} Let $\mathcal{G}$ be a split reductive group scheme over $\mathbb{Z}$. Let $\mathbb{G}:=\mathcal{G}(\overline{\mathbb{F}_q})$, $G:=\mathbb{G}^F=\mathcal{G}(\mathbb{F}_q)$, where $F$ is the Frobenius. By abuse of notation, we sometimes identify the group scheme $\mathcal{G}_{\overline{\mathbb{F}_q}}$ with its $\overline{\mathbb{F}_q}$-points $\mathbb{G}$. Let $\mathbb{G}^*$ be the dual group (over $\overline{\mathbb{F}_q}$) of $\mathbb{G}$, and $F^*$ the dual Frobenius (See \cite[Section 4.2]{carter1985finite}). Fix an isomorphism $\overline{\mathbb{Q}_{\ell}} \cong \mathbb{C}$. 
		
		The definition of regular supercuspidal blocks and regular supercuspidal representations of a finite group of Lie type $\Gamma$ involves modular Deligne-Lusztig theory and block theory. We refer to \cite{deligne1976representations}, \cite{carter1985finite}, and \cite{digne2020representations} for Deligne-Lusztig theory, \cite{michel1989bloc} and \cite{broue1990isometries} for modular Deligne-Lusztig theory, and \cite[Appendix B]{bonnafe2010representations} for generalities on blocks. 
		
		First, let us recall a result in Deligne-Lusztig theory (See \cite[Proposition 11.1.5]{digne2020representations}). 
		
		\begin{proposition}\label{Prop dual torus}
			The set of $\mathbb{G}^F$-conjugacy classes of pairs $(\mathbb{T}, \theta)$, where  $\mathbb{T}$ is a $F$-stable maximal torus of  $\mathbb{G}$ and $\theta \in \widehat{\mathbb{T}^F}$, is in non-canonical bijection to the set of $\mathbb{G^*}^{F^*}$-conjugacy classes of pairs $(\mathbb{T}^*, s)$, where $s$ is a semisimple element of $\mathbb{G}^*$ and $\mathbb{T}^*$ is a $F^*$-stable maximal torus of $\mathbb{G}^*$ such that $s \in {\mathbb{T}^*}^{F^*}$.  Moreover, we could and will fix a compatible system of isomorphisms $\mathbb{F}_{q^n}^* \cong \mathbb{Z}/(q^n-1)\mathbb{Z}$ to pin down this bijection.
		\end{proposition}
		
		Now let $s$ be a \textbf{strongly regular semisimple} 
		%(\textcolor{red}{Is this the standard terminology?}) 
		element of $G^*={\mathbb{G}^*}^{F^*}$ (note we require $s$ to be fixed by $F^*$ here), i.e., the centralizer $C_{\mathbb{G}^*}(s)$ is a $F^*$-stable maximal torus, denoted $\mathbb{T}^*$. Let $\mathbb{T}$ be the dual torus of $\mathbb{T}^*$. Let $T=\mathbb{T}^F$ and $T^*={\mathbb{T}^*}^{F^*}$. Let $T_\ell$ denote the $\ell$-part of $T$.
		
		Recall for $s$ strongly regular semisimple, the (rational) Lusztig series $\mathcal{E}(G, (s))$ consists of only one element, namely, $\pm R_T^G(\hat{s})$, where $\hat{s}=\theta$ is such that $(\mathbb{T}, \theta)$ corresponds to $(\mathbb{T}^*, s)$ via the previous bijection in Proposition \ref{Prop dual torus}. Here and after the sign $\pm$ is taken such that it is an honest representation (See \cite[Section 7.5]{carter1985finite}).
		%	(This follows from, for example, Broué's equivalence. See Theorem \ref{Thm Broué} below.
		%	% Better explanation?
		%	)
		
		\textbf{From now on, assume moreover that $s \in {\mathbb{G}^*}^{F^*}$ has order prime to $\ell$.} In other words, assume $s \in G^*={\mathbb{G}^*}^{F^*}$ is a \textbf{strongly regular semisimple $\ell'$-element}. We are going to define regular blocks, we refer to \cite[Appendix B]{bonnafe2010representations} for generalities on blocks.
		
		Define the \textbf{$\ell$-Lusztig series} 
		$$\mathcal{E}_\ell(G, (s)):=\{\pm R_T^G(\hat{s}\eta)\;|\; \eta \in \widehat{T_\ell}\}.$$ Note the notation $\mathcal{E}_\ell(T, (s))$ also makes sense by putting $G=T$.
		
		By \cite{michel1989bloc}, $\mathcal{E}_\ell(G, (s))$ is a union of $\ell$-blocks of $\Rep_{\overline{\mathbb{Q}_\ell}}(G)$. Such a block (or more precisely, a union of blocks) is called a \textbf{($\ell$-)regular block}. Let $e_s^G \in \overline{\mathbb{Z}_\ell}G$ denote the corresponding central idempotent. Note $e_s^T$ also makes sense by putting $G=T$. We shall see later that a regular block is indeed a block, i.e., indecomposible. (This follows from, for example, Broué's equivalence. See Theorem \ref{Thm Broué} below.)
		
		\begin{definition}[Regular blocks]\label{Def Regular Block}
			Let $s \in G^*={\mathbb{G}^*}^{F^*}$ be a strongly regular semisimple $\ell'$-element.
			We call the block $\overline{\mathbb{Z}_\ell}Ge_s^G$ of the group algebra $\overline{\mathbb{Z}_\ell}G$ corresponding to the central idempotent $e_s^G$ the \textbf{regular block} associated to $s$. Let $\mathcal{A}_s:=\overline{\mathbb{Z}_\ell}Ge_s^G\Modl$ be the corresponding category of modules, this is also referred to as a regular block, by abuse of notation.
			
%			Similarly, the block $\overline{\mathbb{F}_\ell}Ge_s^G$ is called a $\overline{\mathbb{F}_{\ell}}$-block. (However, this notion won't be used later.)
			
		\end{definition}
		
%		\begin{remark}
%			Above all, ``a block" could have three different meanings: $\ell$-block, $\overline{\mathbb{Z}_{\ell}}$-block, and $\overline{\mathbb{F}_{\ell}}$-block. But they are in one-one correspondence to each other, so we often abuse the notation and simply call it ``a block".
%		\end{remark}
		
		
		
%		\begin{remark}
%			We will see later in Theorem \ref{Theorem Pure Cuspidality} that a regular supercuspidal block consists only of supercuspidal representations. In the end, the above definition is equivalent to requiring the torus $\mathbb{T}^F$ to be elliptic, i.e., not contained in any proper parabolic subgroup of $\mathbb{G}^F$ (See Lemma \ref{Lemma Q_l-bar cuspidal}). This is because $\overline{\mathbb{Z}_{\ell}}$-cuspidality could be checked over $\overline{\mathbb{Q}_{\ell}}$ (See the proof of Theorem \ref{Theorem Pure Cuspidality}).
%		\end{remark}
		
		Thanks to \cite{broue1990isometries}, we understand the category $\mathcal{A}_s=\overline{\mathbb{Z}_\ell}Ge_s^G\Modl$ quite well. Roughly speaking, it is equivalent to the category of representations of a torus, via Deligne-Lusztig induction. This is what we are going to explain now.
		
		Let $\mathbb{B} \subseteq \mathbb{G}$ be a Borel subgroup containing our torus $\mathbb{T}$, let $\mathbb{U}$ be the unipotent radical of $\mathbb{B}$. Let $X_{\mathbb{U}}$ be the Deligne-Lusztig variety defined by
		$$X_{\mathbb{U}}:=\{g \in \mathbb{G} \;|\; g^{-1}F(g) \in \mathbb{U}\}.$$
		
		The main result of \cite{broue1990isometries} is the following: The Deligne-Lusztig induction 
		$$\pm R_T^G: \overline{\mathbb{Z}_\ell}T\Modl \to \overline{\mathbb{Z}_\ell}G\Modl$$ induces an equivalence of categories between $\overline{\mathbb{Z}_\ell}Te_s^T\Modl$ and $\overline{\mathbb{Z}_\ell}Ge_s^G\Modl$. In particular, one deduce that the irreducible objects in $\overline{\mathbb{F}_\ell}Ge_s^G\Modl$ lifts to $\overline{\mathbb{Z}_\ell}$. More precisely, let us state it as the following theorem.
		
		\begin{theorem}[Broué's equivalence, {\cite[Theorem 3.3]{broue1990isometries}}]\label{Thm Broué}
			With the previous assumptions and notations, assume $X_{\mathbb{U}}$ is affine of dimension $d$ (which is the case if $q$ is greater than the Coxeter number of $\mathbb{G}$.). The cohomology complex $R\Gamma_c(X_{\mathbb{U}}, \overline{\mathbb{Z}_\ell})=R\Gamma_c(X_{\mathbb{U}}, {\mathbb{Z}_\ell}) \otimes_{\mathbb{Z}_\ell}$$\overline{\mathbb{Z}_\ell}$ is concentrated in degree $d=dimX_{\mathbb{U}}$. And the $(\overline{\mathbb{Z}_\ell}Ge_s^G, \overline{\mathbb{Z}_\ell}Te_s^T)$-bimodule $e_s^GH_c^d(X_{\mathbb{U}}, \overline{\mathbb{Z}_\ell})e_s^T$ induces an equivalence of categories
			$$e_s^GH_c^d(X_{\mathbb{U}}, \overline{\mathbb{Z}_\ell})e_s^T \otimes_{\overline{\mathbb{Z}_\ell}Te_s^T}-: \overline{\mathbb{Z}_\ell}Te_s^T\Modl \to \overline{\mathbb{Z}_\ell}Ge_s^G\Modl.$$
		\end{theorem}
		
		\textbf{From now on, we assume the above Theorem holds for all finite groups of Lie type we encountered in this paper.} we hope this is not a severe restriction. This is the case at least when $q$ is greater than the Coxeter number of $\mathbb{G}$.
		
		Note also that the category $\overline{\mathbb{Z}_\ell}Te_s^T\Modl$ is equivalent to the category $\overline{\mathbb{Z}_\ell}T_{\ell}\Modl$, where $T_{\ell}$ is the order-$\ell$-part of $T$, this is essentially the category of representations of some product of $\mathbb{Z}/\ell^{k_i}\mathbb{Z}$. In particular, it has a unique irreducible representation (simple object), which is already defined over $\overline{\mathbb{F}_{\ell}}$. Let us denote its corresponding character by $\theta_s: T \to \overline{\mathbb{F}_{\ell}}^*$. Accordingly, $\overline{\mathbb{Z}_\ell}Ge_s^G\Modl$ has a unique simple object $\pm R_T^G(\theta_s)$.
		
%		We now define regular supercuspidal representations as those representations that occur in some regular cuspidal block. The term ``cuspidal" in the name ``regular cuspidal" shall be justified later by Theorem \ref{Theorem Pure Cuspidality}.
		

		\subsection{Regular supercuspidal blocks}
		
		Let us first recall the definition of supercuspidal representations.
		
		\begin{definition}\label{Def supercuspidal}
			
			\begin{enumerate}
				\item An irreducible representation is called \textbf{supercuspidal} if it does not occur as a subquotient of any proper parabolic induction.
				\item A representation is called \textbf{supercuspidal} if all its irreducible subquotients are supercuspidal.
			\end{enumerate}
		\end{definition}
		
		Now let us define regular supercuspidal blocks and regular supercuspidal representations.
		
		\begin{definition}\label{Definition regular supercuspidal block}
			%			By a \textbf{regular cuspidal block}, we mean a regular block which contains a cuspidal representation.
			By a \textbf{regular supercuspidal block}, we mean a regular block $\mathcal{A}_s$ whose unique simple object $\pm R_T^G(\theta_s)$ (See the explanations after Theorem \ref{Thm Broué} for definition) is supercuspidal.
		\end{definition}
		
		\begin{definition}\label{Def regular supercuspidal}
			
			\begin{enumerate}
				\item An irreducible representation is called \textbf{regular supercuspidal} if it lies in a regular supercuspidal block.
				\item A representation is called \textbf{regular supercuspidal} if all its irreducible subquotients are regular supercuspidal.
			\end{enumerate}
			%			Let $G$ be a finite group of Lie type. Let $\Lambda=\overline{\mathbb{Z}_{\ell}}$. Let $\rho \in \Rep_{\Lambda}(G)$. Then $\rho$ is called \textbf{regular supercuspidal} if each of its irreducible subquotient $\rho_i$ is cuspidal (See Definition \ref{Def Cuspidal}) and lies in a regular supercuspidal $\overline{\mathbb{Z}_{\ell}}$-block $\mathcal{A}_{s_i}$ of $G$.
		\end{definition}
		
        It is clear from the definitions that we have the following proposition.
		
		\begin{proposition}\label{Theorem Pure SC}
			Let $\mathcal{A}_s$ be a regular supercuspidal block. Then any representation in this block is supercuspidal.
		\end{proposition}
		
		\begin{proof}
			By definition of supercuspidality, it suffices to check that any irreducible representation in this block is supercuspidal. But as we noted before in the explanations after Theorem \ref{Thm Broué}, $\mathcal{A}_s$ has only one irreducible representation -- $\pm R_T^G(\theta_s)$, which we assumed to be supercuspidal in the definition of regular supercuspidal block. So we win!
		\end{proof}
		
		
		
		
		
%		\subsection{Pure Cuspidality}
%		
%		\subsubsection{A digression on cuspidality}
%		
%		Before stating the theorem of pure cuspidality, let us define cuspidality for representations with arbitrary coefficients. Let $\Lambda'$ be any ring. For example, $\Lambda'$ can be $\overline{\mathbb{Q}_{\ell}}$, $\overline{\mathbb{Z}_{\ell}}$, or $\overline{\mathbb{F}_{\ell}}$.
%		
%		First, we define two functors.
%		
%		\begin{definition}[Parabolic induction and restriction] 
%			Let $G$ be a finite group of Lie type. Let $P$ be a parabolic subgroup and $M$ the corresponding Levi subgroup.
%			\begin{enumerate}
%				\item The \textbf{parabolic induction functor} is defined to be the composition 
%				$$i_M^G:= \Ind_P^G \circ f^*,$$ where 
%				$$f^*: \Rep_{\Lambda'}(M) \to \Rep_{\Lambda'}(P)$$
%				is the inflation along the natural projection $f: P \to M$. 
%				\item The \textbf{parabolic restriction functor} is defined to be the composition 
%				$$r_M^G:= (-)_U \circ \Res_P^G,$$ where 
%				$$(-)_U: \Rep_{\Lambda'}(P) \to \Rep_{\Lambda'}(M), V \mapsto V/\left<\{u.v-v | u \in U, v \in V\}\right>_{\Lambda'U\Modl}$$
%				is the functor of taking coinvariance.
%			\end{enumerate}
%		\end{definition}
		
%		We recall that $r_M^G$ is left adjoint to $i_M^G$ and they are both exact under our assumption $\ell \neq p$ (See \cite[II.2.1]{vigneras1996representations}).
%		
%		\begin{definition}[Cuspidal]\label{Def Cuspidal}
%			Let $G$ be a finite group of Lie type. Let $\rho \in \Rep_{\Lambda'}(G)$ be a representation of $G$. Then $\rho$ is called \textbf{($\Lambda'$-)cuspidal} if $\rho$ is not a subrepresentation of any proper parabolic induction, i.e., 
%			$$\Hom_{G}(\rho, i_P^G(\sigma))=0$$ 
%			for any proper parabolic subgroup $P$ of $G$ and any representation $\sigma \in \Rep_{\Lambda'}(M)$, where $M$ is the Levi subgroup corresponding to $P$.
%		\end{definition}
%		
%		For example, let $s \in G^*$ strongly regular semisimple and $T$ is not contained in any proper parabolic subgroup, then 
%		$$\pm R_T^G(\hat{s})$$ 
%		is cuspidal in $\Rep_{\overline{\mathbb{Q}_{\ell}}}(G)$ (See \cite[Theorem 8.3]{deligne1976representations}). 
		%	Moreover, reduction modulo $\ell$ preserves cuspidality (\textcolor{red}{See ?}), hence each irreducible component of the reduction $$r_{\ell}(R_T^G(\hat{s})):=R\Gamma_c(X_{\mathbb{U}}, \overline{\mathbb{F}_\ell})\otimes \hat{s}$$
		%	is cuspidal in $\Rep_{\overline{\mathbb{F}_{\ell}}}(G)$. 
		
%		we record the following equivalent definition of cuspidality for later use.
%		
%		\begin{lemma}\cite[II.2.3]{vigneras1996representations}\label{Lemma Cuspidal}
%			$\rho \in \Rep_{\Lambda'}(G)$ is cuspidal if and only if $r_M^G\rho=0$, for any proper Levi subgroup $M$ of $G$.
%		\end{lemma}
		
		%	Can cuspidality be checked on irreducible subquotients? Let's define subquotient first.
		%	
		%	\begin{definition}[subquotient]
			%		Let $\pi$ be a representation of $G$, a \textbf{subquotient} of $\pi$ is a representation of the form $\pi_1/\pi_2$ for some chain of subrepresentations $\pi_2 \subseteq \pi_1 \subseteq \pi$.
			%	\end{definition}
		
		%	\begin{question}
			%		\textcolor{red}{Let $(\pi, V) \in \Rep_{\Lambda}(G(F))$. If all irreducible subquotients of $\pi$ are cuspidal, is $\pi$ cuspidal?}
			%	\end{question}
		
		
		
		
%		\subsubsection{The theorem of pure cuspidality}
%		
%		We can now state the theorem of pure cuspidality. 
%		
%		As in Broué's paper \cite{broue1990isometries}, we fix a finite integral extension $\mathcal{O}$ of $\mathbb{Z}_{\ell}$, which is big enough. One good thing to work with $\mathcal{O}$ instead of $\overline{\mathbb{Z}_{\ell}}$ is that $\mathcal{O}$ is a discrete valuation ring, while $\overline{\mathbb{Z}_{\ell}}$ is not (even not Noetherian). We assume $\mathcal{O}$ to be big enough (for example, $\mathcal{O}$ contains all roots of unity we encounter) so that all things we need to do representation theory are available without change.
%		
%		\begin{theorem}[Pure Cuspidality]\label{Theorem Pure Cuspidality}
%			Let $G$ be a finite group of Lie type. Let $s \in G^*=\mathbb{G^*}^{F^*}$ be a strongly regular semisimple $\ell'$-element, with corresponding torus $T=\mathbb{T}^F$ and character $\hat{s} \in \hat{T}$ as in Proposition \ref{Prop dual torus}. Assume that $\pm R_T^G(\hat{s})$ is $\overline{\mathbb{Q}_{\ell}}$-cuspildal. Then the $\overline{\mathbb{Z}_{\ell}}$-block $\mathcal{A}_s=\overline{\mathbb{Z}_{\ell}}Ge_s^G\Modl$ consists purely of cuspidal representations.
%		\end{theorem}
%		
%		\begin{proof}
%			%		\textcolor{red}{Minor technical issue: Broué's paper works with a Dedekind ring $\mathcal{O}$, but $\overline{\mathbb{Z}_{\ell}}$ is not Dedekind. So you need to be careful about the results cited from Broué's paper, for example, Lemma 3.4 of Broué's paper.}
%			%    	Let $V:=\overline{\mathbb{Z}_{\ell}}Ge_s^G \in \overline{\mathbb{Z}_{\ell}}G\Modl=\Rep_{\overline{\mathbb{Z}_{\ell}}}(G)$. Let's first show that $V$ is $\overline{\mathbb{Z}_{\ell}}$-cuspidal.
%			
%			Recall Broué's equivalence \ref{Thm Broué}: For $\mathcal{O}$ a finite integral extension of $\mathbb{Z}_{\ell}$, big enough, we have
%			$$F:=e_s^GH^d_c(X_{\mathbb{U}}, \mathcal{O})e_s^T\otimes_{\mathcal{O}Te_s^T}-: \mathcal{O}Te_s^T\Modl \to \mathcal{O}Ge_s^G\Modl$$ is an equivalence of categories. This is moreover an equivalence of abelian categories (See Lemma \ref{Lem abelian}). Let $V:=F(\mathcal{O}Te_s^T)=e_s^GH^d_c(X_{\mathbb{U}}, \mathcal{O})e_s^T$ %(\textcolor{red}{$=\mathcal{O}Ge_s^G$?} \textcolor{blue}{No, not true. Otherwise this is contained in the Harish-Chandler induction.})
%			. Then $V$ is a projective generator of $\mathcal{A}_s$, since $\mathcal{O}Te_s^T$ is a projective generator of $\mathcal{O}Te_s^T\Modl$. We first show that $V$ is $\mathcal{O}$-cuspidal.
%			
%			By classical Deligne-Lusztig theory, $\overline{\mathbb{Q}_{\ell}}V \cong \bigoplus_{\eta \in \hat{T_{\ell}}}\pm R_T^G(\hat{s}\eta)$ 
%			%(\textcolor{red}{Is this right?} \textcolor{blue}{Yes.}) 
%			is $\overline{\mathbb{Q}_{\ell}}$-cuspidal (For details, see Lemma \ref{Lem Q_l-bar cuspidal} below.). 
%			%    In other words, 
%			%    $$dim_{\overline{\mathbb{Q}_{\ell}}}\overline{\mathbb{Q}_{\ell}}V=dim_{\overline{\mathbb{Q}_{\ell}}}(\overline{\mathbb{Q}_{\ell}}V)(U)=dim_{\overline{\mathbb{Q}_{\ell}}}\overline{\mathbb{Q}_{\ell}}V(U).$$ 
%			%    Since $V$ is free $\overline{\mathbb{Z}_{\ell}}$-module, we thus have
%			%    $$rank_{\overline{\mathbb{Z}_{\ell}}}V=rank_{\overline{\mathbb{Z}_{\ell}}}V(U).$$
%			In other words, 
%			$$r^G_{M, \overline{\mathbb{Q}_{\ell}}}(\overline{\mathbb{Q}_{\ell}}V):=\overline{\mathbb{Q}_{\ell}}V/\left<\{u.v-v | u \in U, v \in \overline{\mathbb{Q}_{\ell}}V\}\right>_{\overline{\mathbb{Q}_{\ell}}U\Modl}=0.$$
%			However, note  
%			$$\left<\{u.v-v | u \in U, v \in \overline{\mathbb{Q}_{\ell}}V\}\right>_{\overline{\mathbb{Q}_{\ell}}U\Modl}=\left<\{u.v-v | u \in U, v \in \overline{\mathbb{Q}_{\ell}}V\}\right>_{\mathcal{O}U\Modl}.$$
%			So we have 
%			$$r^G_{M, \mathcal{O}}(\overline{\mathbb{Q}_{\ell}}V):=\overline{\mathbb{Q}_{\ell}}V/\left<\{u.v-v | u \in U, v \in \overline{\mathbb{Q}_{\ell}}V\}\right>_{\mathcal{O}U\Modl}=0.$$
%			
%			Note $V$ is finitely presented and projective over $\mathcal{O}Te_s^T$ (See \cite[Proof of Theorem 3.3]{broue1990isometries}), hence projective over $\mathcal{O}$ (because the restriction functor $\mathcal{O}T\Modl \to \mathcal{O}\Modl$ preserves projectivity, since it's left adjoint to an exact functor, the induction functor), which is a local ring 
%			%(\textcolor{blue}{See Lemma 3 from last manuscript ``Week 24-25"})
%			, hence $V$ is free over $\mathcal{O}$ (See \cite[Theorem 24.4.5]{vakil2017rising}). We thus have an inclusion
%			$$V \xhookrightarrow[]{} \overline{\mathbb{Q}_{\ell}}V:=\overline{\mathbb{Q}_{\ell}}\otimes_{\mathcal{O}}V %\text{(\textcolor{red}{Is this true?} \textcolor{blue}{Yes.})}
%			$$
%			as $\mathcal{O}G$-modules.
%			Recall that the parabolic restriction $r^G_{M, \mathcal{O}}$ is exact (See \cite[II.2.1]{vigneras1996representations}), hence 
%			$$r^G_{M, \mathcal{O}}(\overline{\mathbb{Q}_{\ell}}V)=0$$
%			implies that 
%			$$r^G_{M, \mathcal{O}}(V)=0,$$
%			i.e., $V$ is $\mathcal{O}$-cuspidal. 
%			%(\textcolor{red}{Did we use $U$ pro-$p$ somewhere?} \textcolor{blue}{No. But one can also argue using invariance instead of coinvariance, and for ``invariance = coinvariance" we need $U$ to be a $p$-group.})
%			
%			Moreover, base change to $\overline{\mathbb{Z}_{\ell}}$ we see that $\overline{\mathbb{Z}_{\ell}}V$ is $\overline{\mathbb{Z}_{\ell}}$-cuspidal. Indeed, 
%			$$r^G_{M, \overline{\mathbb{Z}_{\ell}}}(\overline{\mathbb{Z}_{\ell}}V)=\overline{\mathbb{Z}_{\ell}}V/\overline{\mathbb{Z}_{\ell}}V(U)=\overline{\mathbb{Z}_{\ell}}\otimes_{\mathcal{O}}(V/V(U))=\overline{\mathbb{Z}_{\ell}}\otimes_{\mathcal{O}}r^G_{M, \mathcal{O}}(V)=0.$$
%			
%			For general $V' \in \mathcal{A}_s$, we can resolve it by some direct sum of $V$'s, and we see that
%			$$r^G_{M, \overline{\mathbb{Z}_{\ell}}}(V')=0,$$
%			(using $r^G_{M, \overline{\mathbb{Z}_{\ell}}}$ is exact and commutes with arbitrary direct sum) i.e., $V'$ is $\overline{\mathbb{Z}_{\ell}}$-cuspidal.
%		\end{proof}
%		
		
		
		
		
%		\begin{lemma}\label{Lem abelian}
%			$$F:=e_s^GH^d_c(X_{\mathbb{U}}, \mathcal{O})e_s^T\otimes_{\mathcal{O}Te_s^T}-: \mathcal{O}Te_s^T\Modl \to \mathcal{O}Ge_s^G\Modl$$ is an equivalence of abelian categories.
%		\end{lemma}
%		
%		\begin{proof}
%			We already know that $F$ is an equivalence of categories. It remains to show that $F$ is exact and commutes with product.
%			
%			Now $e_s^GH^d_c(X_{\mathbb{U}}, \mathcal{O})e_s^T$ is projective over ${\mathcal{O}Te_s^T}$ (See \cite[Proof of Theorem 3.3]{broue1990isometries}), hence flat over ${\mathcal{O}Te_s^T}$. Hence $F:=e_s^GH^d_c(X_{\mathbb{U}}, \mathcal{O})e_s^T\otimes_{\mathcal{O}Te_s^T}-$ is exact.
%			
%			It is clear that $F:=e_s^GH^d_c(X_{\mathbb{U}}, \mathcal{O})e_s^T\otimes_{\mathcal{O}Te_s^T}-$ commutes with product.
%		\end{proof}
		
%		\begin{lemma}\label{Lem Q_l-bar cuspidal}\label{Lemma Q_l-bar cuspidal}
%			Let $G$ be a finite group of Lie type. Let $s \in G^*=\mathbb{G^*}^{F^*}$ be a strongly regular semisimple $\ell'$-element, with corresponding torus $T=\mathbb{T}^F$ and character $\hat{s} \in \hat{T}$ as before. Assume that $\pm R_T^G(\hat{s})$ is $\overline{\mathbb{Q}_{\ell}}$-cuspildal. Then $\pm R_T^G(\hat{s}\eta)$ is $\overline{\mathbb{Q}_{\ell}}$-cuspidal for any $\eta \in \hat{T_{\ell}}$.
%		\end{lemma}
%		
%		\begin{proof}
%			Broué's equivalence \ref{Thm Broué} makes sure that $\pm R_T^G(\hat{s}\eta)$ is irreducible, hence $\hat{s}\eta$ is in general position. And $\pm R_T^G(\hat{s})$ being cuspidal implies that $T$ is not contained in any proper parabolic subgroup, hence $\pm R_T^G(\hat{s}\eta)$ is $\overline{\mathbb{Q}_{\ell}}$-cuspidal for any $\eta \in \hat{T_{\ell}}$ (See \cite[Theorem 9.3.2]{carter1985finite}).
%		\end{proof}


        
		
		
		
		
		
		
		
    \subsection{Proof of Theorem \ref{Thm SC Red} on supercuspidal reduction}
		
		We now apply the previous results on finite groups of Lie type to representations of the parahoric subgroups of a $p$-adic group. For this, we show that the inflation induces an equivalence of categories between (certain summand of) the category of representations of a finite reductive group and the corresponding parahoric subgroup (See Subsection \ref{Subsection_inflation}). 
		
		Let us get back to the notation at the beginning of this chapter.
		
		Let $G$ be a split reductive group scheme over $\mathbb{Z}$, which is simply connected. Let $F$ be a non-archimedean local field, with ring of integers $\mathcal{O}_F$ and residue field $k_F \cong \mathbb{F}_q$ of residue characteristic $p$. Let $x$ be a vertex of the Bruhat-Tits building $\mathcal{B}(G, F)$, $G_x$ the parahoric subgroup associated to $x$, $G_x^+$ its pro-unipotent radical. Recall that $\overline{G_x}:=G_x/G_x^+$ is a generalized Levi subgroup of $G(k_F)$ with root system $\Phi_x$, see \cite[Theorem 3.17]{rabinoff2003bruhat}.
		
		Let $\Lambda=\overline{\mathbb{Z}_\ell}$, with $\ell \neq p$. Let $\rho \in \Rep_{\Lambda}(G_x)$ be an irreducible representation of $G_x$, which is trivial on $G_x^+$ and whose reduction to the finite group of Lie type $\overline{G_x}=G_x/G_x^+$ is regular supercuspidal. 
		%We make this a definition for later use.
		
		%	\begin{definition}
			%		Let $\rho \in \Rep_{\Lambda}(G_x)$. We say $\rho$ \textbf{has cuspidal reduction} (resp. \textbf{has regualr cuspidal reduction}), if $\rho$ is trivial on $G_x^+$ and whose reduction to the finite group of Lie type $\overline{G_x}=G_x/G_x^+$ is cuspidal (resp. regular cuspidal). Let's denote the reduction of $\rho$ modulo $G_x^+$ by $\overline{\rho} \in \Rep_{\Lambda}(\overline{G_x})$.
			%	\end{definition}
		
		In other words, we start with an irreducible representation $\rho \in \Rep_{\Lambda}(G_x)$ which has regular supercuspidal reduction. Let $\mathcal{B}_{x,1}$ be the ($\overline{\mathbb{Z}_{\ell}}$-)block of $\Rep_{\Lambda}(G_x)$ containing $\rho$. We can now prove Theorem \ref{Thm SC Red}, which we restate as follows.
		
		\begin{theorem} \label{Thm SC Red restate}
			Let $\rho \in \Rep_{\Lambda}(G_x)$ be an irreducible representation of $G_x$, which has regular supercuspidal reduction. Let $\mathcal{B}_{x,1}$ be the $\overline{\mathbb{Z}_{\ell}}$-block of $\Rep_{\Lambda}(G_x)$ containing $\rho$. Then any $\rho' \in \mathcal{B}_{x,1}$ has supercuspidal reduction.
		\end{theorem}
		
		\begin{proof}
			Let $\overline{\rho} \in \Rep_{\Lambda}(\overline{G_x})$ be the reduction of $\rho$ modulo $G_x^+$. $\overline{\rho}$ is irreducible (since $\rho$ is) and regular supercuspidal by assumption, so it is of the form $\pm R_T^G(\theta_s)$, for some strongly regular semisimple $\ell'$-element $s$ of the finite dual group $\overline{G_x}^*$ (See Definition \ref{Def regular supercuspidal}.). (\textcolor{red}{problem hyperlink}) 
			
			Let $\Rep_{\Lambda}(G_x)_0$ be the full subcategory of $\Rep_{\Lambda}(G_x)$ consists of representations of $G_x$ that are trivial on $G_x^+$. The key observation is that $\Rep_{\Lambda}(G_x)_0$ is a summand (as abelian category) of $\Rep_{\Lambda}(G_x)$ (See Lemma \ref{Lem Summand}).
			
			Then since $\rho \in \Rep_{\Lambda}(G_x)_0$, its block $\mathcal{B}_{x,1}$ is a summand of $\Rep_{\Lambda}(G_x)_0$.
			
			On the other hand, notice that the inflation induces an equivalence of categories between $\Rep_{\Lambda}(\overline{G_x})$ and $\Rep_{\Lambda}(G_x)_0$, with inverse the reduction modulo $G_x^+$.
			contained
			So the blocks of $\Rep_{\Lambda}(\overline{G_x})$ and $\Rep_{\Lambda}(G_x)_0$ are in one-one correspondence. Let $\mathcal{A}_{x,1}$ be the corresponding block of $\Rep_{\Lambda}(\overline{G_x})$ to $\mathcal{B}_{x,1}$. Then $\mathcal{A}_{x,1}$ is in the regular supercuspidal block $\mathcal{A}_s$ corresponding to $s$ (recall $\overline{\rho}=\pm R_T^G(\theta_s)$). By Theorem \ref{Theorem Pure SC}, $\mathcal{A}_s$ consists purely of supercuspidal representation. Therefore, $\mathcal{B}_{x,1}$ consists purely of representations that have supercuspidal reductions. 
		\end{proof}
		
		
     \subsection{Inflation induces an equivalence}  \label{Subsection_inflation} 		
		
		
		\begin{lemma}\label{Lem Summand}
			Let $\Rep_{\Lambda}(G_x)_0$ be the full subcategory of $\Rep_{\Lambda}(G_x)$ consists of representations of $G_x$ that are trivial on $G_x^+$. Then $\Rep_{\Lambda}(G_x)_0$ is a summand as abelian category of $\Rep_{\Lambda}(G_x)$.
		\end{lemma}
		
		\begin{remark}
			A similar proof as \cite[Appendix]{dat2009finitude} should work. Nevertheless, I include here an alternative computational proof. 
		\end{remark}
		
		\begin{proof}
			Note $G_x^+$ is pro-$p$ (See \cite[II.5.2.(b)]{vigneras1996representations}), in particular, it has pro-order invertible in $\Lambda$. So we have a normalized Haar measure $\mu$ on $G_x$ such that $\mu(G_x^+)=1$ (See \cite[I.2.4]{vigneras1996representations}). The characteristic function $e:=1_{G_x^+}$ is an idempotent of the Hecke algebra $\mathcal{H}_{\Lambda}(G_x)$ under convolution with respect to the Haar measure $\mu$. We shall show that $e=1_{G_x^+}$ cuts out $\Rep_{\Lambda}(G_x)_0$ as a summand of $\Rep_{\Lambda}(G_x) \cong \mathcal{H}_{\Lambda}(G_x)\Modl$.
			
			Let's first check that $e=1_{G_x^+}$ is central. This can be done by an explicit computation. Recall that we have a descending filtration $\{G_{x,r} | r\in \mathbb{R}_{>0}\}$ of $G_x$ such that 
			\begin{enumerate}
				\item $\forall r \in \mathbb{R}_{>0}, G_{x,r}$ is an open compact pro-$p$ subgroup of $G_x$.
				\item $\forall r \in \mathbb{R}_{>0}, G_{x,r}$ is a normal subgroup of $G_x$.
				\item $G_{x,r}$ form a neighborhood basis of $1$ inside $G_x$. 
			\end{enumerate}
			(See \cite[II.5.1]{vigneras1996representations}.) Therefore, to check $e \ast f=f \ast e$, for all $f \in \mathcal{H}_{\Lambda}(G_x)$, it suffices to check for all $f$ of the form $1_{gG_{x,r}}$, the characteristic function of the (both left and right) coset $gG_{x,r}$($=G_{x,r}g$, by normality) for some $g \in G(F)$ and $r \in \mathbb{R}_{>0}$. Indeed, one can compute that $(e \ast 1_{gG_{x,r}})(y)=\mu(G_x^+\cap G_{x,r}yg^{-1})$ and that $(1_{gG_{x,r}} \ast e)(y)=\mu(gG_{x,r}\cap yG_x^+)$, for any $y \in G_x$. Note that $G_{x,r} \subseteq G_x^+$, we get that $\mu(G_x^+\cap G_{x,r}yg^{-1})=\mu(G_{x,r})$ if $yg^{-1} \in G_x^+$ and $0$ otherwise. Same for $\mu(gG_{x,r}\cap yG_x^+)$. Therefore, $e$ is central.
			%(\textcolor{red}{See ? for details.})
			
			
			Next, under the isomorphism $\Rep_{\Lambda}(G_x) \cong \mathcal{H}_{\Lambda}(G_x)\Modl$, $\Rep_{\Lambda}(G_x)_0$ corresponds to the summand $\mathcal{H}_{\Lambda}(G_x, G_x^+)\Modl=e\mathcal{H}_{\Lambda}(G_x)e\Modl$ corresponding to the central idempotent $e:=1_{G_x^+} \in \mathcal{H}_{\Lambda}(G_x)$ of $\mathcal{H}_{\Lambda}(G_x)\Modl$. 
			
			Finally, note that $G_x$ is compact, so its Hecke algebra $\mathcal{H}(G_x)$ is unital with unit $1$ the normalized characteristic function of $G_x$. Hence
			$$\mathcal{H}_{\Lambda}(G_x)\Modl \cong e\mathcal{H}_{\Lambda}(G_x)e\Modl \oplus (1-e)\mathcal{H}_{\Lambda}(G_x)(1-e)\Modl.$$
			Therefore, $\Rep_{\Lambda}(G_x)_0 \cong e\mathcal{H}_{\Lambda}(G_x)e\Modl$ is a summand of $\Rep_{\Lambda}(G_x) \cong \mathcal{H}_{\Lambda}(G_x)\Modl$. 
		\end{proof}
		
		\begin{lemma}\label{Lemma A to B}
			The inflation induces an equivalence of categories between $\Rep_{\Lambda}(\overline{G_x})$ and $\Rep_{\Lambda}(G_x)_0$. In particular, let $\rho$ as in Theorem \ref{Thm SC Red restate} and let $\mathcal{A}_{x,1}$ be the block of $\Rep_{\Lambda}(\overline{G_x})$ containing $\overline{\rho}$, then the inflation induces an equivalence of categories 
			$$\mathcal{A}_{x,1} \cong \mathcal{B}_{x,1}.$$
		\end{lemma}
		
		\begin{proof}
			The inverse functor is given by the reduction modulo $G_x^+$. One could check by hand that they are equivalences of categories.
		\end{proof}
		
		
		
		
		
		
		
		
		
		
		\section{$\Hom$ between compact inductions}\label{Sec Pf Thm Hom}
		
		Let's now prove Theorem \ref{Thm Hom} which computes the $\Hom$ between compact inductions of $\rho_1$ and $\rho_2$, assuming one of them has supercuspidal reduction.
		
		\begin{proof}[Proof of Theorem \ref{Thm Hom}]
			\begin{equation*}
				\begin{aligned}
					&\Hom_G(\cInd_{G_x}^{G(F)}\rho_1, \cInd_{G_y}^{G(F)}\rho_2)\\
					=\;&\Hom_{G_x}\left(\rho_1,(\cInd_{G_y}^{G(F)}\rho_2)|_{G_x}\right)\\
					=\;& \Hom_{G_x}\left(\rho_1, \bigoplus_{g \in {G_y\backslash G(F)/G_x}}\cInd_{G_x \cap g^{-1}G_yg}^{G_x}\rho_2(g-g^{-1})\right)
				\end{aligned}
			\end{equation*}
			
			Recall that $g^{-1}G_yg=G_{g^{-1}.y}$. So it suffices to show that for $g \in G(F)$ with $G_x \cap g^{-1}G_yg \neq G_x$, or equivalently, for $g \in G(F)$ with $g.x \neq y$ (since $x$ and $y$ are vertices), it holds that
			$$\Hom_{G_x}\left(\rho_1, \cInd_{G_x \cap g^{-1}G_yg}^{G_x}\rho_2(g-g^{-1})\right)=0.$$
			
			Note $G_x/(G_x \cap g^{-1}G_yg)$ is compact, hence $\cInd_{G_x \cap g^{-1}G_yg}^{G_x}=\operatorname{Ind}_{G_x \cap g^{-1}G_yg}^{G_x}$, and we have Frobenius reciprocity in the other direction
			$$\Hom_{G_x}\left(\rho_1, \cInd_{G_x \cap g^{-1}G_yg}^{G_x}\rho_2(g-g^{-1})\right) \cong \Hom_{G_x \cap g^{-1}G_yg}\left(\rho_1, \rho_2(g-g^{-1})\right).$$
			
			So it suffices to show that for $g \in G(F)$ with $g.x \neq y$,
			$$\Hom_{G_x \cap g^{-1}G_yg}\left(\rho_1, \rho_2(g-g^{-1})\right)=0.$$
			Note now this expression is symmetric with respect to $\rho_1$ and $\rho_2$, so is the following argument.
			
			First, if $\rho_2$ has supercuspidal reduction (denoted $\overline{\rho_2}$),
			\begin{align*}    	
				& \Hom_{G_x \cap g^{-1}G_yg}\left(\rho_1, \rho_2(g-g^{-1})\right) \\
				=\;& \Hom_{G_x \cap G_{g^{-1}.y}}\left(\rho_1, \rho_2(g-g^{-1})\right) \\
				\subseteq\;& \Hom_{G_x^+ \cap G_{g^{-1}.y}}\left(\rho_1, \rho_2(g-g^{-1})\right) && %\text{By \eqref{eq:1}}
				\\
				=\;& \Hom_{G_x^+ \cap G_{g^{-1}.y}}(1^{\oplus d_1}, \rho_2(g-g^{-1})) && \text{$\rho_1$ is trivial on $G_x^+$ }\\
				=\;& \Hom_{G_{g.x}^+ \cap G_y}(1^{\oplus d_1}, \rho_2) && \text{Conjugate by $g^{-1}$}\\
				=\;& \Hom_{U_y(g.x)}(1^{\oplus d_1}, \overline{\rho_2}) && \text{Reduction modulo $G_y^+$. See below.}\\
				=\;& 0 && \text{$\overline{\rho_2}$ is supercuspidal. See below.}
			\end{align*}
			
			The last two equations need some explanation. 
			
			The former one uses the following consequence from Bruhat-Tits theory: If $x_1$ and $x_2$ are two different vertices of the Bruhat-Tits building, then $\overline{G_{x_i}}:=G_{x_i}/G_{x_i}^+$ is a generalized Levi subgroup of $\overline{G}=G(\mathbb{F}_q)$, for $i=1, 2$. Moreover, $G_{x_1} \cap G_{x_2}$ projects onto a proper parabolic subgroup $P_{x_1}(x_2)$ of $\overline{G_{x_1}}$ under the reduction map $G_{x_1} \to \overline{G_{x_1}}$. And $G_{x_1} \cap G_{x_2}^+$ projects onto $U_{x_1}(x_2)$, the unipotent radical of $P_{x_1}(x_2)$, under the reduction map $G_{x_1} \to \overline{G_{x_1}}$. For details, see Lemma \ref{Lem Passage to Residue Field} below. Note that the assumption of Lemma \ref{Lem Passage to Residue Field} is satisfied since without loss of generality we may assume $x_1=x$ and $x_2=y$ lies in the closure of a common alcove (since $G$ acts simply transitively on the set of alcoves).
			
			The latter one uses that for a supercuspidal representation $\rho$ of a finite group of Lie type $\Gamma$, 
			$$\Hom_U(1, \rho|_U)=\Hom_U(\rho|_U, 1)=0,$$
			for the unipotent radical $U$ of $P$, where $P$ is any proper parabolic subgroup of $\Gamma$. For details, see Lemma \ref{Lem Hom_U(1_U, SC)} below.
			
			Symmetrically, a similar argument works if $\rho_1$ has supercuspidal reduction. Indeed, if $\rho_1$ has supercuspidal reduction (denoted $\overline{\rho_1}$),
			\begin{align*}    	
				& \Hom_{G_x \cap g^{-1}G_yg}\left(\rho_1, \rho_2(g-g^{-1})\right) \\
				=\;& \Hom_{gG_xg^{-1} \cap G_y}\left(\rho_1(g^{-1}-g), \rho_2\right) && \text{Conjugate by $g^{-1}$}\\ 
				\subseteq\;& \Hom_{gG_xg^{-1} \cap G_y^+}\left(\rho_1(g^{-1}-g), \rho_2\right) && %\text{By \eqref{eq:1}}
				\\
				=\;& \Hom_{gG_xg^{-1} \cap G_y^+}(\rho_1(g^{-1}-g), 1^{\oplus d_2}) && \text{$\rho_2$ is trivial on $G_y^+$ }\\
				=\;& \Hom_{G_x \cap g^{-1}G_y^+g}(\rho_1, 1^{\oplus d_2}) && \text{Conjugate by $g$}\\
				=\;& \Hom_{G_x \cap G_{g^{-1}.y}^+}(\rho_1, 1^{\oplus d_2}) && \\
				=\;& \Hom_{U_x(g^{-1}.y)}(\overline{\rho_1}, 1^{\oplus d_2}) && \text{Reduction modulo $G_x^+$}\\
				=\;& 0 && \text{$\overline{\rho_1}$ is supercuspidal. }
			\end{align*}
			
		\end{proof}
		
		
		
		\begin{lemma}\label{Lem Passage to Residue Field}
			Let $x_1$ and $x_2$ be two points of the Bruhat-Tits building $\mathcal{B}(G, F)$. Assume they lie in the closure of a same alcove.
			\begin{enumerate}
				\item[(i)]   The image of $G_{x_1} \cap G_{x_2}$ in $\overline{G_{x_1}}$ is a parabolic subgroup of $\overline{G_{x_1}}$. Let's denote it by $P_{x_1}(x_2)$. Moreover, the image of $G_{x_1} \cap G_{x_2}^+$ in $\overline{G_{x_1}}$ is the unipotent radical of $P_{x_1}(x_2)$. Let's denote it by $U_{x_1}(x_2)$.
				\item[(ii)] 	Assume moreover that $x_1$ and $x_2$ are two different vertices of the building. Then $P_{x_1}(x_2)$ is a proper parabolic subgroup of $\overline{G_{x_1}}$.
			\end{enumerate}
		\end{lemma}
		
		\begin{proof}
			(i) is \cite[II.5.1.(k)]{vigneras1996representations}.
			
			Let's prove (ii). It suffices to show that $G_{x_1} \neq G_{x_2}$. Assume otherwise that $G_{x_1}=G_{x_2}$, then $x_1$ and $x_2$ lie in the same facet, which contradicts with the assumption that $x_1$ and $x_2$ are two different vertices.
		\end{proof}
		
		\begin{lemma}\label{Lem Hom_U(1_U, SC)}
			Let $\overline{\rho}$ be a supercuspidal representation of a finite group of Lie type $\Gamma$. Let $P$ be a proper parabolic subgroup of $\Gamma$, with unipotent radical $U$. Then
			$$Hom_U(1_U, \overline{\rho})=Hom_U(\overline{\rho}, 1_U)=0.$$
		\end{lemma}
		
		\begin{proof}
			$\Hom_U(\overline{\rho}|_U, 1_U)=\Hom_{\Gamma}(\overline{\rho}, Ind_P^{\Gamma}(\sigma))=0$, where $\sigma=Ind_U^P(1_U)$. The last equality holds because $\overline{\rho}$ is assumed to be supercuspidal. A similar argument shows that $Hom_U(1_U, \overline{\rho})=0$.  
			
%			Moreover, since $U$ is a successive extension of additive groups, $U$ is of order a power of $p$. In particular, $\ell$ does not divide the order of $U$. Hence the category of representations of $U$ with $\Lambda=\overline{\mathbb{Z}_{\ell}}$-coefficients is semisimple \textcolor{red}{No! Even $\overline{\mathbb{Z}_{\ell}}$ is not semisimple as a $\overline{\mathbb{Z}_{\ell}}$ module.}, and
%			$$\Hom_U(1_U, \overline{\rho})=\Hom_U(\overline{\rho}, 1_U)=0.$$
%			\textcolor{red}{No! $\Hom_1(\overline{\mathbb{Z}_\ell}, \overline{\mathbb{F}_{\ell}}) \neq \Hom_1(\overline{\mathbb{F}_{\ell}}, \overline{\mathbb{Z}_{\ell}})$, where $1$: trivial group.}
		\end{proof}
		
		
		
		
		\section{$\Pi_{x,1}$ is a projective generator}\label{Section projective generator}
		
		In this subsection, we prove Theorem \ref{Thm Proj}: $\Pi_{x,1}$ is a projective generator of $\mathcal{C}_{x,1}$. Before doing this, let us recall the setting. Fix a vertex $x$ of the building of $G$. Let $\rho \in \Rep_{\Lambda}(G_x)$ which is trivial on $G_x^+$ and whose reduction to $\overline{G_x}=G_x/G_x^+$ is regular supercuspidal, $\pi=\cInd_{G_x}^{G(F)}\rho$ as before. Let $\mathcal{B}_{x,1}$ be the block of $\Rep_{\Lambda}(G_x)$ containing $\rho$, and $\mathcal{C}_{x,1}$ the block of $\Rep_{\Lambda}(G(F))$ containing $\pi$. 
		
		Let $V$ be the set of equivalence classes of vertices of the Bruhat-Tits building $\mathcal{B}(G, F)$ up to $G(F)$-action. For $y \in V$, let $\sigma_y:=\cInd_{G_y^+}^{G_y}\Lambda$. Let $\Pi:=\bigoplus_{y \in V}\Pi_y$ where $\Pi_y:=\cInd_{G_y^+}^{G(F)}\Lambda$. Then $\Pi$ is a projective generator of the category of depth-zero representations $\Rep_{\Lambda}(G(F))_0$, see \cite[Appendix]{dat2009finitude}. Let $\sigma_{x,1}:=(\sigma_x)|_{\mathcal{B}_{x,1}} \in \mathcal{B}_{x,1} \xhookrightarrow{summand} \Rep_{\Lambda}(G_x)$ be the $\mathcal{B}_{x,1}$-summand of $\sigma_x$. And let $\Pi_{x,1}:=\cInd_{G_x}^{G(F)}\sigma_{x,1}$.
		
		Let's summarize the setting in the following diagram.
		
		\begin{tikzcd}
			{\Rep_{\Lambda}(G_x)} & {\Rep_{\Lambda}(G(F))} \\
			{\Rep_{\Lambda}(G_x)_0} & {\Rep_{\Lambda}(G(F))_0} \\
			{\mathcal{B}_{x,1}} & {\mathcal{C}_{x,1}} \\
			{\text{block of } \rho} & {\text{block of }\pi}
			\arrow[from=2-1, to=2-2]
			\arrow["{\cInd_{G_x}^{G(F)}}", from=1-1, to=1-2]
			\arrow["\subseteq"{description}, sloped, draw=none, from=2-1, to=1-1]
			\arrow["\subseteq"{description}, sloped, draw=none, from=3-1, to=2-1]
			\arrow["\subseteq"{description}, sloped, draw=none, from=3-2, to=2-2]
			\arrow["\subseteq"{description}, sloped, draw=none, from=2-2, to=1-2]
			\arrow["{=:}"{description}, sloped, draw=none, from=4-1, to=3-1]
			\arrow["{:=}"{description}, sloped, draw=none, from=3-2, to=4-2]
			\arrow[from=3-1, to=3-2]
		\end{tikzcd}
		
		\begin{theorem}
			$\Pi_{x,1}=\cInd_{G_x}^{G(F)}\sigma_{x,1}$ is a projective generator of $\mathcal{C}_{x,1}$.
		\end{theorem}
		
		\begin{proof}
			First, let $\Rep_{\Lambda}(G_x)_0$ be the full subcategory of $\Rep_{\Lambda}(G_x)$ consisting of representations that are trivial on $G_x^+$ (Don't confuse with $\Rep_{\Lambda}(G(F))_0$, the depth-zero category of $G$). Note $\Rep_{\Lambda}(G_x)_0$ is a summand of $\Rep_{\Lambda}(G_x)$ (see Lemma \ref{Lem Summand}).
			
			Second, note that $\Rep_{\Lambda}(G_x)_0 \cong \Rep_{\Lambda}(\overline{G_x})$. We may assume $$\Rep_{\Lambda}(G_x)_0=\mathcal{B}_{x,1} \oplus ... \oplus \mathcal{B}_{x,m}$$
			is its block decomposition. So that $\sigma_x=\sigma_{x,1}\oplus...\oplus\sigma_{x,m}$ accordingly. Write $\sigma_x^1:=\sigma_{x,2}\oplus...\oplus\sigma_{x,m}$. Then $\sigma_x=\sigma_{x,1} \oplus \sigma_x^1$, and $\Pi_x=\Pi_{x,1} \oplus \Pi_x^1$ accordingly, where $\Pi_x^1:=\cInd_{G_x}^{G(F)}\sigma_x^1$. And
			$$\Pi=\Pi_{x,1}\oplus \Pi_x^1 \oplus \Pi^x,$$
			where $\Pi^x:=\bigoplus_{y \in V, y \neq x}\Pi_y$. Let $\Pi^{x,1}:=\Pi_x^1 \oplus \Pi^x$, then we have
			$$\Pi=\Pi_{x,1} \oplus \Pi^{x,1}.$$
			
			Recall that $\Pi$ is a projective generator of the category of depth-zero representations $\Rep_{\Lambda}(G(F))_0$. This implies that 
			$$\Hom_G(\Pi, -): \Rep_{\Lambda}(G(F))_0 \to \Modr\End_G(\Pi)$$
			is an equivalence of categories. See \cite[Lemma 22]{bernsteindraft}.
			
			Next, it is not hard to see that Theorem \ref{Thm Hom} implies that 
			$$\Hom_G(\Pi_{x,1}, \Pi^{x,1})=\Hom_G(\Pi^{x,1}, \Pi_{x,1})=0,$$
			see Lemma \ref{Lem Ortho}. This implies that $$\Modr\End_G(\Pi) \cong \Modr\End_G(\Pi_{x,1}) \oplus \Modr\End_G(\Pi^{x,1})$$ is an equivalence of categories.
			
			Now we can combine the above to show that $\Pi^{x,1}$ does not interfere with $\Pi_{x,1}$, i.e.,
			$$\Hom_G(\Pi^{x,1}, X)=0,$$
			for any object $X \in \mathcal{C}_{x,1}$ (see Importent Lemma \ref{Lem Gen}).
			
			However, since $\Pi$ is a projective generator of $\Rep_{\Lambda}(G(F))_0$, we have
			$$\Hom_G(\Pi, X) \neq 0,$$
			for any $X \in \mathcal{C}_{x,1}$. This together with the last paragraph implies that 
			$$\Hom_G(\Pi_{x,1}, X) \neq 0,$$
			for any $X \in \mathcal{C}_{x,1}$, i.e. $\Pi_{x,1}$ is a generator of $\mathcal{C}_{x,1}$.
			
			Finally, note $\Pi_{x,1}$ is projective in $\Rep_{\Lambda}(G(F))_0$ since it is a summand of the projective object $\Pi$. Hence $\Pi_{x,1}$ is projective in $\mathcal{C}_{x,1}$. This together with the last paragraph implies that $\Pi_{x,1}$ is a projective generator of $\mathcal{C}_{x,1}$.
			
			
		\end{proof}
		
		
		
		
		
		\begin{lemma}\label{Lem Ortho}
			$$\Hom_G(\Pi_{x,1}, \Pi^{x,1})=\Hom_G(\Pi^{x,1}, \Pi_{x,1})=0.$$
		\end{lemma}
		
		\begin{proof}
			Recall that
			$\Pi^{x,1}:=\Pi_x^1 \oplus \Pi^x$.
			
			First, we compute
			$$\Hom_G(\Pi_{x,1}, \Pi_x^1)=\Hom_{G_x}(\sigma_{x,1}, \sigma_x^1)=0,$$
			where the first equality is the first case of Theorem \ref{Thm Hom} (note $\sigma_{x,1} \in \mathcal{B}_{x,1}$, hence has supercuspidal reduction by Theorem \ref{Thm SC Red}, and hence the condition of Theorem \ref{Thm Hom} is satisfied), and the second equality is because $\sigma_{x,1}$ and $\sigma_x^1$ lie in different blocks of $\Rep_{\Lambda}(G_x)$ by definition.
			
			Second, recall that $\Pi_{x,1}=\cInd_{G_x}^{G(F)}\sigma_{x,1}$ with $\sigma_{x,1}$ having supercuspidal reduction, and $\Pi_y=\cInd_{G_y}^{G(F)}\sigma_y$. We compute 
			$$\Hom_G(\Pi_{x,1}, \Pi^x)=\bigoplus_{y \in V, y \neq x}\Hom_G(\Pi_{x,1}, \Pi_y)=0,$$
			by the second case of Theorem \ref{Thm Hom}.
			
			Combining the above three paragraphs, we get $\Hom_G(\Pi_{x,1}, \Pi^{x,1})=0$.
			
			A same argument shows that $\Hom_G(\Pi^{x,1}, \Pi_{x,1})=0$.
		\end{proof}
		
		\begin{lemma}[Important Lemma]\label{Lem Gen}
			$\Hom_G(\Pi^{x,1}, X)=0,$
			for any object $X \in \mathcal{C}_{x,1}$.
		\end{lemma}
		
		\begin{proof}
			Recall that 
			$$\Hom_G(\Pi, -): \Rep_{\Lambda}(G(F))_0 \to \Modr\End_G(\Pi) \cong \Modr\End_G(\Pi_{x,1}) \oplus \Modr\End_G(\Pi^{x,1})$$ 
			is an equivalence of categories. It is even an equivalence of abelian categories since $\Hom_G(\Pi, -)$ is exact and commutes with direct product. Hence the image of $\mathcal{C}_{x,1}$ must be indecomposable as $\mathcal{C}_{x,1}$ is indecomposable, i.e., 
			$$\Hom_G(\Pi, -)=\Hom_G(\Pi_{x,1}, -) \oplus \Hom_G(\Pi^{x,1}, -)$$
			can map $\mathcal{C}_{x,1}$ nonzeroly to only one of $\Modr\End_G(\Pi_{x,1})$ and $\Modr\End_G(\Pi^{x,1})$ (See the diagram below). 
			
			\begin{tikzcd}
				{\Rep_{\Lambda}(G(F))_0} &&&& {\Modr \End_G(\Pi)} \\
				\\
				{\mathcal{C}_{x,1}} &&&& {\Modr \End_G(\Pi_{x,1}) \oplus \Modr \End_G(\Pi^{x,1})}
				\arrow["{\Hom_G(\Pi, -)}", from=1-1, to=1-5]
				\arrow["{\Hom_G(\Pi_{x,1}, -) \oplus \Hom_G(\Pi^{x,1}, -)}", from=3-1, to=3-5]
				\arrow["\subseteq", sloped, from=3-1, to=1-1]
				\arrow["\cong", sloped, from=3-5, to=1-5]
			\end{tikzcd}
			
			Then it must be $\Modr\End_G(\Pi_{x,1})$ (that $\Hom_G(\Pi, -)$ maps $\mathcal{C}_{x,1}$ nonzeroly to) since 
			$$\Hom_G(\Pi_{x,1}, \pi)=\Hom_{G_x}(\sigma_{x,1}, \rho)=\Hom_{G_x}(\sigma_x, \rho) \neq 0.$$
			In other words, $\Hom_G(\Pi^{x,1}, -)$ is zero on $\mathcal{C}_{x,1}$.
			
		\end{proof}
		
		
		\section{Application: description of the block $\Rep_{\Lambda}(G(F))_{[\pi]}$}\label{Section rep application}
		
		Recall we denote $\mathcal{A}_{x,1}=\Rep_{\Lambda}(\overline{G_x})_{[\overline{\rho}]}$, $\mathcal{B}_{x,1}=\Rep_{\Lambda}(G_x)_{[\rho]}$, and $\mathcal{C}_{x,1}=\Rep_{\Lambda}(G(F))_{[\pi]}$.
		
		We have proven that the inflation along $G_x \to \overline{G_x}$ induces an equivalence of categories 
		$$\mathcal{A}_{x,1} \cong \mathcal{B}_{x,1},$$
		see Lemma \ref{Lemma A to B}. And we have also proven that the compact induction induces an equivalence of categories
		$$\cInd_{G_x}^{G(F)}: \mathcal{B}_{x,1} \cong \mathcal{C}_{x,1}.$$
		
		Hence $\mathcal{C}_{x,1} \cong \mathcal{A}_{x,1}$, where the latter is isomorphic to the block of a finite torus via Broué's equivalence \ref{Thm Broué}.
		
		We will see in the example (See Chapter \ref{Chapter GL_n}) of $GL_n$ that (up to central characters) such a block of a finite torus corresponds to $\QCoh(\mu)$, where $\mu$ is the group scheme of roots of unity appearing in the computation of the $L$-parameter side (See Theorem \ref{Thm X/G}).
		
		\bibliographystyle{alpha2}
		\bibliography{reference}
		
	\end{document}
	